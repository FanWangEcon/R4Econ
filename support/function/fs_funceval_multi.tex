% Options for packages loaded elsewhere
\PassOptionsToPackage{unicode}{hyperref}
\PassOptionsToPackage{hyphens}{url}
\PassOptionsToPackage{dvipsnames,svgnames*,x11names*}{xcolor}
%
\documentclass[
]{article}
\usepackage{lmodern}
\usepackage{amssymb,amsmath}
\usepackage{ifxetex,ifluatex}
\ifnum 0\ifxetex 1\fi\ifluatex 1\fi=0 % if pdftex
  \usepackage[T1]{fontenc}
  \usepackage[utf8]{inputenc}
  \usepackage{textcomp} % provide euro and other symbols
\else % if luatex or xetex
  \usepackage{unicode-math}
  \defaultfontfeatures{Scale=MatchLowercase}
  \defaultfontfeatures[\rmfamily]{Ligatures=TeX,Scale=1}
\fi
% Use upquote if available, for straight quotes in verbatim environments
\IfFileExists{upquote.sty}{\usepackage{upquote}}{}
\IfFileExists{microtype.sty}{% use microtype if available
  \usepackage[]{microtype}
  \UseMicrotypeSet[protrusion]{basicmath} % disable protrusion for tt fonts
}{}
\makeatletter
\@ifundefined{KOMAClassName}{% if non-KOMA class
  \IfFileExists{parskip.sty}{%
    \usepackage{parskip}
  }{% else
    \setlength{\parindent}{0pt}
    \setlength{\parskip}{6pt plus 2pt minus 1pt}}
}{% if KOMA class
  \KOMAoptions{parskip=half}}
\makeatother
\usepackage{xcolor}
\IfFileExists{xurl.sty}{\usepackage{xurl}}{} % add URL line breaks if available
\IfFileExists{bookmark.sty}{\usepackage{bookmark}}{\usepackage{hyperref}}
\hypersetup{
  pdftitle={DPLYR Evaluate Function with N Arrays of Inputs J times},
  pdfauthor={Fan Wang},
  colorlinks=true,
  linkcolor=Maroon,
  filecolor=Maroon,
  citecolor=Blue,
  urlcolor=blue,
  pdfcreator={LaTeX via pandoc}}
\urlstyle{same} % disable monospaced font for URLs
\usepackage[margin=1in]{geometry}
\usepackage{color}
\usepackage{fancyvrb}
\newcommand{\VerbBar}{|}
\newcommand{\VERB}{\Verb[commandchars=\\\{\}]}
\DefineVerbatimEnvironment{Highlighting}{Verbatim}{commandchars=\\\{\}}
% Add ',fontsize=\small' for more characters per line
\usepackage{framed}
\definecolor{shadecolor}{RGB}{248,248,248}
\newenvironment{Shaded}{\begin{snugshade}}{\end{snugshade}}
\newcommand{\AlertTok}[1]{\textcolor[rgb]{0.94,0.16,0.16}{#1}}
\newcommand{\AnnotationTok}[1]{\textcolor[rgb]{0.56,0.35,0.01}{\textbf{\textit{#1}}}}
\newcommand{\AttributeTok}[1]{\textcolor[rgb]{0.77,0.63,0.00}{#1}}
\newcommand{\BaseNTok}[1]{\textcolor[rgb]{0.00,0.00,0.81}{#1}}
\newcommand{\BuiltInTok}[1]{#1}
\newcommand{\CharTok}[1]{\textcolor[rgb]{0.31,0.60,0.02}{#1}}
\newcommand{\CommentTok}[1]{\textcolor[rgb]{0.56,0.35,0.01}{\textit{#1}}}
\newcommand{\CommentVarTok}[1]{\textcolor[rgb]{0.56,0.35,0.01}{\textbf{\textit{#1}}}}
\newcommand{\ConstantTok}[1]{\textcolor[rgb]{0.00,0.00,0.00}{#1}}
\newcommand{\ControlFlowTok}[1]{\textcolor[rgb]{0.13,0.29,0.53}{\textbf{#1}}}
\newcommand{\DataTypeTok}[1]{\textcolor[rgb]{0.13,0.29,0.53}{#1}}
\newcommand{\DecValTok}[1]{\textcolor[rgb]{0.00,0.00,0.81}{#1}}
\newcommand{\DocumentationTok}[1]{\textcolor[rgb]{0.56,0.35,0.01}{\textbf{\textit{#1}}}}
\newcommand{\ErrorTok}[1]{\textcolor[rgb]{0.64,0.00,0.00}{\textbf{#1}}}
\newcommand{\ExtensionTok}[1]{#1}
\newcommand{\FloatTok}[1]{\textcolor[rgb]{0.00,0.00,0.81}{#1}}
\newcommand{\FunctionTok}[1]{\textcolor[rgb]{0.00,0.00,0.00}{#1}}
\newcommand{\ImportTok}[1]{#1}
\newcommand{\InformationTok}[1]{\textcolor[rgb]{0.56,0.35,0.01}{\textbf{\textit{#1}}}}
\newcommand{\KeywordTok}[1]{\textcolor[rgb]{0.13,0.29,0.53}{\textbf{#1}}}
\newcommand{\NormalTok}[1]{#1}
\newcommand{\OperatorTok}[1]{\textcolor[rgb]{0.81,0.36,0.00}{\textbf{#1}}}
\newcommand{\OtherTok}[1]{\textcolor[rgb]{0.56,0.35,0.01}{#1}}
\newcommand{\PreprocessorTok}[1]{\textcolor[rgb]{0.56,0.35,0.01}{\textit{#1}}}
\newcommand{\RegionMarkerTok}[1]{#1}
\newcommand{\SpecialCharTok}[1]{\textcolor[rgb]{0.00,0.00,0.00}{#1}}
\newcommand{\SpecialStringTok}[1]{\textcolor[rgb]{0.31,0.60,0.02}{#1}}
\newcommand{\StringTok}[1]{\textcolor[rgb]{0.31,0.60,0.02}{#1}}
\newcommand{\VariableTok}[1]{\textcolor[rgb]{0.00,0.00,0.00}{#1}}
\newcommand{\VerbatimStringTok}[1]{\textcolor[rgb]{0.31,0.60,0.02}{#1}}
\newcommand{\WarningTok}[1]{\textcolor[rgb]{0.56,0.35,0.01}{\textbf{\textit{#1}}}}
\usepackage{graphicx,grffile}
\makeatletter
\def\maxwidth{\ifdim\Gin@nat@width>\linewidth\linewidth\else\Gin@nat@width\fi}
\def\maxheight{\ifdim\Gin@nat@height>\textheight\textheight\else\Gin@nat@height\fi}
\makeatother
% Scale images if necessary, so that they will not overflow the page
% margins by default, and it is still possible to overwrite the defaults
% using explicit options in \includegraphics[width, height, ...]{}
\setkeys{Gin}{width=\maxwidth,height=\maxheight,keepaspectratio}
% Set default figure placement to htbp
\makeatletter
\def\fps@figure{htbp}
\makeatother
\setlength{\emergencystretch}{3em} % prevent overfull lines
\providecommand{\tightlist}{%
  \setlength{\itemsep}{0pt}\setlength{\parskip}{0pt}}
\setcounter{secnumdepth}{-\maxdimen} % remove section numbering

\title{DPLYR Evaluate Function with N Arrays of Inputs J times}
\author{Fan Wang}
\date{2020-04-01}

\begin{document}
\maketitle

Go back to \href{http://fanwangecon.github.io/}{fan}'s
\href{https://fanwangecon.github.io/REconTools/}{REconTools} Package,
\href{https://fanwangecon.github.io/R4Econ/}{R4Econ} Repository, or
\href{https://fanwangecon.github.io/Stat4Econ/}{Intro Stats with R}
Repository.

\hypertarget{issue-and-goal}{%
\subsection{Issue and Goal}\label{issue-and-goal}}

We want evaluate nonlinear function f(Q\_i, y\_i, ar\_x, ar\_y, c\_j,
d), where c\_j is an element of some array, d is a constant, and ar\_x
and ar\_y are arrays, both fixed. x\_i and y\_i vary over each row of
matrix. We would like to evaluate this nonlinear function concurrently
across \(N\) individuals. The eventual goal is to find the \(i\)
specific \(Q\) that solves the nonlinear equations.

In
\href{https://fanwangecon.github.io/R4Econ/function/mutatef/fs_funceval.html}{Evaluate
Nonlinear Function with N arrays of Inputs}, we evaluated the function
fixing \(c\), the difference here is that we want to evaluate \(J\)
times the overall matrix of inputs.

The results should be stored in a tibble where each of the \(j \in J\)
evaluations are stacked together with a variable \(c\) that records
which \(c_j\) value was used for this evaluation.

This is achieved here by using
\href{https://fanwangecon.github.io/R4Econ/amto/array/fs_meshr.html}{expand\_grid}.

\hypertarget{set-up}{%
\subsection{Set Up}\label{set-up}}

\begin{Shaded}
\begin{Highlighting}[]
\KeywordTok{rm}\NormalTok{(}\DataTypeTok{list =} \KeywordTok{ls}\NormalTok{(}\DataTypeTok{all.names =} \OtherTok{TRUE}\NormalTok{))}
\KeywordTok{options}\NormalTok{(}\DataTypeTok{knitr.duplicate.label =} \StringTok{'allow'}\NormalTok{)}
\end{Highlighting}
\end{Shaded}

\begin{Shaded}
\begin{Highlighting}[]
\KeywordTok{library}\NormalTok{(tidyverse)}
\KeywordTok{library}\NormalTok{(knitr)}
\KeywordTok{library}\NormalTok{(kableExtra)}
\CommentTok{# file name}
\NormalTok{st_file_name =}\StringTok{ 'fs_funceval'}
\CommentTok{# Generate R File}
\KeywordTok{try}\NormalTok{(}\KeywordTok{purl}\NormalTok{(}\KeywordTok{paste0}\NormalTok{(st_file_name, }\StringTok{".Rmd"}\NormalTok{), }\DataTypeTok{output=}\KeywordTok{paste0}\NormalTok{(st_file_name, }\StringTok{".R"}\NormalTok{), }\DataTypeTok{documentation =} \DecValTok{2}\NormalTok{))}
\CommentTok{# Generate PDF and HTML}
\CommentTok{# rmarkdown::render("C:/Users/fan/R4Econ/function/mutatef/fs_funceval.Rmd", "pdf_document")}
\CommentTok{# rmarkdown::render("C:/Users/fan/R4Econ/function/mutatef/fs_funceval.Rmd", "html_document")}
\end{Highlighting}
\end{Shaded}

\hypertarget{set-up-input-arrays}{%
\subsection{Set up Input Arrays}\label{set-up-input-arrays}}

There is a function that takes \(M=Q+P\) inputs, we want to evaluate
this function \(N\) times. Each time, there are \(M\) inputs, where all
but \(Q\) of the \(M\) inputs, meaning \(P\) of the \(M\) inputs, are
the same. In particular, \(P=Q*N\).

\[M = Q+P = Q + Q*N\]

\begin{Shaded}
\begin{Highlighting}[]
\CommentTok{# it_child_count = N, the number of children}
\NormalTok{it_N_child_cnt =}\StringTok{ }\DecValTok{5}
\CommentTok{# it_heter_param = Q, number of parameters that are heterogeneous across children}
\NormalTok{it_Q_hetpa_cnt =}\StringTok{ }\DecValTok{2}

\CommentTok{# P fixed parameters, nN is N dimensional, nP is P dimensional}
\NormalTok{ar_nN_A =}\StringTok{ }\KeywordTok{seq}\NormalTok{(}\OperatorTok{-}\DecValTok{2}\NormalTok{, }\DecValTok{2}\NormalTok{, }\DataTypeTok{length.out =}\NormalTok{ it_N_child_cnt)}
\NormalTok{ar_nN_alpha =}\StringTok{ }\KeywordTok{seq}\NormalTok{(}\FloatTok{0.1}\NormalTok{, }\FloatTok{0.9}\NormalTok{, }\DataTypeTok{length.out =}\NormalTok{ it_N_child_cnt)}
\NormalTok{ar_nP_A_alpha =}\StringTok{ }\KeywordTok{c}\NormalTok{(ar_nN_A, ar_nN_alpha)}
\NormalTok{ar_nN_N_choice =}\StringTok{ }\KeywordTok{seq}\NormalTok{(}\DecValTok{1}\NormalTok{,it_N_child_cnt)}\OperatorTok{/}\KeywordTok{sum}\NormalTok{(}\KeywordTok{seq}\NormalTok{(}\DecValTok{1}\NormalTok{,it_N_child_cnt))}

\CommentTok{# N by Q varying parameters}
\NormalTok{mt_nN_by_nQ_A_alpha =}\StringTok{ }\KeywordTok{cbind}\NormalTok{(ar_nN_A, ar_nN_alpha, ar_nN_N_choice)}
\end{Highlighting}
\end{Shaded}

Now generate a vector of \(\rho\), which represents varying planner
preference, and mesh it together with the matrix
\emph{mt\_nN\_by\_nQ\_A\_alpha}.

\begin{Shaded}
\begin{Highlighting}[]
\CommentTok{# Vector of Planner Preference}
\NormalTok{ar_rho =}\StringTok{ }\KeywordTok{c}\NormalTok{(}\OperatorTok{-}\FloatTok{0.25}\NormalTok{, }\FloatTok{-0.15}\NormalTok{, }\FloatTok{0.15}\NormalTok{, }\FloatTok{0.25}\NormalTok{)}

\CommentTok{# N by Q varying parameters but now Mesh with RHO, the J elements we want to vary over}
\NormalTok{mt_nN_by_nQ_A_alpha_mesh_rho <-}\StringTok{ }\KeywordTok{as_tibble}\NormalTok{(mt_nN_by_nQ_A_alpha) }\OperatorTok\StringTok{ }\KeywordTok{expand_grid}\NormalTok{(}\DataTypeTok{rho =}\NormalTok{ ar_rho) }\OperatorTok
\StringTok{                                  }\KeywordTok{arrange}\NormalTok{(rho, ar_nN_A, ar_nN_alpha, ar_nN_N_choice)}

\CommentTok{# Show}
\KeywordTok{kable}\NormalTok{(mt_nN_by_nQ_A_alpha_mesh_rho) }\OperatorTok
\StringTok{  }\KeywordTok{kable_styling}\NormalTok{(}\DataTypeTok{bootstrap_options =} \KeywordTok{c}\NormalTok{(}\StringTok{"striped"}\NormalTok{, }\StringTok{"hover"}\NormalTok{, }\StringTok{"condensed"}\NormalTok{, }\StringTok{"responsive"}\NormalTok{))}
\end{Highlighting}
\end{Shaded}

ar\_nN\_A

ar\_nN\_alpha

ar\_nN\_N\_choice

rho

-2

0.1

0.0666667

-0.25

-1

0.3

0.1333333

-0.25

0

0.5

0.2000000

-0.25

1

0.7

0.2666667

-0.25

2

0.9

0.3333333

-0.25

-2

0.1

0.0666667

-0.15

-1

0.3

0.1333333

-0.15

0

0.5

0.2000000

-0.15

1

0.7

0.2666667

-0.15

2

0.9

0.3333333

-0.15

-2

0.1

0.0666667

0.15

-1

0.3

0.1333333

0.15

0

0.5

0.2000000

0.15

1

0.7

0.2666667

0.15

2

0.9

0.3333333

0.15

-2

0.1

0.0666667

0.25

-1

0.3

0.1333333

0.25

0

0.5

0.2000000

0.25

1

0.7

0.2666667

0.25

2

0.9

0.3333333

0.25

And the aggregate resources available for allocations, along with
observed allocations.

\begin{Shaded}
\begin{Highlighting}[]
\CommentTok{# Total Resources available}
\NormalTok{fl_N_agg =}\StringTok{ }\DecValTok{100}
\end{Highlighting}
\end{Shaded}

\hypertarget{define-function}{%
\subsection{Define Function}\label{define-function}}

\begin{Shaded}
\begin{Highlighting}[]
\CommentTok{# Define Implicit Function}
\NormalTok{ffi_nonlin_dplyrdo <-}\StringTok{ }\ControlFlowTok{function}\NormalTok{(fl_A, fl_alpha, fl_N, fl_rho, ar_A, ar_alpha, fl_N_agg)\{}

  \CommentTok{# Test Parameters}
  \CommentTok{# ar_A = ar_nN_A}
  \CommentTok{# ar_alpha = ar_nN_alpha}
  \CommentTok{# fl_N = 100}
  \CommentTok{# fl_rho = -1}
  \CommentTok{# fl_N_q = 10}

  \CommentTok{# Apply Function}
\NormalTok{  ar_p1_s1 =}\StringTok{ }\KeywordTok{exp}\NormalTok{((fl_A }\OperatorTok{-}\StringTok{ }\NormalTok{ar_A)}\OperatorTok{*}\NormalTok{fl_rho)}
\NormalTok{  ar_p1_s2 =}\StringTok{ }\NormalTok{(fl_alpha}\OperatorTok{/}\NormalTok{ar_alpha)}
\NormalTok{  ar_p1_s3 =}\StringTok{ }\NormalTok{(}\DecValTok{1}\OperatorTok{/}\NormalTok{(ar_alpha}\OperatorTok{*}\NormalTok{fl_rho }\OperatorTok{-}\StringTok{ }\DecValTok{1}\NormalTok{))}
\NormalTok{  ar_p1 =}\StringTok{ }\NormalTok{(ar_p1_s1}\OperatorTok{*}\NormalTok{ar_p1_s2)}\OperatorTok{^}\NormalTok{ar_p1_s3}
\NormalTok{  ar_p2 =}\StringTok{ }\NormalTok{(fl_N}\OperatorTok{*}\NormalTok{fl_N_agg)}\OperatorTok{^}\NormalTok{((fl_alpha}\OperatorTok{*}\NormalTok{fl_rho}\DecValTok{-1}\NormalTok{)}\OperatorTok{/}\NormalTok{(ar_alpha}\OperatorTok{*}\NormalTok{fl_rho}\DecValTok{-1}\NormalTok{))}
\NormalTok{  ar_overall =}\StringTok{ }\NormalTok{ar_p1}\OperatorTok{*}\NormalTok{ar_p2}
\NormalTok{  fl_overall =}\StringTok{ }\NormalTok{fl_N_agg }\OperatorTok{-}\StringTok{ }\KeywordTok{sum}\NormalTok{(ar_overall)}

  \KeywordTok{return}\NormalTok{(fl_overall)}
\NormalTok{\}}
\end{Highlighting}
\end{Shaded}

\hypertarget{evaluate-nonlinear-function-using-dplyr-mutate-once}{%
\subsection{Evaluate Nonlinear Function using dplyr mutate
Once}\label{evaluate-nonlinear-function-using-dplyr-mutate-once}}

Below, repeat what we did in
\href{https://fanwangecon.github.io/R4Econ/function/mutatef/fs_funceval.html}{Evaluate
Nonlinear Function with N arrays of Inputs}.

\begin{Shaded}
\begin{Highlighting}[]
\CommentTok{# Convert Matrix to Tibble}
\NormalTok{ar_st_col_names =}\StringTok{ }\KeywordTok{c}\NormalTok{(}\StringTok{'fl_A'}\NormalTok{, }\StringTok{'fl_alpha'}\NormalTok{, }\StringTok{'fl_N'}\NormalTok{)}
\NormalTok{tb_nN_by_nQ_A_alpha <-}\StringTok{ }\KeywordTok{as_tibble}\NormalTok{(mt_nN_by_nQ_A_alpha) }\OperatorTok\StringTok{ }\KeywordTok{rename_all}\NormalTok{(}\OperatorTok{~}\KeywordTok{c}\NormalTok{(ar_st_col_names))}

\CommentTok{# fl_A, fl_alpha are from columns of tb_nN_by_nQ_A_alpha}
\NormalTok{fl_rho_here =}\StringTok{ }\NormalTok{ar_rho[}\DecValTok{1}\NormalTok{]}
\NormalTok{tb_nN_by_nQ_A_alpha =}\StringTok{ }\NormalTok{tb_nN_by_nQ_A_alpha }\OperatorTok\StringTok{ }\KeywordTok{rowwise}\NormalTok{() }\OperatorTok
\StringTok{                        }\KeywordTok{mutate}\NormalTok{(}\DataTypeTok{dplyr_eval =} \KeywordTok{ffi_nonlin_dplyrdo}\NormalTok{(fl_A, fl_alpha, fl_N, fl_rho_here,}
\NormalTok{                                                               ar_nN_A, ar_nN_alpha,}
\NormalTok{                                                               fl_N_agg))}
\CommentTok{# Show}
\KeywordTok{kable}\NormalTok{(tb_nN_by_nQ_A_alpha) }\OperatorTok
\StringTok{  }\KeywordTok{kable_styling}\NormalTok{(}\DataTypeTok{bootstrap_options =} \KeywordTok{c}\NormalTok{(}\StringTok{"striped"}\NormalTok{, }\StringTok{"hover"}\NormalTok{, }\StringTok{"condensed"}\NormalTok{, }\StringTok{"responsive"}\NormalTok{))}
\end{Highlighting}
\end{Shaded}

fl\_A

fl\_alpha

fl\_N

dplyr\_eval

-2

0.1

0.0666667

37.31686

-1

0.3

0.1333333

37.79243

0

0.5

0.2000000

17.11137

1

0.7

0.2666667

-18.21348

2

0.9

0.3333333

-73.95521

\hypertarget{evaluate-nonlinear-function-using-dplyr-multiple-times-and-stack-results}{%
\subsection{Evaluate Nonlinear Function using dplyr multiple times and
stack
results}\label{evaluate-nonlinear-function-using-dplyr-multiple-times-and-stack-results}}

Evaluate multiple times and stack results.

\begin{Shaded}
\begin{Highlighting}[]
\CommentTok{# Convert Matrix to Tibble}
\NormalTok{ar_st_col_names =}\StringTok{ }\KeywordTok{c}\NormalTok{(}\StringTok{'fl_A'}\NormalTok{, }\StringTok{'fl_alpha'}\NormalTok{, }\StringTok{'fl_N'}\NormalTok{, }\StringTok{'fl_rho'}\NormalTok{)}
\NormalTok{tb_nN_by_nQ_A_alpha_mesh_rho <-}\StringTok{ }\KeywordTok{as_tibble}\NormalTok{(mt_nN_by_nQ_A_alpha_mesh_rho) }\OperatorTok\StringTok{ }\KeywordTok{rename_all}\NormalTok{(}\OperatorTok{~}\KeywordTok{c}\NormalTok{(ar_st_col_names))}

\CommentTok{# fl_A, fl_alpha are from columns of tb_nN_by_nQ_A_alpha}
\NormalTok{tb_nN_by_nQ_A_alpha_mesh_rho =}\StringTok{ }\NormalTok{tb_nN_by_nQ_A_alpha_mesh_rho }\OperatorTok\StringTok{ }\KeywordTok{rowwise}\NormalTok{() }\OperatorTok
\StringTok{                        }\KeywordTok{mutate}\NormalTok{(}\DataTypeTok{dplyr_eval =} \KeywordTok{ffi_nonlin_dplyrdo}\NormalTok{(fl_A, fl_alpha, fl_N, fl_rho,}
\NormalTok{                                                               ar_nN_A, ar_nN_alpha,}
\NormalTok{                                                               fl_N_agg))}
\CommentTok{# Show}
\KeywordTok{kable}\NormalTok{(tb_nN_by_nQ_A_alpha_mesh_rho) }\OperatorTok
\StringTok{  }\KeywordTok{kable_styling}\NormalTok{(}\DataTypeTok{bootstrap_options =} \KeywordTok{c}\NormalTok{(}\StringTok{"striped"}\NormalTok{, }\StringTok{"hover"}\NormalTok{, }\StringTok{"condensed"}\NormalTok{, }\StringTok{"responsive"}\NormalTok{))}
\end{Highlighting}
\end{Shaded}

fl\_A

fl\_alpha

fl\_N

fl\_rho

dplyr\_eval

-2

0.1

0.0666667

-0.25

37.316863

-1

0.3

0.1333333

-0.25

37.792429

0

0.5

0.2000000

-0.25

17.111367

1

0.7

0.2666667

-0.25

-18.213484

2

0.9

0.3333333

-0.25

-73.955215

-2

0.1

0.0666667

-0.15

13.010884

-1

0.3

0.1333333

-0.15

25.426336

0

0.5

0.2000000

-0.15

14.171990

1

0.7

0.2666667

-0.15

-5.214865

2

0.9

0.3333333

-0.15

-32.498236

-2

0.1

0.0666667

0.15

-329.652680

-1

0.3

0.1333333

0.15

-108.179601

0

0.5

0.2000000

0.15

-38.299405

1

0.7

0.2666667

0.15

3.380053

2

0.9

0.3333333

0.15

31.566650

-2

0.1

0.0666667

0.25

-951.057786

-1

0.3

0.1333333

0.25

-285.694559

0

0.5

0.2000000

0.25

-97.790666

1

0.7

0.2666667

0.25

-6.281001

2

0.9

0.3333333

0.25

42.377578

\end{document}
