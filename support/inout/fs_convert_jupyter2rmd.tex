% Options for packages loaded elsewhere
\PassOptionsToPackage{unicode}{hyperref}
\PassOptionsToPackage{hyphens}{url}
\PassOptionsToPackage{dvipsnames,svgnames*,x11names*}{xcolor}
%
\documentclass[
]{article}
\usepackage{lmodern}
\usepackage{amssymb,amsmath}
\usepackage{ifxetex,ifluatex}
\ifnum 0\ifxetex 1\fi\ifluatex 1\fi=0 % if pdftex
  \usepackage[T1]{fontenc}
  \usepackage[utf8]{inputenc}
  \usepackage{textcomp} % provide euro and other symbols
\else % if luatex or xetex
  \usepackage{unicode-math}
  \defaultfontfeatures{Scale=MatchLowercase}
  \defaultfontfeatures[\rmfamily]{Ligatures=TeX,Scale=1}
\fi
% Use upquote if available, for straight quotes in verbatim environments
\IfFileExists{upquote.sty}{\usepackage{upquote}}{}
\IfFileExists{microtype.sty}{% use microtype if available
  \usepackage[]{microtype}
  \UseMicrotypeSet[protrusion]{basicmath} % disable protrusion for tt fonts
}{}
\makeatletter
\@ifundefined{KOMAClassName}{% if non-KOMA class
  \IfFileExists{parskip.sty}{%
    \usepackage{parskip}
  }{% else
    \setlength{\parindent}{0pt}
    \setlength{\parskip}{6pt plus 2pt minus 1pt}}
}{% if KOMA class
  \KOMAoptions{parskip=half}}
\makeatother
\usepackage{xcolor}
\IfFileExists{xurl.sty}{\usepackage{xurl}}{} % add URL line breaks if available
\IfFileExists{bookmark.sty}{\usepackage{bookmark}}{\usepackage{hyperref}}
\hypersetup{
  pdftitle={Convert Jupyter Files to RMD},
  colorlinks=true,
  linkcolor=Maroon,
  filecolor=Maroon,
  citecolor=Blue,
  urlcolor=blue,
  pdfcreator={LaTeX via pandoc}}
\urlstyle{same} % disable monospaced font for URLs
\usepackage[margin=1in]{geometry}
\usepackage{color}
\usepackage{fancyvrb}
\newcommand{\VerbBar}{|}
\newcommand{\VERB}{\Verb[commandchars=\\\{\}]}
\DefineVerbatimEnvironment{Highlighting}{Verbatim}{commandchars=\\\{\}}
% Add ',fontsize=\small' for more characters per line
\usepackage{framed}
\definecolor{shadecolor}{RGB}{248,248,248}
\newenvironment{Shaded}{\begin{snugshade}}{\end{snugshade}}
\newcommand{\AlertTok}[1]{\textcolor[rgb]{0.94,0.16,0.16}{#1}}
\newcommand{\AnnotationTok}[1]{\textcolor[rgb]{0.56,0.35,0.01}{\textbf{\textit{#1}}}}
\newcommand{\AttributeTok}[1]{\textcolor[rgb]{0.77,0.63,0.00}{#1}}
\newcommand{\BaseNTok}[1]{\textcolor[rgb]{0.00,0.00,0.81}{#1}}
\newcommand{\BuiltInTok}[1]{#1}
\newcommand{\CharTok}[1]{\textcolor[rgb]{0.31,0.60,0.02}{#1}}
\newcommand{\CommentTok}[1]{\textcolor[rgb]{0.56,0.35,0.01}{\textit{#1}}}
\newcommand{\CommentVarTok}[1]{\textcolor[rgb]{0.56,0.35,0.01}{\textbf{\textit{#1}}}}
\newcommand{\ConstantTok}[1]{\textcolor[rgb]{0.00,0.00,0.00}{#1}}
\newcommand{\ControlFlowTok}[1]{\textcolor[rgb]{0.13,0.29,0.53}{\textbf{#1}}}
\newcommand{\DataTypeTok}[1]{\textcolor[rgb]{0.13,0.29,0.53}{#1}}
\newcommand{\DecValTok}[1]{\textcolor[rgb]{0.00,0.00,0.81}{#1}}
\newcommand{\DocumentationTok}[1]{\textcolor[rgb]{0.56,0.35,0.01}{\textbf{\textit{#1}}}}
\newcommand{\ErrorTok}[1]{\textcolor[rgb]{0.64,0.00,0.00}{\textbf{#1}}}
\newcommand{\ExtensionTok}[1]{#1}
\newcommand{\FloatTok}[1]{\textcolor[rgb]{0.00,0.00,0.81}{#1}}
\newcommand{\FunctionTok}[1]{\textcolor[rgb]{0.00,0.00,0.00}{#1}}
\newcommand{\ImportTok}[1]{#1}
\newcommand{\InformationTok}[1]{\textcolor[rgb]{0.56,0.35,0.01}{\textbf{\textit{#1}}}}
\newcommand{\KeywordTok}[1]{\textcolor[rgb]{0.13,0.29,0.53}{\textbf{#1}}}
\newcommand{\NormalTok}[1]{#1}
\newcommand{\OperatorTok}[1]{\textcolor[rgb]{0.81,0.36,0.00}{\textbf{#1}}}
\newcommand{\OtherTok}[1]{\textcolor[rgb]{0.56,0.35,0.01}{#1}}
\newcommand{\PreprocessorTok}[1]{\textcolor[rgb]{0.56,0.35,0.01}{\textit{#1}}}
\newcommand{\RegionMarkerTok}[1]{#1}
\newcommand{\SpecialCharTok}[1]{\textcolor[rgb]{0.00,0.00,0.00}{#1}}
\newcommand{\SpecialStringTok}[1]{\textcolor[rgb]{0.31,0.60,0.02}{#1}}
\newcommand{\StringTok}[1]{\textcolor[rgb]{0.31,0.60,0.02}{#1}}
\newcommand{\VariableTok}[1]{\textcolor[rgb]{0.00,0.00,0.00}{#1}}
\newcommand{\VerbatimStringTok}[1]{\textcolor[rgb]{0.31,0.60,0.02}{#1}}
\newcommand{\WarningTok}[1]{\textcolor[rgb]{0.56,0.35,0.01}{\textbf{\textit{#1}}}}
\usepackage{graphicx,grffile}
\makeatletter
\def\maxwidth{\ifdim\Gin@nat@width>\linewidth\linewidth\else\Gin@nat@width\fi}
\def\maxheight{\ifdim\Gin@nat@height>\textheight\textheight\else\Gin@nat@height\fi}
\makeatother
% Scale images if necessary, so that they will not overflow the page
% margins by default, and it is still possible to overwrite the defaults
% using explicit options in \includegraphics[width, height, ...]{}
\setkeys{Gin}{width=\maxwidth,height=\maxheight,keepaspectratio}
% Set default figure placement to htbp
\makeatletter
\def\fps@figure{htbp}
\makeatother
\setlength{\emergencystretch}{3em} % prevent overfull lines
\providecommand{\tightlist}{%
  \setlength{\itemsep}{0pt}\setlength{\parskip}{0pt}}
\setcounter{secnumdepth}{-\maxdimen} % remove section numbering

\title{Convert Jupyter Files to RMD}
\author{}
\date{\vspace{-2.5em}}

\begin{document}
\maketitle

Back to \href{https://fanwangecon.github.io}{Fan}'s
\href{https://fanwangecon.github.io/R4Econ/}{Reusable R Code} table of
content.

\begin{Shaded}
\begin{Highlighting}[]
\KeywordTok{options}\NormalTok{(}\DataTypeTok{knitr.duplicate.label =} \StringTok{'allow'}\NormalTok{)}
\end{Highlighting}
\end{Shaded}

\begin{Shaded}
\begin{Highlighting}[]
\KeywordTok{library}\NormalTok{(tidyverse)}
\KeywordTok{library}\NormalTok{(tidyr)}
\KeywordTok{library}\NormalTok{(rmarkdown)}
\KeywordTok{library}\NormalTok{(knitr)}
\KeywordTok{library}\NormalTok{(kableExtra)}
\CommentTok{# file name}
\NormalTok{st_file_name =}\StringTok{ 'fs_convert_jupyter2rmd'}
\CommentTok{# Generate R File}
\KeywordTok{try}\NormalTok{(}\KeywordTok{purl}\NormalTok{(}\KeywordTok{paste0}\NormalTok{(st_file_name, }\StringTok{".Rmd"}\NormalTok{), }\DataTypeTok{output=}\KeywordTok{paste0}\NormalTok{(st_file_name, }\StringTok{".R"}\NormalTok{), }\DataTypeTok{documentation =} \DecValTok{2}\NormalTok{))}
\CommentTok{# Generate PDF and HTML}
\CommentTok{# rmarkdown::render("C:/Users/fan/R4Econ/amto/array/fs_meshr.Rmd", "pdf_document")}
\CommentTok{# rmarkdown::render("C:/Users/fan/R4Econ/amto/array/fs_meshr.Rmd", "html_document")}
\end{Highlighting}
\end{Shaded}

\hypertarget{jupyter-files-and-rmd}{%
\subsection{Jupyter Files and RMD}\label{jupyter-files-and-rmd}}

Rmarkdown in Rstudio is easier for debugging, and allows for easier
interaction with current workspace.

\hypertarget{single-jupyter-to-rmd-conversion}{%
\subsubsection{Single Jupyter to RMD
conversion}\label{single-jupyter-to-rmd-conversion}}

Rmarkdown has a conversion program:
\href{https://rmarkdown.rstudio.com/docs/reference/convert_ipynb.html}{convert\_ipynb}.

\begin{Shaded}
\begin{Highlighting}[]
\CommentTok{# Generate Paths}
\NormalTok{spt_file_root =}\StringTok{ 'C:/Users/fan/Stat4Econ/descriptive/'}
\NormalTok{spt_file_name =}\StringTok{ 'DataBasketball'}
\NormalTok{spt_file_full_ipynb =}\StringTok{ }\KeywordTok{paste0}\NormalTok{(spt_file_root, spt_file_name, }\StringTok{'.ipynb'}\NormalTok{)}
\NormalTok{spt_file_full_rmd =}\StringTok{ }\KeywordTok{paste0}\NormalTok{(spt_file_root, spt_file_name, }\StringTok{'.rmd'}\NormalTok{)}

\CommentTok{# Convert from IPYNB to RMD}
\NormalTok{file_nb_rmd =}\StringTok{ }\NormalTok{rmarkdown}\OperatorTok{:::}\KeywordTok{convert_ipynb}\NormalTok{(spt_file_full_ipynb)}
\NormalTok{st_nb_rmd =}\StringTok{ }\NormalTok{xfun}\OperatorTok{::}\KeywordTok{file_string}\NormalTok{(file_nb_rmd)}

\CommentTok{# Save RMD}
\NormalTok{fileConn <-}\StringTok{ }\KeywordTok{file}\NormalTok{(spt_file_full_rmd)}
\KeywordTok{writeLines}\NormalTok{(st_nb_rmd, fileConn)}
\KeywordTok{close}\NormalTok{(fileConn)}

\CommentTok{# Convert to PDF and HTML}
\NormalTok{rmarkdown}\OperatorTok{::}\KeywordTok{render}\NormalTok{(spt_file_full_rmd, }\StringTok{"pdf_document"}\NormalTok{)}
\end{Highlighting}
\end{Shaded}

\begin{verbatim}
## 
## 
## processing file: DataBasketball.rmd
\end{verbatim}

\begin{verbatim}
##   |                                                                                              |                                                                                      |   0%  |                                                                                              |............                                                                          |  14%
##   ordinary text without R code
## 
##   |                                                                                              |.........................                                                             |  29%
## label: unnamed-chunk-1-2
\end{verbatim}

\begin{verbatim}
## Warning in in_dir(input_dir(), evaluate(code, envir = env, new_device = FALSE, : You changed the
## working directory to C:/Users/fan/Stat4Econ/data (probably via setwd()). It will be restored to
## C:/Users/fan/Stat4Econ/descriptive. See the Note section in ?knitr::knit
\end{verbatim}

\begin{verbatim}
##   |                                                                                              |.....................................                                                 |  43%
##   ordinary text without R code
## 
##   |                                                                                              |.................................................                                     |  57%
## label: unnamed-chunk-3
##   |                                                                                              |.............................................................                         |  71%
##   ordinary text without R code
## 
##   |                                                                                              |..........................................................................            |  86%
## label: unnamed-chunk-4
##   |                                                                                              |......................................................................................| 100%
##   ordinary text without R code
\end{verbatim}

\begin{verbatim}
## output file: DataBasketball.knit.md
\end{verbatim}

\begin{verbatim}
## "C:/Program Files/RStudio/bin/pandoc/pandoc" +RTS -K512m -RTS DataBasketball.utf8.md --to latex --from markdown+autolink_bare_uris+tex_math_single_backslash --output DataBasketball.tex --self-contained --highlight-style tango --pdf-engine pdflatex --variable graphics --lua-filter "C:/Users/fan/Documents/R/win-library/3.6/rmarkdown/rmd/lua/pagebreak.lua" --lua-filter "C:/Users/fan/Documents/R/win-library/3.6/rmarkdown/rmd/lua/latex-div.lua" --variable "geometry:margin=1in"
\end{verbatim}

\begin{verbatim}
## 
## Output created: DataBasketball.pdf
\end{verbatim}

\begin{Shaded}
\begin{Highlighting}[]
\NormalTok{rmarkdown}\OperatorTok{::}\KeywordTok{render}\NormalTok{(spt_file_full_rmd, }\StringTok{"html_document"}\NormalTok{)}
\end{Highlighting}
\end{Shaded}

\begin{verbatim}
## 
## 
## processing file: DataBasketball.rmd
\end{verbatim}

\begin{verbatim}
##   |                                                                                              |                                                                                      |   0%  |                                                                                              |............                                                                          |  14%
##   ordinary text without R code
## 
##   |                                                                                              |.........................                                                             |  29%
## label: unnamed-chunk-1-2
\end{verbatim}

\begin{verbatim}
## Warning in in_dir(input_dir(), evaluate(code, envir = env, new_device = FALSE, : You changed the
## working directory to C:/Users/fan/Stat4Econ/data (probably via setwd()). It will be restored to
## C:/Users/fan/Stat4Econ/descriptive. See the Note section in ?knitr::knit
\end{verbatim}

\begin{verbatim}
##   |                                                                                              |.....................................                                                 |  43%
##   ordinary text without R code
## 
##   |                                                                                              |.................................................                                     |  57%
## label: unnamed-chunk-3
##   |                                                                                              |.............................................................                         |  71%
##   ordinary text without R code
## 
##   |                                                                                              |..........................................................................            |  86%
## label: unnamed-chunk-4
##   |                                                                                              |......................................................................................| 100%
##   ordinary text without R code
\end{verbatim}

\begin{verbatim}
## output file: DataBasketball.knit.md
\end{verbatim}

\begin{verbatim}
## "C:/Program Files/RStudio/bin/pandoc/pandoc" +RTS -K512m -RTS DataBasketball.utf8.md --to html4 --from markdown+autolink_bare_uris+tex_math_single_backslash+smart --output DataBasketball.html --email-obfuscation none --self-contained --standalone --section-divs --template "C:\Users\fan\Documents\R\win-library\3.6\rmarkdown\rmd\h\default.html" --no-highlight --variable highlightjs=1 --variable "theme:bootstrap" --include-in-header "C:\Users\fan\AppData\Local\Temp\RtmpQ5qumo\rmarkdown-str8858638e2dbd.html" --mathjax --variable "mathjax-url:https://mathjax.rstudio.com/latest/MathJax.js?config=TeX-AMS-MML_HTMLorMML" --lua-filter "C:/Users/fan/Documents/R/win-library/3.6/rmarkdown/rmd/lua/pagebreak.lua" --lua-filter "C:/Users/fan/Documents/R/win-library/3.6/rmarkdown/rmd/lua/latex-div.lua"
\end{verbatim}

\begin{verbatim}
## 
## Output created: DataBasketball.html
\end{verbatim}

\hypertarget{multiple-jupyter-to-rmd-conversion}{%
\subsubsection{Multiple Jupyter to RMD
conversion}\label{multiple-jupyter-to-rmd-conversion}}

Search in folder for ipynb files, and then convert collectively to Rmd.

\begin{Shaded}
\begin{Highlighting}[]
\CommentTok{# Generate Paths}
\NormalTok{spt_file_root =}\StringTok{ 'C:/Users/fan/Stat4Econ/descriptive/'}
\NormalTok{spt_file_name =}\StringTok{ 'DataBasketball'}
\NormalTok{spt_file_full_ipynb =}\StringTok{ }\KeywordTok{paste0}\NormalTok{(spt_file_root, spt_file_name, }\StringTok{'.ipynb'}\NormalTok{)}
\NormalTok{spt_file_full_rmd =}\StringTok{ }\KeywordTok{paste0}\NormalTok{(spt_file_root, spt_file_name, }\StringTok{'.rmd'}\NormalTok{)}

\CommentTok{# Convert from IPYNB to RMD}
\NormalTok{file_nb_rmd =}\StringTok{ }\NormalTok{rmarkdown}\OperatorTok{:::}\KeywordTok{convert_ipynb}\NormalTok{(spt_file_full_ipynb)}
\NormalTok{st_nb_rmd =}\StringTok{ }\NormalTok{xfun}\OperatorTok{::}\KeywordTok{file_string}\NormalTok{(file_nb_rmd)}

\CommentTok{# Save RMD}
\NormalTok{fileConn <-}\StringTok{ }\KeywordTok{file}\NormalTok{(spt_file_full_rmd)}
\KeywordTok{writeLines}\NormalTok{(st_nb_rmd, fileConn)}
\KeywordTok{close}\NormalTok{(fileConn)}

\CommentTok{# Convert to PDF and HTML}
\NormalTok{rmarkdown}\OperatorTok{::}\KeywordTok{render}\NormalTok{(spt_file_full_rmd, }\StringTok{"pdf_document"}\NormalTok{)}
\end{Highlighting}
\end{Shaded}

\begin{verbatim}
## 
## 
## processing file: DataBasketball.rmd
\end{verbatim}

\begin{verbatim}
##   |                                                                                              |                                                                                      |   0%  |                                                                                              |............                                                                          |  14%
##   ordinary text without R code
## 
##   |                                                                                              |.........................                                                             |  29%
## label: unnamed-chunk-1-2
\end{verbatim}

\begin{verbatim}
## Warning in in_dir(input_dir(), evaluate(code, envir = env, new_device = FALSE, : You changed the
## working directory to C:/Users/fan/Stat4Econ/data (probably via setwd()). It will be restored to
## C:/Users/fan/Stat4Econ/descriptive. See the Note section in ?knitr::knit
\end{verbatim}

\begin{verbatim}
##   |                                                                                              |.....................................                                                 |  43%
##   ordinary text without R code
## 
##   |                                                                                              |.................................................                                     |  57%
## label: unnamed-chunk-3
##   |                                                                                              |.............................................................                         |  71%
##   ordinary text without R code
## 
##   |                                                                                              |..........................................................................            |  86%
## label: unnamed-chunk-4
##   |                                                                                              |......................................................................................| 100%
##   ordinary text without R code
\end{verbatim}

\begin{verbatim}
## output file: DataBasketball.knit.md
\end{verbatim}

\begin{verbatim}
## "C:/Program Files/RStudio/bin/pandoc/pandoc" +RTS -K512m -RTS DataBasketball.utf8.md --to latex --from markdown+autolink_bare_uris+tex_math_single_backslash --output DataBasketball.tex --self-contained --highlight-style tango --pdf-engine pdflatex --variable graphics --lua-filter "C:/Users/fan/Documents/R/win-library/3.6/rmarkdown/rmd/lua/pagebreak.lua" --lua-filter "C:/Users/fan/Documents/R/win-library/3.6/rmarkdown/rmd/lua/latex-div.lua" --variable "geometry:margin=1in"
\end{verbatim}

\begin{verbatim}
## 
## Output created: DataBasketball.pdf
\end{verbatim}

\begin{Shaded}
\begin{Highlighting}[]
\NormalTok{rmarkdown}\OperatorTok{::}\KeywordTok{render}\NormalTok{(spt_file_full_rmd, }\StringTok{"html_document"}\NormalTok{)}
\end{Highlighting}
\end{Shaded}

\begin{verbatim}
## 
## 
## processing file: DataBasketball.rmd
\end{verbatim}

\begin{verbatim}
##   |                                                                                              |                                                                                      |   0%  |                                                                                              |............                                                                          |  14%
##   ordinary text without R code
## 
##   |                                                                                              |.........................                                                             |  29%
## label: unnamed-chunk-1-2
\end{verbatim}

\begin{verbatim}
## Warning in in_dir(input_dir(), evaluate(code, envir = env, new_device = FALSE, : You changed the
## working directory to C:/Users/fan/Stat4Econ/data (probably via setwd()). It will be restored to
## C:/Users/fan/Stat4Econ/descriptive. See the Note section in ?knitr::knit
\end{verbatim}

\begin{verbatim}
##   |                                                                                              |.....................................                                                 |  43%
##   ordinary text without R code
## 
##   |                                                                                              |.................................................                                     |  57%
## label: unnamed-chunk-3
##   |                                                                                              |.............................................................                         |  71%
##   ordinary text without R code
## 
##   |                                                                                              |..........................................................................            |  86%
## label: unnamed-chunk-4
##   |                                                                                              |......................................................................................| 100%
##   ordinary text without R code
\end{verbatim}

\begin{verbatim}
## output file: DataBasketball.knit.md
\end{verbatim}

\begin{verbatim}
## "C:/Program Files/RStudio/bin/pandoc/pandoc" +RTS -K512m -RTS DataBasketball.utf8.md --to html4 --from markdown+autolink_bare_uris+tex_math_single_backslash+smart --output DataBasketball.html --email-obfuscation none --self-contained --standalone --section-divs --template "C:\Users\fan\Documents\R\win-library\3.6\rmarkdown\rmd\h\default.html" --no-highlight --variable highlightjs=1 --variable "theme:bootstrap" --include-in-header "C:\Users\fan\AppData\Local\Temp\RtmpQ5qumo\rmarkdown-str8858282e6ba8.html" --mathjax --variable "mathjax-url:https://mathjax.rstudio.com/latest/MathJax.js?config=TeX-AMS-MML_HTMLorMML" --lua-filter "C:/Users/fan/Documents/R/win-library/3.6/rmarkdown/rmd/lua/pagebreak.lua" --lua-filter "C:/Users/fan/Documents/R/win-library/3.6/rmarkdown/rmd/lua/latex-div.lua"
\end{verbatim}

\begin{verbatim}
## 
## Output created: DataBasketball.html
\end{verbatim}

\end{document}
