% Options for packages loaded elsewhere
\PassOptionsToPackage{unicode}{hyperref}
\PassOptionsToPackage{hyphens}{url}
\PassOptionsToPackage{dvipsnames,svgnames*,x11names*}{xcolor}
%
\documentclass[
]{article}
\usepackage{lmodern}
\usepackage{amssymb,amsmath}
\usepackage{ifxetex,ifluatex}
\ifnum 0\ifxetex 1\fi\ifluatex 1\fi=0 % if pdftex
  \usepackage[T1]{fontenc}
  \usepackage[utf8]{inputenc}
  \usepackage{textcomp} % provide euro and other symbols
\else % if luatex or xetex
  \usepackage{unicode-math}
  \defaultfontfeatures{Scale=MatchLowercase}
  \defaultfontfeatures[\rmfamily]{Ligatures=TeX,Scale=1}
\fi
% Use upquote if available, for straight quotes in verbatim environments
\IfFileExists{upquote.sty}{\usepackage{upquote}}{}
\IfFileExists{microtype.sty}{% use microtype if available
  \usepackage[]{microtype}
  \UseMicrotypeSet[protrusion]{basicmath} % disable protrusion for tt fonts
}{}
\makeatletter
\@ifundefined{KOMAClassName}{% if non-KOMA class
  \IfFileExists{parskip.sty}{%
    \usepackage{parskip}
  }{% else
    \setlength{\parindent}{0pt}
    \setlength{\parskip}{6pt plus 2pt minus 1pt}}
}{% if KOMA class
  \KOMAoptions{parskip=half}}
\makeatother
\usepackage{xcolor}
\IfFileExists{xurl.sty}{\usepackage{xurl}}{} % add URL line breaks if available
\IfFileExists{bookmark.sty}{\usepackage{bookmark}}{\usepackage{hyperref}}
\hypersetup{
  pdfauthor={Fan Wang},
  colorlinks=true,
  linkcolor=Maroon,
  filecolor=Maroon,
  citecolor=Blue,
  urlcolor=blue,
  pdfcreator={LaTeX via pandoc}}
\urlstyle{same} % disable monospaced font for URLs
\usepackage[margin=1in]{geometry}
\usepackage{color}
\usepackage{fancyvrb}
\newcommand{\VerbBar}{|}
\newcommand{\VERB}{\Verb[commandchars=\\\{\}]}
\DefineVerbatimEnvironment{Highlighting}{Verbatim}{commandchars=\\\{\}}
% Add ',fontsize=\small' for more characters per line
\usepackage{framed}
\definecolor{shadecolor}{RGB}{248,248,248}
\newenvironment{Shaded}{\begin{snugshade}}{\end{snugshade}}
\newcommand{\AlertTok}[1]{\textcolor[rgb]{0.94,0.16,0.16}{#1}}
\newcommand{\AnnotationTok}[1]{\textcolor[rgb]{0.56,0.35,0.01}{\textbf{\textit{#1}}}}
\newcommand{\AttributeTok}[1]{\textcolor[rgb]{0.77,0.63,0.00}{#1}}
\newcommand{\BaseNTok}[1]{\textcolor[rgb]{0.00,0.00,0.81}{#1}}
\newcommand{\BuiltInTok}[1]{#1}
\newcommand{\CharTok}[1]{\textcolor[rgb]{0.31,0.60,0.02}{#1}}
\newcommand{\CommentTok}[1]{\textcolor[rgb]{0.56,0.35,0.01}{\textit{#1}}}
\newcommand{\CommentVarTok}[1]{\textcolor[rgb]{0.56,0.35,0.01}{\textbf{\textit{#1}}}}
\newcommand{\ConstantTok}[1]{\textcolor[rgb]{0.00,0.00,0.00}{#1}}
\newcommand{\ControlFlowTok}[1]{\textcolor[rgb]{0.13,0.29,0.53}{\textbf{#1}}}
\newcommand{\DataTypeTok}[1]{\textcolor[rgb]{0.13,0.29,0.53}{#1}}
\newcommand{\DecValTok}[1]{\textcolor[rgb]{0.00,0.00,0.81}{#1}}
\newcommand{\DocumentationTok}[1]{\textcolor[rgb]{0.56,0.35,0.01}{\textbf{\textit{#1}}}}
\newcommand{\ErrorTok}[1]{\textcolor[rgb]{0.64,0.00,0.00}{\textbf{#1}}}
\newcommand{\ExtensionTok}[1]{#1}
\newcommand{\FloatTok}[1]{\textcolor[rgb]{0.00,0.00,0.81}{#1}}
\newcommand{\FunctionTok}[1]{\textcolor[rgb]{0.00,0.00,0.00}{#1}}
\newcommand{\ImportTok}[1]{#1}
\newcommand{\InformationTok}[1]{\textcolor[rgb]{0.56,0.35,0.01}{\textbf{\textit{#1}}}}
\newcommand{\KeywordTok}[1]{\textcolor[rgb]{0.13,0.29,0.53}{\textbf{#1}}}
\newcommand{\NormalTok}[1]{#1}
\newcommand{\OperatorTok}[1]{\textcolor[rgb]{0.81,0.36,0.00}{\textbf{#1}}}
\newcommand{\OtherTok}[1]{\textcolor[rgb]{0.56,0.35,0.01}{#1}}
\newcommand{\PreprocessorTok}[1]{\textcolor[rgb]{0.56,0.35,0.01}{\textit{#1}}}
\newcommand{\RegionMarkerTok}[1]{#1}
\newcommand{\SpecialCharTok}[1]{\textcolor[rgb]{0.00,0.00,0.00}{#1}}
\newcommand{\SpecialStringTok}[1]{\textcolor[rgb]{0.31,0.60,0.02}{#1}}
\newcommand{\StringTok}[1]{\textcolor[rgb]{0.31,0.60,0.02}{#1}}
\newcommand{\VariableTok}[1]{\textcolor[rgb]{0.00,0.00,0.00}{#1}}
\newcommand{\VerbatimStringTok}[1]{\textcolor[rgb]{0.31,0.60,0.02}{#1}}
\newcommand{\WarningTok}[1]{\textcolor[rgb]{0.56,0.35,0.01}{\textbf{\textit{#1}}}}
\usepackage{graphicx,grffile}
\makeatletter
\def\maxwidth{\ifdim\Gin@nat@width>\linewidth\linewidth\else\Gin@nat@width\fi}
\def\maxheight{\ifdim\Gin@nat@height>\textheight\textheight\else\Gin@nat@height\fi}
\makeatother
% Scale images if necessary, so that they will not overflow the page
% margins by default, and it is still possible to overwrite the defaults
% using explicit options in \includegraphics[width, height, ...]{}
\setkeys{Gin}{width=\maxwidth,height=\maxheight,keepaspectratio}
% Set default figure placement to htbp
\makeatletter
\def\fps@figure{htbp}
\makeatother
\setlength{\emergencystretch}{3em} % prevent overfull lines
\providecommand{\tightlist}{%
  \setlength{\itemsep}{0pt}\setlength{\parskip}{0pt}}
\setcounter{secnumdepth}{5}

\title{Generate R Package Example\\
roxygen2 and pkgdown}
\author{Fan Wang}
\date{2020-04-01}

\begin{document}
\maketitle

Go back to \href{http://fanwangecon.github.io/}{fan}'s
\href{https://fanwangecon.github.io/REconTools/}{REconTools} Package,
\href{https://fanwangecon.github.io/R4Econ/}{R4Econ} Repository, or
\href{https://fanwangecon.github.io/Stat4Econ/}{Intro Stats with R}
Repository.

\hypertarget{objective}{%
\section{Objective}\label{objective}}

Document and generate sharable R package.

\begin{enumerate}
\def\labelenumi{\arabic{enumi}.}
\tightlist
\item
  Use \href{https://github.com/hadley/roxygen2}{roxygen2} from
  \href{https://github.com/hadley}{Hadley} to document package.
\item
  Use \href{https://github.com/r-lib/pkgdown}{pkgdown} to package file
  and publish to github pages.
\end{enumerate}

\hypertarget{file-structure-and-naming-convention}{%
\section{File Structure and Naming
Convention}\label{file-structure-and-naming-convention}}

\textbf{Folders}:

\begin{enumerate}
\def\labelenumi{\arabic{enumi}.}
\tightlist
\item
  R function functions in the \emph{/R} folder
\item
  RData files in the \emph{/data} folder
\end{enumerate}

\textbf{Naming Conventions}:

\begin{enumerate}
\def\labelenumi{\arabic{enumi}.}
\tightlist
\item
  functions files and functions all should be \emph{snake\_case\_names}
\item
  function name prefix:

  \begin{itemize}
  \tightlist
  \item
    \emph{fs\_}: for non-project specific script files
  \item
    \emph{ffs\_}: for project specific functions script files
  \item
    \emph{fv\_}: for non-project specific vignettes files, generally RMD
  \item
    \emph{ffv\_}: for project specific functions vignettes files,
    generally RMD
  \item
    \emph{ff\_}: for non-project specific functions files
  \item
    \emph{ffd\_}: for project specific data description files
  \item
    \emph{ffp\_}: for project specific functions files
  \item
    \emph{ffy\_}: for project specific utility files
  \item
    Each function file could be prepared to have multiple functions
    inside, each file have the root which is the function name.
  \end{itemize}
\end{enumerate}

Additionally, follow these general structures for \emph{functions} in
\emph{/R} folder:

\begin{enumerate}
\def\labelenumi{\arabic{enumi}.}
\tightlist
\item
  3 letter/digit project name
\item
  3 letter/digit current file name, in R4E, three letter func group
  name: \emph{ffp\_opt\_lin.R}
\item
  5 letter/digit function name within file: \emph{function
  ffp\_opt\_lin\_solum}
\end{enumerate}

With this structure, we end up with potentially fairly long names, but
hopefully also not too long and clear.

Additionally, follow these general structures for \emph{vignettes} in
\emph{/vignettes} folder:

\begin{enumerate}
\def\labelenumi{\arabic{enumi}.}
\tightlist
\item
  Follow conventions for function name vignettes should be associated
  with functions
\item
  There should be a core vignette that has the same name as the R
  function it is trying to provide line by line details or math
  descriptions of. If the function is called:
  \emph{ffp\_opt\_lin\_solum.r}, the vignette shold be
  **ffv\_opt\_lin\_solum.Rmd*.
\item
  Append onto existing function name 5 letter/digit to describe what
  this vignettes is suppose to achieve:
  \emph{ffv\_opt\_lin\_solum\_vign1.Rmd}.

  \begin{itemize}
  \tightlist
  \item
    note the file starts with \emph{ffv}
  \end{itemize}
\end{enumerate}

\hypertarget{create-r-project-directory}{%
\section{Create R Project Directory}\label{create-r-project-directory}}

\hypertarget{folder-does-not-exist-yet}{%
\subsection{Folder does not exist yet}\label{folder-does-not-exist-yet}}

If the project/folder does not yet exist:

\begin{enumerate}
\def\labelenumi{\arabic{enumi}.}
\tightlist
\item
  \emph{devtools:create} the folder of interest
\item
  Move your files over to \emph{/R} and \emph{/data} folders.

  \begin{itemize}
  \tightlist
  \item
    Write files following conventions above and with r descriptions
  \end{itemize}
\item
  \emph{pkgdown::build\_site()}
\end{enumerate}

\begin{Shaded}
\begin{Highlighting}[]
\NormalTok{devtools}\OperatorTok{::}\KeywordTok{create}\NormalTok{(}\StringTok{"C:/Users/fan/R4Econ"}\NormalTok{)}
\end{Highlighting}
\end{Shaded}

\hypertarget{folder-already-exists}{%
\subsection{Folder already exists}\label{folder-already-exists}}

If there is already a folder with a bunch of files including R and not R
files, and the folder needs to be converted to become a R package and
was previously not a R package.

The idea is to use a folder somewhere to generate a generic template
folder. This folder will have files and structure we need for our actual
folder. Each time, we will just copy that folder's contents into the
folder that we want to turn into a R folder. And do a global search to
replace the template folder's folder name with the actual project name.
To avoid confusion, generate this folder outside of an existing R
package.

If there is already a R folder in your existing project, delete that or
rename that and move files back into the \emph{/R} folder after
completion. Make sure there are no duplicate folder or file names in the
old and the new project.

Search replace the word \emph{rprjtemplate} inside your old project
folder with the new files, replace that with your project name. Should
appear in three different spots.

\hypertarget{r-project-template}{%
\subsubsection{R project template}\label{r-project-template}}

The
{[}rprjtemplate{]}{[}\url{https://github.com/FanWangEcon/Tex4Econ/tree/master/nontex/rprjtemplate}{]}
serves this templating role.

\begin{Shaded}
\begin{Highlighting}[]
\CommentTok{# Create project}
\NormalTok{devtools}\OperatorTok{::}\KeywordTok{create}\NormalTok{(}\StringTok{"C:/Users/fan/Tex4Econ/nontex/rprjtemplate"}\NormalTok{)}
\end{Highlighting}
\end{Shaded}

Running the \emph{devtools::create()} command will create the core
needed folder structure with:

\begin{enumerate}
\def\labelenumi{\arabic{enumi}.}
\tightlist
\item
  NAMESPACE
\item
  DESCRIPTION
\item
  \emph{.gitignore}
\item
  \emph{.Rbuildignore}
\item
  \emph{.Rproj}
\item
  Empty R folder
\end{enumerate}

Now customize this folder for future use with

\begin{enumerate}
\def\labelenumi{\arabic{enumi}.}
\tightlist
\item
  custom \emph{.gitignore}
\item
  MIT LICENSE
\item
  etc.
\end{enumerate}

For adding vignette later:

Add these to the DESCRIPTION file: - Suggests: knitr, rmarkdown -
VignetteBuilder: knitr

\hypertarget{develop-and-package-r-project}{%
\section{Develop and Package R
Project}\label{develop-and-package-r-project}}

Developing a package means write various vignettes and functions. My
process is to write various vignettes first, to basically test out core
functionalities. Then convert parts of those vignettes to functions.

As functions are developed, some vignettes can be invoked differently,
taking advantage of newly written functions. Make sure keep the original
vignette that is not dependent on any other functions for ease of going
to the initial point and figuring out what is happening. Names could be
written to indicate vignette vintage.

\hypertarget{development-process}{%
\subsection{Development Process}\label{development-process}}

\begin{enumerate}
\def\labelenumi{\arabic{enumi}.}
\tightlist
\item
  Write original vignette with dependencies, minimize dependencies, for
  example, do not import tidyverse, since package will not allow for all
  tidyverse components to be added, too much space needed.

  \begin{itemize}
  \tightlist
  \item
    this is not really a vignette, it is a function testing sandbox that
    does not call any package functions but provides line by line
    descriptions for the core functions with print outputs.
  \end{itemize}
\item
  Test vignette inside Rstudio/VSCode etc. (These vignette do not depend
  on the function to be written below, they do not exist yet)
\item
  When raw vignettes work, start converting vignette components to
  functions, include minimal dependencies,

  \begin{itemize}
  \tightlist
  \item
    in function: \emph{@import}
  \item
    add to DESCRIPTION: \textbf{usethis::use\_package(``tidyr'')} if
    tidyr is needed as \emph{IMPORT}
  \item
    \emph{DEPENDS} vs \emph{IMPORTS}
  \item
    \emph{important}: slowly check what does the function depend on, for
    example \emph{R4Econ} has its own dependencies, when installing as
    packages, make those dependencies explicit in the new function if it
    calls \emph{R4Econ}. Otherwise, \emph{test()} might throw error.
    This is resolved if \emph{R4Econ} add several things under depends.
  \end{itemize}
\item
  Run the documentation tool:

  \begin{itemize}
  \tightlist
  \item
    \textbf{setwd(`C:/Users/fan/PrjOptiAlloc')}
  \item
    \textbf{devtools::document()}
  \item
    this updates \emph{NAMESPACE}
  \item
    generates new \emph{.Rd} files
  \end{itemize}
\item
  Test @examples from Roxygen documentations

  \begin{itemize}
  \tightlist
  \item
    \textbf{devtools::run\_examples()}
  \item
    if also want to test vignette, also do:
    \textbf{devtools::build\_vignettes()}
  \item
    this tests if \emph{@examples} are working.
  \end{itemize}
\item
  Continue 3 to 4 - {[} {]} o convert various components of vignette.
\item
  Build package and load package using:

  \begin{itemize}
  \tightlist
  \item
    \textbf{devtools::check()}
  \item
    this will run \emph{run\_examples} as well, but at the very end,
    will test everything for package.\\
  \end{itemize}
\item
  Now install package

  \begin{itemize}
  \tightlist
  \item
    \textbf{devtools::install()}
  \item
    \emph{library(PrjOptiAlloc)} should work now
  \item
    \emph{ls(``package:PrjOptiAlloc'')} to list all objects in package\\
  \item
    \textbf{devtools::reload()}
  \end{itemize}
\item
  Publish webpage, review local, using \emph{pkgdown}:

  \begin{itemize}
  \tightlist
  \item
    \textbf{pkgdown::build\_site()}
  \end{itemize}
\end{enumerate}

\begin{Shaded}
\begin{Highlighting}[]
\CommentTok{# Step 3}
\KeywordTok{rm}\NormalTok{(}\DataTypeTok{list =} \KeywordTok{ls}\NormalTok{(}\DataTypeTok{all.names =} \OtherTok{TRUE}\NormalTok{))}
\KeywordTok{setwd}\NormalTok{(}\StringTok{'C:/Users/fan/PrjOptiAlloc'}\NormalTok{)}
\CommentTok{# setwd('C:/Users/fan/REconTools')}
\NormalTok{devtools}\OperatorTok{::}\KeywordTok{document}\NormalTok{()}
\CommentTok{# Step 4}
\NormalTok{devtools}\OperatorTok{::}\KeywordTok{run_examples}\NormalTok{()}
\CommentTok{# devtools::build_vignettes()}
\CommentTok{# Step 6}
\NormalTok{devtools}\OperatorTok{::}\KeywordTok{check}\NormalTok{()}
\CommentTok{# Step 7}
\NormalTok{devtools}\OperatorTok{::}\KeywordTok{install}\NormalTok{()}
\CommentTok{#library(PrjOptiAlloc)}
\CommentTok{#ls("package:PrjOptiAlloc")}
\NormalTok{devtools}\OperatorTok{::}\KeywordTok{reload}\NormalTok{()}
\CommentTok{# Step 8}
\NormalTok{pkgdown}\OperatorTok{::}\KeywordTok{build_site}\NormalTok{()}
\end{Highlighting}
\end{Shaded}

During development, after new R functions have been written, testing
latested functions in vignette.

\begin{Shaded}
\begin{Highlighting}[]
\NormalTok{devtools}\OperatorTok{::}\KeywordTok{load_all}\NormalTok{()}
\KeywordTok{library}\NormalTok{(PrjOptiAlloc)}
\end{Highlighting}
\end{Shaded}

bookdown

\begin{Shaded}
\begin{Highlighting}[]
\KeywordTok{setwd}\NormalTok{(}\StringTok{'C:/Users/fan/R4Econ'}\NormalTok{)}
\NormalTok{bookdown}\OperatorTok{::}\KeywordTok{render_book}\NormalTok{(}\StringTok{'index.Rmd'}\NormalTok{, }\StringTok{'bookdown::gitbook'}\NormalTok{)}
\NormalTok{bookdown}\OperatorTok{::}\KeywordTok{render_book}\NormalTok{(}\StringTok{'index.Rmd'}\NormalTok{, }\StringTok{'bookdown::pdf_book'}\NormalTok{)}
\end{Highlighting}
\end{Shaded}

\hypertarget{attach-vs-import}{%
\subsection{Attach vs Import}\label{attach-vs-import}}

Packages that are loaded in under *usethis::use\_package** are imports,
not attached.

According to \href{http://r-pkgs.had.co.nz/namespace.html}{Wickham}: -
\emph{apkg::bfnc()}: when packages are loaded, its components can be
accessed with the \emph{::}. That is, if I want to use the
\emph{document()} from from \emph{devtools()}, if I first attach, I can
directly call document(). Without attaching, just loading, I have to do
\emph{devtools::document()}. - \emph{attach(), library()} : when
attaching, must first load. After attached, can directly call function
names, for example, \emph{filter}, it is on the search path.

See
\href{http://blog.obeautifulcode.com/R/How-R-Searches-And-Finds-Stuff/}{How
R Searches and Finds Stuff}:

\begin{quote}
The better solution would have been to stuff reshape's cast() function
into imports:ggplot2 using the Imports feature. In that case, we would
have travelled from 2 to 3 and stopped. Now you can see why the choice
between Imports and Depends is not arbitrary. With so many packages on
CRAN and so many of us working in related disciplines its no surprise
that same-named functions appear in multiple packages. Depends is less
safe. Depends makes a package vulnerable to whatever other packages are
loaded by the user.
\end{quote}

This is
\href{https://stackoverflow.com/questions/8637993/better-explanation-of-when-to-use-imports-depends}{question}
as well.

\hypertarget{dealing-with-datasets}{%
\subsection{Dealing with Datasets}\label{dealing-with-datasets}}

Datasets to be used with the project should be in the \emph{/data}
folder. The name of the data file should appear in several spots and be
consistent, suppose data is called \emph{abc}

\begin{enumerate}
\def\labelenumi{\arabic{enumi}.}
\tightlist
\item
  Rdata file name: \emph{/data/abc.Rdata}
\item
  Rdata file contains a dataframe inside that has to be called abc as
  well, so open the Rdata file inside Rstudio or R, what is the file
  called? is it abc? Look under environment
\item
  In the \emph{/R/ffp\_abc.R}, the last line should be ``abc'', but the
  file name does not need to be.
\item
  If any functions uses a dataset, this could be done in two ways:

  \begin{itemize}
  \tightlist
  \item
    a dataset is a parameter for a function. In @example, can load in
    dataframes declared in dependencies.
  \item
    inside a function, if direclty use a dataframe, that data frame
    should be declared following 1 to 3 here in \emph{/data}, make sure
    it has the same name, and write: \emph{data(data\_name)} at the
    beginning of the function to explicitly load the function in.
  \end{itemize}
\end{enumerate}

\hypertarget{logging-and-printing}{%
\subsection{Logging and Printing}\label{logging-and-printing}}

Logging and printing control is as usual important. See:

\begin{enumerate}
\def\labelenumi{\arabic{enumi}.}
\tightlist
\item
  \href{https://stackoverflow.com/questions/36699272/why-is-message-a-better-choice-than-print-in-r-for-writing-a-package/36700294\#36700294}{Why
  is message() a better choice than print() in R for writing a package?}
\item
  \href{https://adv-r.hadley.nz/conditions.html}{Conditions}
\end{enumerate}

For simple package, for regression results, table outputs etc, use
\emph{print}, print statements can be suppressed by
\emph{invisible(capture.output(abc \textless-
ffy\_opt\_dtgch\_cbem4()))}.

\hypertarget{check-and-run-examples}{%
\subsection{Check and Run Examples}\label{check-and-run-examples}}

\href{http://r-pkgs.had.co.nz/check.html}{check} to make sure file
structures etc are all correct. Will also test the @examples inside r
functions in ROxygen comments.

\begin{Shaded}
\begin{Highlighting}[]
\CommentTok{# will check the file structure, but also @example in functions.}
\NormalTok{devtools}\OperatorTok{::}\KeywordTok{check}\NormalTok{(}\StringTok{'C:/Users/fan/PrjOptiAlloc'}\NormalTok{)}
\NormalTok{devtools}\OperatorTok{::}\KeywordTok{check}\NormalTok{(}\StringTok{'C:/Users/fan/PrjOptiAlloc'}\NormalTok{, }\DataTypeTok{manual=}\OtherTok{FALSE}\NormalTok{)}
\NormalTok{devtools}\OperatorTok{::}\KeywordTok{run_examples}\NormalTok{(}\StringTok{'C:/Users/fan/PrjOptiAlloc'}\NormalTok{)}
\NormalTok{devtools}\OperatorTok{::}\KeywordTok{build}\NormalTok{(}\StringTok{'C:/Users/fan/PrjOptiAlloc'}\NormalTok{)}
\end{Highlighting}
\end{Shaded}

\hypertarget{add-depenencies}{%
\subsection{Add Depenencies}\label{add-depenencies}}

Assuming that we have used roxygen2 formats to write functions, now
generate \emph{.Rd} automatically with the \emph{document()} function.
This should create a \emph{/man} folder in which various \emph{.Rd}
files are stored. Note that their .Rd are for specific R functions, not
for files that contain multiple functions.

\begin{Shaded}
\begin{Highlighting}[]
\CommentTok{# add dependencies:}
\KeywordTok{setwd}\NormalTok{(}\StringTok{'C:/Users/fan/PrjOptiAlloc'}\NormalTok{)}
\NormalTok{usethis}\OperatorTok{::}\KeywordTok{use_package}\NormalTok{(}\StringTok{"tidyr"}\NormalTok{)}
\NormalTok{usethis}\OperatorTok{::}\KeywordTok{use_package}\NormalTok{(}\StringTok{"dplyr"}\NormalTok{)}
\NormalTok{usethis}\OperatorTok{::}\KeywordTok{use_package}\NormalTok{(}\StringTok{"stringr"}\NormalTok{)}
\NormalTok{usethis}\OperatorTok{::}\KeywordTok{use_package}\NormalTok{(}\StringTok{"broom"}\NormalTok{)}
\NormalTok{usethis}\OperatorTok{::}\KeywordTok{use_package}\NormalTok{(}\StringTok{"ggplot2"}\NormalTok{)}
\NormalTok{usethis}\OperatorTok{::}\KeywordTok{use_package}\NormalTok{(}\StringTok{"R4Econ"}\NormalTok{)}
\CommentTok{# document will generate new NAMESPACE, so after dependencies added, need to document.}
\CommentTok{# funciton import must be consistent with these}
\NormalTok{devtools}\OperatorTok{::}\KeywordTok{document}\NormalTok{(}\StringTok{'C:/Users/fan/PrjOptiAlloc'}\NormalTok{)}
\end{Highlighting}
\end{Shaded}

\hypertarget{vignettes-generation}{%
\subsubsection{Vignettes Generation}\label{vignettes-generation}}

To generate Vignettes, follow instructions
\href{https://kbroman.org/pkg_primer/pages/vignettes.html}{here}:

\begin{enumerate}
\def\labelenumi{\arabic{enumi}.}
\tightlist
\item
  create a vignette directory
\item
  put RMD files in there, RMD front matter should look like below
\item
  add to DESCRIPTION file:

  \begin{itemize}
  \tightlist
  \item
    Suggests: knitr, rmarkdown
  \item
    VignetteBuilder: knitr
  \end{itemize}
\item
  run \emph{devtools::build\_vignettes()}. Running build should build
  all? automatically builds vignette
\end{enumerate}

\begin{verbatim}
title: "Put the title of your vignette here"
output: rmarkdown::html_vignette
vignette: >
  %\VignetteIndexEntry{Put the title of your vignette here}
  %\VignetteEngine{knitr::rmarkdown}
  \usepackage[utf8]{inputenc}
\end{verbatim}

\begin{Shaded}
\begin{Highlighting}[]
\NormalTok{devtools}\OperatorTok{::}\KeywordTok{build_vignettes}\NormalTok{(}\StringTok{'C:/Users/fan/PrjOptiAlloc'}\NormalTok{)}
\NormalTok{devtools}\OperatorTok{::}\KeywordTok{check}\NormalTok{()}
\end{Highlighting}
\end{Shaded}

\hypertarget{r-project-build-site-with-pkgdown}{%
\subsection{R project build site with
pkgdown}\label{r-project-build-site-with-pkgdown}}

Once we have used \emph{pkgdown::build\_site()}, by default, all html
and other package presentation files and reference files are saved in
the \emph{/docs} folder, build\_site is for building site, not for
generating \emph{/man} documenation. If R files change, need to document
first, before rebuilding site.

\begin{Shaded}
\begin{Highlighting}[]
\NormalTok{pkgdown}\OperatorTok{::}\KeywordTok{build_site}\NormalTok{(}\StringTok{"C:/Users/fan/PrjOptiAlloc"}\NormalTok{)}
\end{Highlighting}
\end{Shaded}

\end{document}
