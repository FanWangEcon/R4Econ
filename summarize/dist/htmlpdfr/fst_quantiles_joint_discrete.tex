% Options for packages loaded elsewhere
\PassOptionsToPackage{unicode}{hyperref}
\PassOptionsToPackage{hyphens}{url}
\PassOptionsToPackage{dvipsnames,svgnames*,x11names*}{xcolor}
%
\documentclass[
]{article}
\usepackage{lmodern}
\usepackage{amssymb,amsmath}
\usepackage{ifxetex,ifluatex}
\ifnum 0\ifxetex 1\fi\ifluatex 1\fi=0 % if pdftex
  \usepackage[T1]{fontenc}
  \usepackage[utf8]{inputenc}
  \usepackage{textcomp} % provide euro and other symbols
\else % if luatex or xetex
  \usepackage{unicode-math}
  \defaultfontfeatures{Scale=MatchLowercase}
  \defaultfontfeatures[\rmfamily]{Ligatures=TeX,Scale=1}
\fi
% Use upquote if available, for straight quotes in verbatim environments
\IfFileExists{upquote.sty}{\usepackage{upquote}}{}
\IfFileExists{microtype.sty}{% use microtype if available
  \usepackage[]{microtype}
  \UseMicrotypeSet[protrusion]{basicmath} % disable protrusion for tt fonts
}{}
\makeatletter
\@ifundefined{KOMAClassName}{% if non-KOMA class
  \IfFileExists{parskip.sty}{%
    \usepackage{parskip}
  }{% else
    \setlength{\parindent}{0pt}
    \setlength{\parskip}{6pt plus 2pt minus 1pt}}
}{% if KOMA class
  \KOMAoptions{parskip=half}}
\makeatother
\usepackage{xcolor}
\IfFileExists{xurl.sty}{\usepackage{xurl}}{} % add URL line breaks if available
\IfFileExists{bookmark.sty}{\usepackage{bookmark}}{\usepackage{hyperref}}
\hypersetup{
  pdftitle={Joint Quantiles from Multiple Continuous Variables as a Categorical Variable with Linear Index},
  pdfauthor={Fan Wang},
  colorlinks=true,
  linkcolor=Maroon,
  filecolor=Maroon,
  citecolor=Blue,
  urlcolor=blue,
  pdfcreator={LaTeX via pandoc}}
\urlstyle{same} % disable monospaced font for URLs
\usepackage[margin=1in]{geometry}
\usepackage{color}
\usepackage{fancyvrb}
\newcommand{\VerbBar}{|}
\newcommand{\VERB}{\Verb[commandchars=\\\{\}]}
\DefineVerbatimEnvironment{Highlighting}{Verbatim}{commandchars=\\\{\}}
% Add ',fontsize=\small' for more characters per line
\usepackage{framed}
\definecolor{shadecolor}{RGB}{248,248,248}
\newenvironment{Shaded}{\begin{snugshade}}{\end{snugshade}}
\newcommand{\AlertTok}[1]{\textcolor[rgb]{0.94,0.16,0.16}{#1}}
\newcommand{\AnnotationTok}[1]{\textcolor[rgb]{0.56,0.35,0.01}{\textbf{\textit{#1}}}}
\newcommand{\AttributeTok}[1]{\textcolor[rgb]{0.77,0.63,0.00}{#1}}
\newcommand{\BaseNTok}[1]{\textcolor[rgb]{0.00,0.00,0.81}{#1}}
\newcommand{\BuiltInTok}[1]{#1}
\newcommand{\CharTok}[1]{\textcolor[rgb]{0.31,0.60,0.02}{#1}}
\newcommand{\CommentTok}[1]{\textcolor[rgb]{0.56,0.35,0.01}{\textit{#1}}}
\newcommand{\CommentVarTok}[1]{\textcolor[rgb]{0.56,0.35,0.01}{\textbf{\textit{#1}}}}
\newcommand{\ConstantTok}[1]{\textcolor[rgb]{0.00,0.00,0.00}{#1}}
\newcommand{\ControlFlowTok}[1]{\textcolor[rgb]{0.13,0.29,0.53}{\textbf{#1}}}
\newcommand{\DataTypeTok}[1]{\textcolor[rgb]{0.13,0.29,0.53}{#1}}
\newcommand{\DecValTok}[1]{\textcolor[rgb]{0.00,0.00,0.81}{#1}}
\newcommand{\DocumentationTok}[1]{\textcolor[rgb]{0.56,0.35,0.01}{\textbf{\textit{#1}}}}
\newcommand{\ErrorTok}[1]{\textcolor[rgb]{0.64,0.00,0.00}{\textbf{#1}}}
\newcommand{\ExtensionTok}[1]{#1}
\newcommand{\FloatTok}[1]{\textcolor[rgb]{0.00,0.00,0.81}{#1}}
\newcommand{\FunctionTok}[1]{\textcolor[rgb]{0.00,0.00,0.00}{#1}}
\newcommand{\ImportTok}[1]{#1}
\newcommand{\InformationTok}[1]{\textcolor[rgb]{0.56,0.35,0.01}{\textbf{\textit{#1}}}}
\newcommand{\KeywordTok}[1]{\textcolor[rgb]{0.13,0.29,0.53}{\textbf{#1}}}
\newcommand{\NormalTok}[1]{#1}
\newcommand{\OperatorTok}[1]{\textcolor[rgb]{0.81,0.36,0.00}{\textbf{#1}}}
\newcommand{\OtherTok}[1]{\textcolor[rgb]{0.56,0.35,0.01}{#1}}
\newcommand{\PreprocessorTok}[1]{\textcolor[rgb]{0.56,0.35,0.01}{\textit{#1}}}
\newcommand{\RegionMarkerTok}[1]{#1}
\newcommand{\SpecialCharTok}[1]{\textcolor[rgb]{0.00,0.00,0.00}{#1}}
\newcommand{\SpecialStringTok}[1]{\textcolor[rgb]{0.31,0.60,0.02}{#1}}
\newcommand{\StringTok}[1]{\textcolor[rgb]{0.31,0.60,0.02}{#1}}
\newcommand{\VariableTok}[1]{\textcolor[rgb]{0.00,0.00,0.00}{#1}}
\newcommand{\VerbatimStringTok}[1]{\textcolor[rgb]{0.31,0.60,0.02}{#1}}
\newcommand{\WarningTok}[1]{\textcolor[rgb]{0.56,0.35,0.01}{\textbf{\textit{#1}}}}
\usepackage{graphicx,grffile}
\makeatletter
\def\maxwidth{\ifdim\Gin@nat@width>\linewidth\linewidth\else\Gin@nat@width\fi}
\def\maxheight{\ifdim\Gin@nat@height>\textheight\textheight\else\Gin@nat@height\fi}
\makeatother
% Scale images if necessary, so that they will not overflow the page
% margins by default, and it is still possible to overwrite the defaults
% using explicit options in \includegraphics[width, height, ...]{}
\setkeys{Gin}{width=\maxwidth,height=\maxheight,keepaspectratio}
% Set default figure placement to htbp
\makeatletter
\def\fps@figure{htbp}
\makeatother
\setlength{\emergencystretch}{3em} % prevent overfull lines
\providecommand{\tightlist}{%
  \setlength{\itemsep}{0pt}\setlength{\parskip}{0pt}}
\setcounter{secnumdepth}{-\maxdimen} % remove section numbering
\usepackage{bbm}
\usepackage{booktabs}
\usepackage{longtable}
\usepackage{array}
\usepackage{multirow}
\usepackage{wrapfig}
\usepackage{float}
\usepackage{colortbl}
\usepackage{pdflscape}
\usepackage{tabu}
\usepackage{threeparttable}
\usepackage{threeparttablex}
\usepackage[normalem]{ulem}
\usepackage{makecell}
\usepackage{xcolor}
% \setcounter{secnumdepth}{5}
% \setcounter{tocdepth}{5}

\title{Joint Quantiles from Multiple Continuous Variables as a Categorical
Variable with Linear Index}
\author{Fan Wang}
\date{2020-04-01}

\begin{document}
\maketitle

{
\hypersetup{linkcolor=}
\setcounter{tocdepth}{3}
\tableofcontents
}
\hypertarget{joint-quantiles-from-continuous}{%
\subsubsection{Joint Quantiles from
Continuous}\label{joint-quantiles-from-continuous}}

\begin{quote}
Go to the
\href{https://github.com/FanWangEcon//R4Econ/blob/master/summarize/dist//fst_quantiles_joint_discrete.Rmd}{\textbf{RMD}},
\href{https://github.com/FanWangEcon//R4Econ/blob/master/summarize/dist//htmlpdfr/fst_quantiles_joint_discrete.R}{\textbf{R}},
\href{https://github.com/FanWangEcon//R4Econ/blob/master/summarize/dist//htmlpdfr/fst_quantiles_joint_discrete.pdf}{\textbf{PDF}},
or
\href{https://fanwangecon.github.io//R4Econ/summarize/dist//htmlpdfr/fst_quantiles_joint_discrete.html}{\textbf{HTML}}
version of this file. Go back to
\href{http://fanwangecon.github.io/}{fan}'s
\href{https://fanwangecon.github.io/REconTools/}{REconTools} Package,
\href{https://fanwangecon.github.io/R4Econ/}{R4Econ} Repository
(\href{https://fanwangecon.github.io/R4Econ/bookdown}{bookdown site}),
or \href{https://fanwangecon.github.io/Stat4Econ/}{Intro Stats with R}
Repository.
\end{quote}

There are multiple or a single continuous variables. Find which quantile
each observation belongs to for each of the variables. Then also
generate a joint/interaction variable of all combinations of quantiles
from different variables.

The program has these features:

\begin{enumerate}
\def\labelenumi{\arabic{enumi}.}
\tightlist
\item
  Quantiles breaks are generated based on group\_by characteristics,
  meaning quantiles for individual level characteristics when data is
  panel
\item
  Quantiles variables apply to full panel at within-group observation
  levels.
\item
  Robust to non-unique breaks for quantiles (non-unique grouped
  together)
\item
  Quantile categories have detailed labeling (specifying which
  non-unique groupings belong to quantile)
\end{enumerate}

When joining multiple quantile variables together:

\begin{enumerate}
\def\labelenumi{\arabic{enumi}.}
\tightlist
\item
  First check if only calculate quantiles at observations where all
  quantile base variables are not null
\item
  Calculate Quantiles for each variable, with different quantile levels
  for sub-groups of variables
\item
  Summary statistics by mulltiple quantile-categorical variables,
  summary
\end{enumerate}

\hypertarget{build-program}{%
\paragraph{Build Program}\label{build-program}}

\hypertarget{support-functions}{%
\subparagraph{Support Functions}\label{support-functions}}

\begin{Shaded}
\begin{Highlighting}[]
\CommentTok{# Quantiles for any variable}
\NormalTok{gen_quantiles <-}\StringTok{ }\ControlFlowTok{function}\NormalTok{(var, df, }\DataTypeTok{prob=}\KeywordTok{c}\NormalTok{(}\FloatTok{0.25}\NormalTok{, }\FloatTok{0.50}\NormalTok{, }\FloatTok{0.75}\NormalTok{)) \{}
  \KeywordTok{enframe}\NormalTok{(}\KeywordTok{quantile}\NormalTok{(}\KeywordTok{as.numeric}\NormalTok{(df[[var]]), }
\NormalTok{                   prob, }\DataTypeTok{na.rm=}\OtherTok{TRUE}\NormalTok{), }\StringTok{'quant.perc'}\NormalTok{, var)}
\NormalTok{\}}
\CommentTok{# Support Functions for Variable Suffix}
\NormalTok{f_Q_suffix <-}\StringTok{ }\ControlFlowTok{function}\NormalTok{(seq.quantiles) \{}
\NormalTok{  quantile.suffix <-}\StringTok{ }\KeywordTok{paste0}\NormalTok{(}\StringTok{'Qs'}\NormalTok{, }\KeywordTok{min}\NormalTok{(seq.quantiles),}
                            \StringTok{'e'}\NormalTok{, }\KeywordTok{max}\NormalTok{(seq.quantiles),}
                            \StringTok{'n'}\NormalTok{, (}\KeywordTok{length}\NormalTok{(seq.quantiles)}\OperatorTok{-}\DecValTok{1}\NormalTok{))}
\NormalTok{\}}
\CommentTok{# Support Functions for Quantile Labeling}
\NormalTok{f_Q_label <-}\StringTok{ }\ControlFlowTok{function}\NormalTok{(arr.quantiles,}
\NormalTok{                      arr.sort.unique.quantile,}
\NormalTok{                      seq.quantiles) \{}
  \KeywordTok{paste0}\NormalTok{(}\StringTok{'('}\NormalTok{,}
         \KeywordTok{paste0}\NormalTok{(}\KeywordTok{which}\NormalTok{(arr.quantiles }\OperatorTok\StringTok{ }
\StringTok{                        }\NormalTok{arr.sort.unique.quantile), }\DataTypeTok{collapse=}\StringTok{','}\NormalTok{),}
         \StringTok{') of '}\NormalTok{, }\KeywordTok{f_Q_suffix}\NormalTok{(seq.quantiles))}
\NormalTok{\}}
\CommentTok{# Generate New Variable Names with Quantile Suffix}
\NormalTok{f_var_rename <-}\StringTok{ }\ControlFlowTok{function}\NormalTok{(name, seq.quantiles) \{}
\NormalTok{  quantile.suffix <-}\StringTok{ }\KeywordTok{paste0}\NormalTok{(}\StringTok{'_'}\NormalTok{, }\KeywordTok{f_Q_suffix}\NormalTok{(seq.quantiles))}
  \KeywordTok{return}\NormalTok{(}\KeywordTok{sub}\NormalTok{(}\StringTok{'_q'}\NormalTok{, quantile.suffix, name))}
\NormalTok{\}}

\CommentTok{# Check Are Values within Group By Unique? If not, STOP}
\NormalTok{f_check_distinct_ingroup <-}\StringTok{ }
\StringTok{  }\ControlFlowTok{function}\NormalTok{(df, vars.group_by, vars.values_in_group) \{}
  
\NormalTok{  df.uniqus.in.group <-}\StringTok{ }\NormalTok{df }\OperatorTok\StringTok{ }\KeywordTok{group_by}\NormalTok{(}\OperatorTok{!!!}\KeywordTok{syms}\NormalTok{(vars.group_by)) }\OperatorTok
\StringTok{    }\KeywordTok{mutate}\NormalTok{(}\DataTypeTok{quant_vars_paste =} 
             \KeywordTok{paste}\NormalTok{(}\OperatorTok{!!!}\NormalTok{(}\KeywordTok{syms}\NormalTok{(vars.values_in_group)), }\DataTypeTok{sep=}\StringTok{'-'}\NormalTok{)) }\OperatorTok
\StringTok{    }\KeywordTok{mutate}\NormalTok{(}\DataTypeTok{unique_in_group =} \KeywordTok{n_distinct}\NormalTok{(quant_vars_paste)) }\OperatorTok
\StringTok{    }\KeywordTok{slice}\NormalTok{(1L) }\OperatorTok
\StringTok{    }\KeywordTok{ungroup}\NormalTok{() }\OperatorTok
\StringTok{    }\KeywordTok{group_by}\NormalTok{(unique_in_group) }\OperatorTok
\StringTok{    }\KeywordTok{summarise}\NormalTok{(}\DataTypeTok{n=}\KeywordTok{n}\NormalTok{())}
  
  \ControlFlowTok{if}\NormalTok{ (}\KeywordTok{sum}\NormalTok{(df.uniqus.in.group}\OperatorTok{$}\NormalTok{unique_in_group) }\OperatorTok{>}\StringTok{ }\DecValTok{1}\NormalTok{) \{}
    \KeywordTok{print}\NormalTok{(df.uniqus.in.group)}
    \KeywordTok{print}\NormalTok{(}\KeywordTok{paste}\NormalTok{(}\StringTok{'vars.values_in_group'}\NormalTok{, vars.values_in_group, }\DataTypeTok{sep=}\StringTok{':'}\NormalTok{))}
    \KeywordTok{print}\NormalTok{(}\KeywordTok{paste}\NormalTok{(}\StringTok{'vars.group_by'}\NormalTok{, vars.group_by, }\DataTypeTok{sep=}\StringTok{':'}\NormalTok{))}
    \KeywordTok{stop}\NormalTok{(}\KeywordTok{paste0}\NormalTok{(}\StringTok{"The variables for which quantiles are to be',}
\StringTok{                'taken are not identical within the group variables"}\NormalTok{))}
\NormalTok{  \}}
\NormalTok{\}}
\end{Highlighting}
\end{Shaded}

\hypertarget{data-slicing-and-quantile-generation}{%
\subparagraph{Data Slicing and Quantile
Generation}\label{data-slicing-and-quantile-generation}}

\begin{itemize}
\tightlist
\item
  Function 1: generate quantiles based on group-specific
  characteristics. the groups could be at the panel observation level as
  well.
\end{itemize}

\begin{Shaded}
\begin{Highlighting}[]
\CommentTok{# First Step, given groups, generate quantiles based on group characteristics}
\CommentTok{# vars.cts2quantile <- c('wealthIdx', 'hgt0', 'wgt0')}
\CommentTok{# seq.quantiles <- c(0, 0.3333, 0.6666, 1.0)}
\CommentTok{# vars.group_by <- c('indi.id')}
\CommentTok{# vars.arrange <- c('indi.id', 'svymthRound')}
\CommentTok{# vars.continuous <- c('wealthIdx', 'hgt0', 'wgt0')}
\NormalTok{df_sliced_quantiles <-}\StringTok{ }\ControlFlowTok{function}\NormalTok{(df, vars.cts2quantile, seq.quantiles,}
\NormalTok{                                vars.group_by, vars.arrange) \{}

    \CommentTok{# Slicing data}
\NormalTok{    df.grp.L1 <-}\StringTok{ }\NormalTok{df }\OperatorTok\StringTok{ }\KeywordTok{group_by}\NormalTok{(}\OperatorTok{!!!}\KeywordTok{syms}\NormalTok{(vars.group_by)) }\OperatorTok\StringTok{ }
\StringTok{      }\KeywordTok{arrange}\NormalTok{(}\OperatorTok{!!!}\KeywordTok{syms}\NormalTok{(vars.arrange)) }\OperatorTok\StringTok{ }\KeywordTok{slice}\NormalTok{(1L) }\OperatorTok\StringTok{ }\KeywordTok{ungroup}\NormalTok{()}

    \CommentTok{# Quantiles based on sliced data}
\NormalTok{    df.sliced.quantiles <-}\StringTok{ }
\StringTok{      }\KeywordTok{lapply}\NormalTok{(vars.cts2quantile, gen_quantiles, }\DataTypeTok{df=}\NormalTok{df.grp.L1, }\DataTypeTok{prob=}\NormalTok{seq.quantiles) }\OperatorTok\StringTok{ }
\StringTok{      }\KeywordTok{reduce}\NormalTok{(full_join)}

    \KeywordTok{return}\NormalTok{(}\KeywordTok{list}\NormalTok{(}\DataTypeTok{df.sliced.quantiles=}\NormalTok{df.sliced.quantiles,}
                \DataTypeTok{df.grp.L1=}\NormalTok{df.grp.L1))}
\NormalTok{\}}
\end{Highlighting}
\end{Shaded}

\hypertarget{data-cutting}{%
\subparagraph{Data Cutting}\label{data-cutting}}

\begin{itemize}
\tightlist
\item
  Function 2: cut groups for full panel dataframe based on
  group-specific characteristics quantiles.
\end{itemize}

\begin{Shaded}
\begin{Highlighting}[]
\CommentTok{# Cutting Function, Cut Continuous Variables into Quantiles with labeing}
\NormalTok{f_cut <-}\StringTok{ }\ControlFlowTok{function}\NormalTok{(var, df.sliced.quantiles, seq.quantiles, }
                  \DataTypeTok{include.lowest=}\OtherTok{TRUE}\NormalTok{, }\DataTypeTok{fan.labels=}\OtherTok{TRUE}\NormalTok{, }\DataTypeTok{print=}\OtherTok{FALSE}\NormalTok{) \{}
  
  \CommentTok{# unparsed string variable name}
\NormalTok{  var.str <-}\StringTok{ }\KeywordTok{substitute}\NormalTok{(var)}
  
  \CommentTok{# Breaks}
\NormalTok{  arr.quantiles <-}\StringTok{ }\NormalTok{df.sliced.quantiles[[var.str]]}
\NormalTok{  arr.sort.unique.quantiles <-}\StringTok{ }\KeywordTok{sort}\NormalTok{(}\KeywordTok{unique}\NormalTok{(arr.quantiles))}
  \ControlFlowTok{if}\NormalTok{ (print) \{}
    \KeywordTok{print}\NormalTok{(arr.sort.unique.quantiles)}
\NormalTok{  \}}
  
  \CommentTok{# Regular cutting With Standard Labels}
  \CommentTok{# TRUE, means the lowest group has closed bracket left and right}
\NormalTok{  var.quantile <-}\StringTok{ }\KeywordTok{cut}\NormalTok{(var, }\DataTypeTok{breaks=}\NormalTok{arr.sort.unique.quantiles, }
                      \DataTypeTok{include.lowest=}\NormalTok{include.lowest)}
  
  \CommentTok{# Use my custom labels}
  \ControlFlowTok{if}\NormalTok{ (fan.labels) \{}
\NormalTok{    levels.suffix <-}\StringTok{ }
\StringTok{      }\KeywordTok{lapply}\NormalTok{(arr.sort.unique.quantiles[}\DecValTok{1}\OperatorTok{:}\NormalTok{(}\KeywordTok{length}\NormalTok{(arr.sort.unique.quantiles)}\OperatorTok{-}\DecValTok{1}\NormalTok{)],}
\NormalTok{             f_Q_label,}
             \DataTypeTok{arr.quantiles=}\NormalTok{arr.quantiles,}
             \DataTypeTok{seq.quantiles=}\NormalTok{seq.quantiles)}
    \ControlFlowTok{if}\NormalTok{ (print) \{}
      \KeywordTok{print}\NormalTok{(levels.suffix)}
\NormalTok{    \}}
    \KeywordTok{levels}\NormalTok{(var.quantile) <-}\StringTok{ }\KeywordTok{paste0}\NormalTok{(}\KeywordTok{levels}\NormalTok{(var.quantile), }\StringTok{'; '}\NormalTok{, levels.suffix)}
\NormalTok{  \}}
  
  \CommentTok{# Return}
  \KeywordTok{return}\NormalTok{(var.quantile)}
\NormalTok{\}}
\end{Highlighting}
\end{Shaded}

\begin{Shaded}
\begin{Highlighting}[]
\CommentTok{# Combo Quantile Function}
\CommentTok{# vars.cts2quantile <- c('wealthIdx', 'hgt0', 'wgt0')}
\CommentTok{# seq.quantiles <- c(0, 0.3333, 0.6666, 1.0)}
\CommentTok{# vars.group_by <- c('indi.id')}
\CommentTok{# vars.arrange <- c('indi.id', 'svymthRound')}
\CommentTok{# vars.continuous <- c('wealthIdx', 'hgt0', 'wgt0')}
\NormalTok{df_cut_by_sliced_quantiles <-}\StringTok{ }\ControlFlowTok{function}\NormalTok{(df, vars.cts2quantile, seq.quantiles,}
\NormalTok{                                       vars.group_by, vars.arrange) \{}
  
  
  \CommentTok{# Check Are Values within Group By Unique? If not, STOP}
  \KeywordTok{f_check_distinct_ingroup}\NormalTok{(df, vars.group_by, }\DataTypeTok{vars.values_in_group=}\NormalTok{vars.cts2quantile)}
  
  \CommentTok{# First Step Slicing}
\NormalTok{  df.sliced <-}\StringTok{ }\KeywordTok{df_sliced_quantiles}\NormalTok{(df, vars.cts2quantile, }
\NormalTok{                                   seq.quantiles, vars.group_by, vars.arrange)}
  
  \CommentTok{# Second Step Generate Categorical Variables of Quantiles}
\NormalTok{  df.with.cut.quant <-}\StringTok{ }\NormalTok{df }\OperatorTok\StringTok{ }
\StringTok{    }\KeywordTok{mutate_at}\NormalTok{(vars.cts2quantile,}
              \KeywordTok{funs}\NormalTok{(}\DataTypeTok{q=}\KeywordTok{f_cut}\NormalTok{(., df.sliced}\OperatorTok{$}\NormalTok{df.sliced.quantiles,}
                           \DataTypeTok{seq.quantiles=}\NormalTok{seq.quantiles,}
                           \DataTypeTok{include.lowest=}\OtherTok{TRUE}\NormalTok{, }\DataTypeTok{fan.labels=}\OtherTok{TRUE}\NormalTok{)))}
  
  \ControlFlowTok{if}\NormalTok{ (}\KeywordTok{length}\NormalTok{(vars.cts2quantile) }\OperatorTok{>}\StringTok{ }\DecValTok{1}\NormalTok{) \{}
\NormalTok{    df.with.cut.quant <-}\StringTok{ }
\StringTok{      }\NormalTok{df.with.cut.quant }\OperatorTok
\StringTok{      }\KeywordTok{rename_at}\NormalTok{(}\KeywordTok{vars}\NormalTok{(}\KeywordTok{contains}\NormalTok{(}\StringTok{'_q'}\NormalTok{)), }
                \KeywordTok{funs}\NormalTok{(}\KeywordTok{f_var_rename}\NormalTok{(., }\DataTypeTok{seq.quantiles=}\NormalTok{seq.quantiles)))}
\NormalTok{  \} }\ControlFlowTok{else}\NormalTok{ \{}
\NormalTok{    new.var.name <-}\StringTok{ }\KeywordTok{paste0}\NormalTok{(vars.cts2quantile[}\DecValTok{1}\NormalTok{], }\StringTok{'_'}\NormalTok{, }\KeywordTok{f_Q_suffix}\NormalTok{(seq.quantiles))}
\NormalTok{    df.with.cut.quant <-}\StringTok{ }\NormalTok{df.with.cut.quant }\OperatorTok\StringTok{ }\KeywordTok{rename}\NormalTok{(}\OperatorTok{!!}\DataTypeTok{new.var.name :=}\NormalTok{ q)}
\NormalTok{  \}}
  
  \CommentTok{# Newly Generated Quantile-Cut Variables}
\NormalTok{  vars.quantile.cut <-}\StringTok{ }\NormalTok{df.with.cut.quant }\OperatorTok
\StringTok{    }\KeywordTok{select}\NormalTok{(}\KeywordTok{matches}\NormalTok{(}\KeywordTok{paste0}\NormalTok{(vars.cts2quantile, }\DataTypeTok{collapse=}\StringTok{'|'}\NormalTok{))) }\OperatorTok
\StringTok{    }\KeywordTok{select}\NormalTok{(}\KeywordTok{matches}\NormalTok{(}\KeywordTok{f_Q_suffix}\NormalTok{(seq.quantiles)))}
  
  \CommentTok{# Return}
  \KeywordTok{return}\NormalTok{(}\KeywordTok{list}\NormalTok{(}\DataTypeTok{df.with.cut.quant =}\NormalTok{ df.with.cut.quant,}
              \DataTypeTok{df.sliced.quantiles=}\NormalTok{df.sliced}\OperatorTok{$}\NormalTok{df.sliced.quantiles,}
              \DataTypeTok{df.grp.L1=}\NormalTok{df.sliced}\OperatorTok{$}\NormalTok{df.grp.L1,}
              \DataTypeTok{vars.quantile.cut=}\NormalTok{vars.quantile.cut))}
  
\NormalTok{\}}
\end{Highlighting}
\end{Shaded}

\hypertarget{different-vars-different-probabilities-joint-quantiles}{%
\subparagraph{Different Vars Different Probabilities Joint
Quantiles}\label{different-vars-different-probabilities-joint-quantiles}}

\begin{itemize}
\tightlist
\item
  Accomondate multiple continuousv ariables
\item
  Different percentiles
\item
  list of lists
\item
  generate joint categorical variables
\item
  keep only values that exist for all quantile base vars
\end{itemize}

\begin{Shaded}
\begin{Highlighting}[]
\CommentTok{# Function to handle list inputs with different quantiles vars and probabilities}
\NormalTok{df_cut_by_sliced_quantiles_grps <-}\StringTok{ }
\StringTok{  }\ControlFlowTok{function}\NormalTok{(quantile.grp.list, df, vars.group_by, vars.arrange) \{}
\NormalTok{    vars.cts2quantile <-}\StringTok{ }\NormalTok{quantile.grp.list}\OperatorTok{$}\NormalTok{vars}
\NormalTok{    seq.quantiles <-}\StringTok{ }\NormalTok{quantile.grp.list}\OperatorTok{$}\NormalTok{prob}
    \KeywordTok{return}\NormalTok{(}\KeywordTok{df_cut_by_sliced_quantiles}\NormalTok{(}
\NormalTok{      df, vars.cts2quantile, seq.quantiles, vars.group_by, vars.arrange))}
\NormalTok{  \}}
\CommentTok{# Show Results}
\NormalTok{df_cut_by_sliced_quantiles_joint_results_grped <-}\StringTok{ }
\StringTok{  }\ControlFlowTok{function}\NormalTok{(df.with.cut.quant.all, vars.cts2quantile, vars.group_by, vars.arrange,}
\NormalTok{           vars.quantile.cut.all, var.qjnt.grp.idx) \{}
    \CommentTok{# Show ALL}
\NormalTok{    df.group.panel.cnt.mean <-}\StringTok{ }\NormalTok{df.with.cut.quant.all }\OperatorTok\StringTok{ }
\StringTok{      }\KeywordTok{group_by}\NormalTok{(}\OperatorTok{!!!}\KeywordTok{syms}\NormalTok{(vars.quantile.cut.all), }
               \OperatorTok{!!}\KeywordTok{sym}\NormalTok{(var.qjnt.grp.idx)) }\OperatorTok
\StringTok{      }\KeywordTok{summarise_at}\NormalTok{(vars.cts2quantile, }\KeywordTok{funs}\NormalTok{(mean, }\KeywordTok{n}\NormalTok{()))}
    
    \CommentTok{# Show Based on SLicing first}
\NormalTok{    df.group.slice1.cnt.mean <-}\StringTok{ }\NormalTok{df.with.cut.quant.all }\OperatorTok\StringTok{ }
\StringTok{      }\KeywordTok{group_by}\NormalTok{(}\OperatorTok{!!!}\KeywordTok{syms}\NormalTok{(vars.group_by)) }\OperatorTok\StringTok{ }
\StringTok{      }\KeywordTok{arrange}\NormalTok{(}\OperatorTok{!!!}\KeywordTok{syms}\NormalTok{(vars.arrange)) }\OperatorTok\StringTok{ }\KeywordTok{slice}\NormalTok{(1L) }\OperatorTok
\StringTok{      }\KeywordTok{group_by}\NormalTok{(}\OperatorTok{!!!}\KeywordTok{syms}\NormalTok{(vars.quantile.cut.all), }\OperatorTok{!!}\KeywordTok{sym}\NormalTok{(var.qjnt.grp.idx)) }\OperatorTok
\StringTok{      }\KeywordTok{summarise_at}\NormalTok{(vars.cts2quantile, }\KeywordTok{funs}\NormalTok{(mean, }\KeywordTok{n}\NormalTok{()))}
    
    \KeywordTok{return}\NormalTok{(}\KeywordTok{list}\NormalTok{(}\DataTypeTok{df.group.panel.cnt.mean=}\NormalTok{df.group.panel.cnt.mean,}
                \DataTypeTok{df.group.slice1.cnt.mean=}\NormalTok{df.group.slice1.cnt.mean))}
\NormalTok{  \}}
\end{Highlighting}
\end{Shaded}

\begin{Shaded}
\begin{Highlighting}[]
\CommentTok{# # Joint Quantile Group Name}
\CommentTok{# var.qjnt.grp.idx <- 'group.index'}
\CommentTok{# # Generate Categorical Variables of Quantiles}
\CommentTok{# vars.group_by <- c('indi.id')}
\CommentTok{# vars.arrange <- c('indi.id', 'svymthRound')}
\CommentTok{# # Quantile Variables and Quantiles}
\CommentTok{# vars.cts2quantile.wealth <- c('wealthIdx')}
\CommentTok{# seq.quantiles.wealth <- c(0, .5, 1.0)}
\CommentTok{# vars.cts2quantile.wgthgt <- c('hgt0', 'wgt0')}
\CommentTok{# seq.quantiles.wgthgt <- c(0, .3333, 0.6666, 1.0)}
\CommentTok{# drop.any.quantile.na <- TRUE}
\CommentTok{# # collect to list}
\CommentTok{# list.cts2quantile <- list(list(vars=vars.cts2quantile.wealth,}
\CommentTok{#                                prob=seq.quantiles.wealth),}
\CommentTok{#                           list(vars=vars.cts2quantile.wgthgt,}
\CommentTok{#                                prob=seq.quantiles.wgthgt))}

\NormalTok{df_cut_by_sliced_quantiles_joint <-}\StringTok{ }
\StringTok{  }\ControlFlowTok{function}\NormalTok{(df, var.qjnt.grp.idx,}
\NormalTok{           list.cts2quantile,}
\NormalTok{           vars.group_by, vars.arrange,}
           \DataTypeTok{drop.any.quantile.na =} \OtherTok{TRUE}\NormalTok{,}
           \DataTypeTok{toprint =} \OtherTok{TRUE}\NormalTok{) \{}
    
    \CommentTok{#  Original dimensions}
    \ControlFlowTok{if}\NormalTok{(toprint) \{}
      \KeywordTok{print}\NormalTok{(}\KeywordTok{dim}\NormalTok{(df))}
\NormalTok{    \}}
    
    \CommentTok{# All Continuous Variables from lists}
\NormalTok{    vars.cts2quantile <-}\StringTok{ }\KeywordTok{unlist}\NormalTok{(}\KeywordTok{lapply}\NormalTok{(list.cts2quantile, }
                                       \ControlFlowTok{function}\NormalTok{(elist) elist}\OperatorTok{$}\NormalTok{vars))}
\NormalTok{    vars.cts2quantile}
    
    \CommentTok{# Keep only if not NA for all Quantile variables}
    \ControlFlowTok{if}\NormalTok{ (drop.any.quantile.na) \{}
\NormalTok{      df.select <-}\StringTok{ }\NormalTok{df }\OperatorTok\StringTok{ }\KeywordTok{drop_na}\NormalTok{(}\KeywordTok{c}\NormalTok{(vars.group_by, vars.arrange, vars.cts2quantile))}
\NormalTok{    \} }\ControlFlowTok{else}\NormalTok{ \{}
\NormalTok{      df.select <-}\StringTok{ }\NormalTok{df}
\NormalTok{    \}}
    
    \ControlFlowTok{if}\NormalTok{(toprint) \{}
      \KeywordTok{print}\NormalTok{(}\KeywordTok{dim}\NormalTok{(df.select))}
\NormalTok{    \}}
    
    \CommentTok{# Apply qunatile function to all elements of list of list}
\NormalTok{    df.cut.list <-}\StringTok{ }
\StringTok{      }\KeywordTok{lapply}\NormalTok{(list.cts2quantile, df_cut_by_sliced_quantiles_grps,}
             \DataTypeTok{df=}\NormalTok{df.select, }\DataTypeTok{vars.group_by=}\NormalTok{vars.group_by, }\DataTypeTok{vars.arrange=}\NormalTok{vars.arrange)}
    
    \CommentTok{# Reduce Resulting Core Panel Matrix Together}
\NormalTok{    df.with.cut.quant.all <-}\StringTok{ }
\StringTok{      }\KeywordTok{lapply}\NormalTok{(df.cut.list, }\ControlFlowTok{function}\NormalTok{(elist) elist}\OperatorTok{$}\NormalTok{df.with.cut.quant) }\OperatorTok\StringTok{ }\KeywordTok{reduce}\NormalTok{(left_join)}
\NormalTok{    df.sliced.quantiles.all <-}\StringTok{ }
\StringTok{      }\KeywordTok{lapply}\NormalTok{(df.cut.list, }\ControlFlowTok{function}\NormalTok{(elist) elist}\OperatorTok{$}\NormalTok{df.sliced.quantiles)}
    
    \ControlFlowTok{if}\NormalTok{(toprint) \{}
      \KeywordTok{print}\NormalTok{(}\KeywordTok{dim}\NormalTok{(df.with.cut.quant.all))}
\NormalTok{    \}}
    
    \CommentTok{# Obrain Newly Created Quantile Group Variables}
\NormalTok{    vars.quantile.cut.all <-}\StringTok{ }
\StringTok{      }\KeywordTok{unlist}\NormalTok{(}\KeywordTok{lapply}\NormalTok{(df.cut.list, }\ControlFlowTok{function}\NormalTok{(elist) }\KeywordTok{names}\NormalTok{(elist}\OperatorTok{$}\NormalTok{vars.quantile.cut)))}
    \ControlFlowTok{if}\NormalTok{(toprint) \{}
      \KeywordTok{print}\NormalTok{(vars.quantile.cut.all)}
      \KeywordTok{print}\NormalTok{(}\KeywordTok{summary}\NormalTok{(df.with.cut.quant.all }\OperatorTok\StringTok{ }\KeywordTok{select}\NormalTok{(}\KeywordTok{one_of}\NormalTok{(vars.quantile.cut.all))))}
\NormalTok{    \}}
    
    \CommentTok{# Generate Joint Quantile Index Variable}
\NormalTok{    df.with.cut.quant.all <-}\StringTok{ }\NormalTok{df.with.cut.quant.all }\OperatorTok\StringTok{ }
\StringTok{      }\KeywordTok{mutate}\NormalTok{(}\OperatorTok{!!}\DataTypeTok{var.qjnt.grp.idx :=} \KeywordTok{group_indices}\NormalTok{(., }\OperatorTok{!!!}\KeywordTok{syms}\NormalTok{(vars.quantile.cut.all)))}
    
    \CommentTok{# Quantile Groups}
\NormalTok{    arr.group.idx <-}\StringTok{ }\KeywordTok{t}\NormalTok{(}\KeywordTok{sort}\NormalTok{(}\KeywordTok{unique}\NormalTok{(df.with.cut.quant.all[[var.qjnt.grp.idx]])))}
    
    \CommentTok{# Results Display}
\NormalTok{    df.group.print <-}\StringTok{ }\KeywordTok{df_cut_by_sliced_quantiles_joint_results_grped}\NormalTok{(}
\NormalTok{      df.with.cut.quant.all, vars.cts2quantile,}
\NormalTok{      vars.group_by, vars.arrange,}
\NormalTok{      vars.quantile.cut.all, var.qjnt.grp.idx)}
    
    \CommentTok{# list to Return}
    \CommentTok{# These returns are the same as returns earlier: df_cut_by_sliced_quantiles}
    \CommentTok{# Except that they are combined together}
    \KeywordTok{return}\NormalTok{(}\KeywordTok{list}\NormalTok{(}\DataTypeTok{df.with.cut.quant =}\NormalTok{ df.with.cut.quant.all,}
                \DataTypeTok{df.sliced.quantiles =}\NormalTok{ df.sliced.quantiles.all,}
                \DataTypeTok{df.grp.L1 =}\NormalTok{ (df.cut.list[[}\DecValTok{1}\NormalTok{]])}\OperatorTok{$}\NormalTok{df.grp.L1,}
                \DataTypeTok{vars.quantile.cut =} 
\NormalTok{                  vars.quantile.cut.all,}
                \DataTypeTok{df.group.panel.cnt.mean =} 
\NormalTok{                  df.group.print}\OperatorTok{$}\NormalTok{df.group.panel.cnt.mean,}
                \DataTypeTok{df.group.slice1.cnt.mean =} 
\NormalTok{                  df.group.print}\OperatorTok{$}\NormalTok{df.group.slice1.cnt.mean))}
    
\NormalTok{\}}
\end{Highlighting}
\end{Shaded}

\hypertarget{program-testing}{%
\paragraph{Program Testing}\label{program-testing}}

Load Data

\begin{Shaded}
\begin{Highlighting}[]
\CommentTok{# Library}
\KeywordTok{library}\NormalTok{(tidyverse)}

\CommentTok{# Load Sample Data}
\KeywordTok{setwd}\NormalTok{(}\StringTok{'C:/Users/fan/R4Econ/_data/'}\NormalTok{)}
\NormalTok{df <-}\StringTok{ }\KeywordTok{read_csv}\NormalTok{(}\StringTok{'height_weight.csv'}\NormalTok{)}
\end{Highlighting}
\end{Shaded}

\begin{verbatim}
## Parsed with column specification:
## cols(
##   S.country = col_character(),
##   vil.id = col_double(),
##   indi.id = col_double(),
##   sex = col_character(),
##   svymthRound = col_double(),
##   momEdu = col_double(),
##   wealthIdx = col_double(),
##   hgt = col_double(),
##   wgt = col_double(),
##   hgt0 = col_double(),
##   wgt0 = col_double(),
##   prot = col_double(),
##   cal = col_double(),
##   p.A.prot = col_double(),
##   p.A.nProt = col_double()
## )
\end{verbatim}

\hypertarget{hgt0-3-groups}{%
\subparagraph{Hgt0 3 Groups}\label{hgt0-3-groups}}

\begin{Shaded}
\begin{Highlighting}[]
\CommentTok{# Joint Quantile Group Name}
\NormalTok{var.qjnt.grp.idx <-}\StringTok{ 'group.index'}
\NormalTok{list.cts2quantile <-}\StringTok{ }\KeywordTok{list}\NormalTok{(}\KeywordTok{list}\NormalTok{(}\DataTypeTok{vars=}\KeywordTok{c}\NormalTok{(}\StringTok{'hgt0'}\NormalTok{), }\DataTypeTok{prob=}\KeywordTok{c}\NormalTok{(}\DecValTok{0}\NormalTok{, }\FloatTok{.3333}\NormalTok{, }\FloatTok{0.6666}\NormalTok{, }\FloatTok{1.0}\NormalTok{)))}
\NormalTok{results <-}\StringTok{ }\KeywordTok{df_cut_by_sliced_quantiles_joint}\NormalTok{(}
\NormalTok{  df, var.qjnt.grp.idx, list.cts2quantile,}
  \DataTypeTok{vars.group_by =} \KeywordTok{c}\NormalTok{(}\StringTok{'indi.id'}\NormalTok{), }\DataTypeTok{vars.arrange =} \KeywordTok{c}\NormalTok{(}\StringTok{'indi.id'}\NormalTok{, }\StringTok{'svymthRound'}\NormalTok{),}
  \DataTypeTok{drop.any.quantile.na =} \OtherTok{TRUE}\NormalTok{, }\DataTypeTok{toprint =} \OtherTok{FALSE}\NormalTok{)}
\CommentTok{# Show Results}
\NormalTok{results}\OperatorTok{$}\NormalTok{df.group.slice1.cnt.mean}
\end{Highlighting}
\end{Shaded}

\begin{verbatim}
## # A tibble: 3 x 4
## # Groups:   hgt0_Qs0e1n3 [3]
##   hgt0_Qs0e1n3                group.index  mean     n
##   <fct>                             <int> <dbl> <int>
## 1 [40.6,48.5]; (1) of Qs0e1n3           1  47.0   580
## 2 (48.5,50.2]; (2) of Qs0e1n3           2  49.4   561
## 3 (50.2,58]; (3) of Qs0e1n3             3  51.7   568
\end{verbatim}

\hypertarget{wealth-5-groups-guatemala}{%
\subparagraph{Wealth 5 Groups
Guatemala}\label{wealth-5-groups-guatemala}}

\begin{Shaded}
\begin{Highlighting}[]
\CommentTok{# Joint Quantile Group Name}
\NormalTok{var.qjnt.grp.idx <-}\StringTok{ 'wltQuintle.index'}
\NormalTok{list.cts2quantile <-}\StringTok{ }\KeywordTok{list}\NormalTok{(}\KeywordTok{list}\NormalTok{(}\DataTypeTok{vars=}\KeywordTok{c}\NormalTok{(}\StringTok{'wealthIdx'}\NormalTok{), }\DataTypeTok{prob=}\KeywordTok{seq}\NormalTok{(}\DecValTok{0}\NormalTok{, }\FloatTok{1.0}\NormalTok{, }\FloatTok{0.20}\NormalTok{)))}
\NormalTok{results <-}\StringTok{ }\KeywordTok{df_cut_by_sliced_quantiles_joint}\NormalTok{((}
\NormalTok{  df }\OperatorTok\StringTok{ }\KeywordTok{filter}\NormalTok{(S.country }\OperatorTok{==}\StringTok{ 'Guatemala'}\NormalTok{)),}
\NormalTok{  var.qjnt.grp.idx, list.cts2quantile,}
  \DataTypeTok{vars.group_by =} \KeywordTok{c}\NormalTok{(}\StringTok{'indi.id'}\NormalTok{), }\DataTypeTok{vars.arrange =} \KeywordTok{c}\NormalTok{(}\StringTok{'indi.id'}\NormalTok{, }\StringTok{'svymthRound'}\NormalTok{),}
  \DataTypeTok{drop.any.quantile.na =} \OtherTok{TRUE}\NormalTok{, }\DataTypeTok{toprint =} \OtherTok{FALSE}\NormalTok{)}
\CommentTok{# Show Results}
\NormalTok{results}\OperatorTok{$}\NormalTok{df.group.slice1.cnt.mean}
\end{Highlighting}
\end{Shaded}

\begin{verbatim}
## # A tibble: 5 x 4
## # Groups:   wealthIdx_Qs0e1n5 [5]
##   wealthIdx_Qs0e1n5         wltQuintle.index  mean     n
##   <fct>                                <int> <dbl> <int>
## 1 [1,1.6]; (1) of Qs0e1n5                  1  1.25   151
## 2 (1.6,2.1]; (2) of Qs0e1n5                2  1.82   139
## 3 (2.1,2.3]; (3) of Qs0e1n5                3  2.25   139
## 4 (2.3,2.9]; (4) of Qs0e1n5                4  2.70   134
## 5 (2.9,6.6]; (5) of Qs0e1n5                5  3.77   111
\end{verbatim}

\hypertarget{hgt0-2-groups-wgt0-2-groups-too}{%
\subparagraph{Hgt0 2 groups, Wgt0 2 groups
too}\label{hgt0-2-groups-wgt0-2-groups-too}}

\begin{Shaded}
\begin{Highlighting}[]
\CommentTok{# Joint Quantile Group Name}
\NormalTok{var.qjnt.grp.idx <-}\StringTok{ 'group.index'}
\NormalTok{list.cts2quantile <-}\StringTok{ }\KeywordTok{list}\NormalTok{(}\KeywordTok{list}\NormalTok{(}\DataTypeTok{vars=}\KeywordTok{c}\NormalTok{(}\StringTok{'hgt0'}\NormalTok{, }\StringTok{'wgt0'}\NormalTok{), }\DataTypeTok{prob=}\KeywordTok{c}\NormalTok{(}\DecValTok{0}\NormalTok{, }\FloatTok{.5}\NormalTok{, }\FloatTok{1.0}\NormalTok{)))}
\NormalTok{results <-}\StringTok{ }\KeywordTok{df_cut_by_sliced_quantiles_joint}\NormalTok{(}
\NormalTok{  df, var.qjnt.grp.idx, list.cts2quantile,}
  \DataTypeTok{vars.group_by =} \KeywordTok{c}\NormalTok{(}\StringTok{'indi.id'}\NormalTok{), }\DataTypeTok{vars.arrange =} \KeywordTok{c}\NormalTok{(}\StringTok{'indi.id'}\NormalTok{, }\StringTok{'svymthRound'}\NormalTok{),}
  \DataTypeTok{drop.any.quantile.na =} \OtherTok{TRUE}\NormalTok{, }\DataTypeTok{toprint =} \OtherTok{FALSE}\NormalTok{)}
\end{Highlighting}
\end{Shaded}

\begin{verbatim}
## Joining, by = "quant.perc"
\end{verbatim}

\begin{Shaded}
\begin{Highlighting}[]
\CommentTok{# Show Results}
\NormalTok{results}\OperatorTok{$}\NormalTok{df.group.slice1.cnt.mean}
\end{Highlighting}
\end{Shaded}

\begin{verbatim}
## # A tibble: 4 x 7
## # Groups:   hgt0_Qs0e1n2, wgt0_Qs0e1n2 [4]
##   hgt0_Qs0e1n2                wgt0_Qs0e1n2                        group.index hgt0_mean wgt0_mean hgt0_n wgt0_n
##   <fct>                       <fct>                                     <int>     <dbl>     <dbl>  <int>  <int>
## 1 [40.6,49.4]; (1) of Qs0e1n2 [1.4e+03,3.01e+03]; (1) of Qs0e1n2            1      47.4     2650.    652    652
## 2 [40.6,49.4]; (1) of Qs0e1n2 (3.01e+03,5.49e+03]; (2) of Qs0e1n2           2      48.5     3244.    228    228
## 3 (49.4,58]; (2) of Qs0e1n2   [1.4e+03,3.01e+03]; (1) of Qs0e1n2            3      50.4     2829.    202    202
## 4 (49.4,58]; (2) of Qs0e1n2   (3.01e+03,5.49e+03]; (2) of Qs0e1n2           4      51.3     3483.    626    626
\end{verbatim}

\hypertarget{hgt0-2-groups-wealth-2-groups-cebu-only}{%
\subparagraph{Hgt0 2 groups, Wealth 2 groups, Cebu
Only}\label{hgt0-2-groups-wealth-2-groups-cebu-only}}

\begin{Shaded}
\begin{Highlighting}[]
\CommentTok{# Joint Quantile Group Name}
\NormalTok{var.qjnt.grp.idx <-}\StringTok{ 'group.index'}
\NormalTok{list.cts2quantile <-}\StringTok{ }\KeywordTok{list}\NormalTok{(}
  \KeywordTok{list}\NormalTok{(}\DataTypeTok{vars=}\KeywordTok{c}\NormalTok{(}\StringTok{'wealthIdx'}\NormalTok{), }\DataTypeTok{prob=}\KeywordTok{c}\NormalTok{(}\DecValTok{0}\NormalTok{, }\FloatTok{.5}\NormalTok{, }\FloatTok{1.0}\NormalTok{)), }
  \KeywordTok{list}\NormalTok{(}\DataTypeTok{vars=}\KeywordTok{c}\NormalTok{(}\StringTok{'hgt0'}\NormalTok{), }\DataTypeTok{prob=}\KeywordTok{c}\NormalTok{(}\DecValTok{0}\NormalTok{, }\FloatTok{.333}\NormalTok{, }\FloatTok{0.666}\NormalTok{, }\FloatTok{1.0}\NormalTok{)))}
\NormalTok{results <-}\StringTok{ }\KeywordTok{df_cut_by_sliced_quantiles_joint}\NormalTok{(}
\NormalTok{  (df }\OperatorTok\StringTok{ }\KeywordTok{filter}\NormalTok{(S.country }\OperatorTok{==}\StringTok{ 'Cebu'}\NormalTok{)),}
\NormalTok{  var.qjnt.grp.idx, list.cts2quantile,}
  \DataTypeTok{vars.group_by =} \KeywordTok{c}\NormalTok{(}\StringTok{'indi.id'}\NormalTok{), }\DataTypeTok{vars.arrange =} \KeywordTok{c}\NormalTok{(}\StringTok{'indi.id'}\NormalTok{, }\StringTok{'svymthRound'}\NormalTok{),}
  \DataTypeTok{drop.any.quantile.na =} \OtherTok{TRUE}\NormalTok{, }\DataTypeTok{toprint =} \OtherTok{FALSE}\NormalTok{)}
\end{Highlighting}
\end{Shaded}

\begin{verbatim}
## Joining, by = c("S.country", "vil.id", "indi.id", "sex", "svymthRound", "momEdu", "wealthIdx", "hgt", "wgt", "hgt0", "wgt0", "prot", "cal", "p.A.prot",
## "p.A.nProt")
\end{verbatim}

\begin{Shaded}
\begin{Highlighting}[]
\CommentTok{# Show Results}
\NormalTok{results}\OperatorTok{$}\NormalTok{df.group.slice1.cnt.mean}
\end{Highlighting}
\end{Shaded}

\begin{verbatim}
## # A tibble: 6 x 7
## # Groups:   wealthIdx_Qs0e1n2, hgt0_Qs0e1n3 [6]
##   wealthIdx_Qs0e1n2          hgt0_Qs0e1n3                group.index wealthIdx_mean hgt0_mean wealthIdx_n hgt0_n
##   <fct>                      <fct>                             <int>          <dbl>     <dbl>       <int>  <int>
## 1 [5.2,8.3]; (1) of Qs0e1n2  [41.1,48.4]; (1) of Qs0e1n3           1           7.15      46.9         270    270
## 2 [5.2,8.3]; (1) of Qs0e1n2  (48.4,50.1]; (2) of Qs0e1n3           2           7.18      49.2         269    269
## 3 [5.2,8.3]; (1) of Qs0e1n2  (50.1,58]; (3) of Qs0e1n3             3           7.13      51.3         236    236
## 4 (8.3,19.3]; (2) of Qs0e1n2 [41.1,48.4]; (1) of Qs0e1n3           4          11.1       47.2         179    179
## 5 (8.3,19.3]; (2) of Qs0e1n2 (48.4,50.1]; (2) of Qs0e1n3           5          11.2       49.3         185    185
## 6 (8.3,19.3]; (2) of Qs0e1n2 (50.1,58]; (3) of Qs0e1n3             6          11.6       51.7         207    207
\end{verbatim}

\hypertarget{results-of-income-wgt0-hgt0-joint-gruops-in-cebu}{%
\subparagraph{Results of income + Wgt0 + Hgt0 joint Gruops in
Cebu}\label{results-of-income-wgt0-hgt0-joint-gruops-in-cebu}}

Weight at month 0 below and above median, height at month zero into
three terciles.

\begin{Shaded}
\begin{Highlighting}[]
\CommentTok{# Joint Quantile Group Name}
\NormalTok{var.qjnt.grp.idx <-}\StringTok{ 'wltHgt0Wgt0.index'}
\NormalTok{list.cts2quantile <-}\StringTok{ }\KeywordTok{list}\NormalTok{(}
  \KeywordTok{list}\NormalTok{(}\DataTypeTok{vars=}\KeywordTok{c}\NormalTok{(}\StringTok{'wealthIdx'}\NormalTok{), }\DataTypeTok{prob=}\KeywordTok{c}\NormalTok{(}\DecValTok{0}\NormalTok{, }\FloatTok{.5}\NormalTok{, }\FloatTok{1.0}\NormalTok{)), }
  \KeywordTok{list}\NormalTok{(}\DataTypeTok{vars=}\KeywordTok{c}\NormalTok{(}\StringTok{'hgt0'}\NormalTok{, }\StringTok{'wgt0'}\NormalTok{), }\DataTypeTok{prob=}\KeywordTok{c}\NormalTok{(}\DecValTok{0}\NormalTok{, }\FloatTok{.5}\NormalTok{, }\FloatTok{1.0}\NormalTok{)))}
\NormalTok{results <-}\StringTok{ }\KeywordTok{df_cut_by_sliced_quantiles_joint}\NormalTok{(}
\NormalTok{  (df }\OperatorTok\StringTok{ }\KeywordTok{filter}\NormalTok{(S.country }\OperatorTok{==}\StringTok{ 'Cebu'}\NormalTok{)),}
\NormalTok{  var.qjnt.grp.idx, list.cts2quantile,}
  \DataTypeTok{vars.group_by =} \KeywordTok{c}\NormalTok{(}\StringTok{'indi.id'}\NormalTok{), }\DataTypeTok{vars.arrange =} \KeywordTok{c}\NormalTok{(}\StringTok{'indi.id'}\NormalTok{, }\StringTok{'svymthRound'}\NormalTok{),}
  \DataTypeTok{drop.any.quantile.na =} \OtherTok{TRUE}\NormalTok{, }\DataTypeTok{toprint =} \OtherTok{FALSE}\NormalTok{)}
\end{Highlighting}
\end{Shaded}

\begin{verbatim}
## Joining, by = "quant.perc"Joining, by = c("S.country", "vil.id", "indi.id", "sex", "svymthRound", "momEdu", "wealthIdx", "hgt", "wgt", "hgt0", "wgt0",
## "prot", "cal", "p.A.prot", "p.A.nProt")
\end{verbatim}

\begin{Shaded}
\begin{Highlighting}[]
\CommentTok{# Show Results}
\NormalTok{results}\OperatorTok{$}\NormalTok{df.group.slice1.cnt.mean}
\end{Highlighting}
\end{Shaded}

\begin{verbatim}
## # A tibble: 8 x 10
## # Groups:   wealthIdx_Qs0e1n2, hgt0_Qs0e1n2, wgt0_Qs0e1n2 [8]
##   wealthIdx_Qs0e1n2       hgt0_Qs0e1n2            wgt0_Qs0e1n2                wltHgt0Wgt0.ind~ wealthIdx_mean hgt0_mean wgt0_mean wealthIdx_n hgt0_n wgt0_n
##   <fct>                   <fct>                   <fct>                                  <int>          <dbl>     <dbl>     <dbl>       <int>  <int>  <int>
## 1 [5.2,8.3]; (1) of Qs0e~ [41.1,49.2]; (1) of Qs~ [1.4e+03,2.98e+03]; (1) of~                1           7.16      47.3     2607.         308    308    308
## 2 [5.2,8.3]; (1) of Qs0e~ [41.1,49.2]; (1) of Qs~ (2.98e+03,5.49e+03]; (2) o~                2           7.27      48.4     3156.         102    102    102
## 3 [5.2,8.3]; (1) of Qs0e~ (49.2,58]; (2) of Qs0e~ [1.4e+03,2.98e+03]; (1) of~                3           7.00      50.2     2781.          97     97     97
## 4 [5.2,8.3]; (1) of Qs0e~ (49.2,58]; (2) of Qs0e~ (2.98e+03,5.49e+03]; (2) o~                4           7.16      51.0     3328.         268    268    268
## 5 (8.3,19.3]; (2) of Qs0~ [41.1,49.2]; (1) of Qs~ [1.4e+03,2.98e+03]; (1) of~                5          10.9       47.4     2632.         186    186    186
## 6 (8.3,19.3]; (2) of Qs0~ [41.1,49.2]; (1) of Qs~ (2.98e+03,5.49e+03]; (2) o~                6          11.3       48.5     3196.          81     81     81
## 7 (8.3,19.3]; (2) of Qs0~ (49.2,58]; (2) of Qs0e~ [1.4e+03,2.98e+03]; (1) of~                7          11.3       50.2     2779.          82     82     82
## 8 (8.3,19.3]; (2) of Qs0~ (49.2,58]; (2) of Qs0e~ (2.98e+03,5.49e+03]; (2) o~                8          11.7       51.4     3431.         222    222    222
\end{verbatim}

\hypertarget{line-by-linequantiles-var-by-var}{%
\paragraph{Line by Line--Quantiles Var by
Var}\label{line-by-linequantiles-var-by-var}}

The idea of the function is to generate quantiles levels first, and then
use those to generate the categories based on quantiles. Rather than
doing this in one step. These are done in two steps, to increase clarity
in the quantiles used for quantile category generation. And a dataframe
with these quantiles are saved as a separate output of the function.

\hypertarget{dataframe-of-variables-group-by-level-quantiles}{%
\subparagraph{Dataframe of Variables' Group-by Level
Quantiles}\label{dataframe-of-variables-group-by-level-quantiles}}

Quantiles from Different Variables. Note that these variables are
specific to the individual, not individual/month. So we need to first
slick the data, so that we only get the first rows.

Do this in several steps to clarify group\_by level. No speed loss.

\begin{Shaded}
\begin{Highlighting}[]
\CommentTok{# Selected Variables, many Percentiles}
\NormalTok{vars.group_by <-}\StringTok{ }\KeywordTok{c}\NormalTok{(}\StringTok{'indi.id'}\NormalTok{)}
\NormalTok{vars.arrange <-}\StringTok{ }\KeywordTok{c}\NormalTok{(}\StringTok{'indi.id'}\NormalTok{, }\StringTok{'svymthRound'}\NormalTok{)}
\NormalTok{vars.cts2quantile <-}\StringTok{ }\KeywordTok{c}\NormalTok{(}\StringTok{'wealthIdx'}\NormalTok{, }\StringTok{'hgt0'}\NormalTok{, }\StringTok{'wgt0'}\NormalTok{)}
\NormalTok{seq.quantiles <-}\StringTok{ }\KeywordTok{c}\NormalTok{(}\DecValTok{0}\NormalTok{, }\FloatTok{0.3333}\NormalTok{, }\FloatTok{0.6666}\NormalTok{, }\FloatTok{1.0}\NormalTok{)}
\NormalTok{df.sliced <-}\StringTok{ }\KeywordTok{df_sliced_quantiles}\NormalTok{(}
\NormalTok{  df, vars.cts2quantile, seq.quantiles, vars.group_by, vars.arrange)}
\end{Highlighting}
\end{Shaded}

\begin{verbatim}
## Joining, by = "quant.perc"Joining, by = "quant.perc"
\end{verbatim}

\begin{Shaded}
\begin{Highlighting}[]
\NormalTok{df.sliced.quantiles <-}\StringTok{ }\NormalTok{df.sliced}\OperatorTok{$}\NormalTok{df.sliced.quantiles}
\NormalTok{df.grp.L1 <-}\StringTok{ }\NormalTok{df.sliced}\OperatorTok{$}\NormalTok{df.grp.L1}
\end{Highlighting}
\end{Shaded}

\begin{Shaded}
\begin{Highlighting}[]
\NormalTok{df.sliced.quantiles}
\end{Highlighting}
\end{Shaded}

\begin{verbatim}
## # A tibble: 4 x 4
##   quant.perc wealthIdx  hgt0  wgt0
##   <chr>          <dbl> <dbl> <dbl>
## 1 0%               1    40.6 1402.
## 2 33.33%           5.2  48.5 2843.
## 3 66.66%           8.3  50.2 3209.
## 4 100%            19.3  58   5494.
\end{verbatim}

\begin{Shaded}
\begin{Highlighting}[]
\CommentTok{# Quantiles all Variables}
\KeywordTok{suppressMessages}\NormalTok{(}\KeywordTok{lapply}\NormalTok{(}
  \KeywordTok{names}\NormalTok{(df), gen_quantiles, }\DataTypeTok{df=}\NormalTok{df.grp.L1, }
  \DataTypeTok{prob=}\KeywordTok{seq}\NormalTok{(}\FloatTok{0.1}\NormalTok{,}\FloatTok{0.9}\NormalTok{,}\FloatTok{0.10}\NormalTok{)) }\OperatorTok\StringTok{ }\KeywordTok{reduce}\NormalTok{(full_join)) }\OperatorTok\StringTok{ }
\StringTok{  }\KeywordTok{kable}\NormalTok{() }\OperatorTok\StringTok{ }\KeywordTok{kable_styling_fc_wide}\NormalTok{()}
\end{Highlighting}
\end{Shaded}

\begin{verbatim}
## Warning in quantile(as.numeric(df[[var]]), prob, na.rm = TRUE): NAs introduced by coercion

## Warning in quantile(as.numeric(df[[var]]), prob, na.rm = TRUE): NAs introduced by coercion
\end{verbatim}

\begin{table}[!h]
\centering
\resizebox{\linewidth}{!}{
\begin{tabular}{l|r|r|r|r|r|r|r|r|r|r|r|r|r|r|r}
\hline
quant.perc & S.country & vil.id & indi.id & sex & svymthRound & momEdu & wealthIdx & hgt & wgt & hgt0 & wgt0 & prot & cal & p.A.prot & p.A.nProt\\
\hline
\rowcolor{gray!6}  10\% & NA & 3 & 203.2 & NA & 0 & 5.70 & 1.7 & 46.3 & 1396.94 & 46.60 & 2500.28 & 0.5 & 0.50 & 24.28 & 0.50\\
\hline
20\% & NA & 4 & 405.4 & NA & 0 & 6.90 & 2.3 & 47.3 & 1839.64 & 47.70 & 2686.28 & 0.5 & 0.50 & 172.30 & 0.50\\
\hline
\rowcolor{gray!6}  30\% & NA & 6 & 607.6 & NA & 0 & 7.70 & 3.3 & 48.0 & 2271.69 & 48.30 & 2803.89 & 0.5 & 0.50 & 721.08 & 1.06\\
\hline
40\% & NA & 8 & 809.8 & NA & 0 & 8.60 & 6.3 & 48.7 & 2669.16 & 48.80 & 2909.68 & 0.5 & 0.50 & 1009.88 & 19.00\\
\hline
\rowcolor{gray!6}  50\% & NA & 9 & 1012.0 & NA & 0 & 9.30 & 7.3 & 49.4 & 3050.10 & 49.40 & 3013.00 & 0.5 & 0.50 & 1273.30 & 110.95\\
\hline
60\% & NA & 13 & 1214.2 & NA & 0 & 10.40 & 8.3 & 49.9 & 3439.50 & 49.90 & 3126.08 & 0.5 & 3.88 & 1614.40 & 221.92\\
\hline
\rowcolor{gray!6}  70\% & NA & 14 & 1416.4 & NA & 0 & 11.36 & 8.3 & 50.5 & 3857.28 & 50.40 & 3249.52 & 0.7 & 8.26 & 2679.54 & 256.80\\
\hline
80\% & NA & 17 & 1618.6 & NA & 0 & 12.70 & 9.3 & 51.2 & 4258.12 & 51.04 & 3417.86 & 1.2 & 11.50 & 4761.14 & 298.12\\
\hline
\rowcolor{gray!6}  90\% & NA & 26 & 1820.8 & NA & 0 & 14.60 & 11.3 & 52.3 & 4703.62 & 52.00 & 3682.83 & 1.6 & 15.60 & 10867.72 & 365.46\\
\hline
\end{tabular}}
\end{table}

\hypertarget{cut-quantile-categorical-variables}{%
\subparagraph{Cut Quantile Categorical
Variables}\label{cut-quantile-categorical-variables}}

Using the Quantiles we have generate, cut the continuous variables to
generate categorical quantile variables in the full dataframe.

Note that we can only cut based on unique breaks, but sometimes quantile
break-points are the same if some values are often observed, and also if
there are too few observations with respect to quantile groups.

To resolve this issue, we only look at unique quantiles.

We need several support Functions: 1. support functions to generate
suffix for quantile variables based on quantile cuts 2. support for
labeling variables of resulting quantiles beyond bracketing

\begin{Shaded}
\begin{Highlighting}[]
\CommentTok{# Function Testing}
\NormalTok{arr.quantiles <-}\StringTok{ }\NormalTok{df.sliced.quantiles[[}\KeywordTok{substitute}\NormalTok{(}\StringTok{'wealthIdx'}\NormalTok{)]]}
\NormalTok{arr.quantiles}
\end{Highlighting}
\end{Shaded}

\begin{verbatim}
## [1]  1.0  5.2  8.3 19.3
\end{verbatim}

\begin{Shaded}
\begin{Highlighting}[]
\NormalTok{arr.sort.unique.quantiles <-}\StringTok{ }
\StringTok{  }\KeywordTok{sort}\NormalTok{(}\KeywordTok{unique}\NormalTok{(df.sliced.quantiles[[}\KeywordTok{substitute}\NormalTok{(}\StringTok{'wealthIdx'}\NormalTok{)]]))}
\NormalTok{arr.sort.unique.quantiles}
\end{Highlighting}
\end{Shaded}

\begin{verbatim}
## [1]  1.0  5.2  8.3 19.3
\end{verbatim}

\begin{Shaded}
\begin{Highlighting}[]
\KeywordTok{f_Q_label}\NormalTok{(arr.quantiles, arr.sort.unique.quantiles[}\DecValTok{1}\NormalTok{], seq.quantiles)}
\end{Highlighting}
\end{Shaded}

\begin{verbatim}
## [1] "(1) of Qs0e1n3"
\end{verbatim}

\begin{Shaded}
\begin{Highlighting}[]
\KeywordTok{f_Q_label}\NormalTok{(arr.quantiles, arr.sort.unique.quantiles[}\DecValTok{2}\NormalTok{], seq.quantiles)}
\end{Highlighting}
\end{Shaded}

\begin{verbatim}
## [1] "(2) of Qs0e1n3"
\end{verbatim}

\begin{Shaded}
\begin{Highlighting}[]
\KeywordTok{lapply}\NormalTok{(arr.sort.unique.quantiles[}\DecValTok{1}\OperatorTok{:}\NormalTok{(}\KeywordTok{length}\NormalTok{(arr.sort.unique.quantiles)}\OperatorTok{-}\DecValTok{1}\NormalTok{)],}
\NormalTok{       f_Q_label,}
       \DataTypeTok{arr.quantiles=}\NormalTok{arr.quantiles,}
       \DataTypeTok{seq.quantiles=}\NormalTok{seq.quantiles)}
\end{Highlighting}
\end{Shaded}

\begin{verbatim}
## [[1]]
## [1] "(1) of Qs0e1n3"
## 
## [[2]]
## [1] "(2) of Qs0e1n3"
## 
## [[3]]
## [1] "(3) of Qs0e1n3"
\end{verbatim}

\begin{Shaded}
\begin{Highlighting}[]
\CommentTok{# Generate Categorical Variables of Quantiles}
\NormalTok{vars.group_by <-}\StringTok{ }\KeywordTok{c}\NormalTok{(}\StringTok{'indi.id'}\NormalTok{)}
\NormalTok{vars.arrange <-}\StringTok{ }\KeywordTok{c}\NormalTok{(}\StringTok{'indi.id'}\NormalTok{, }\StringTok{'svymthRound'}\NormalTok{)}
\NormalTok{vars.cts2quantile <-}\StringTok{ }\KeywordTok{c}\NormalTok{(}\StringTok{'wealthIdx'}\NormalTok{, }\StringTok{'hgt0'}\NormalTok{, }\StringTok{'wgt0'}\NormalTok{)}
\NormalTok{seq.quantiles <-}\StringTok{ }\KeywordTok{c}\NormalTok{(}\DecValTok{0}\NormalTok{, }\FloatTok{0.3333}\NormalTok{, }\FloatTok{0.6666}\NormalTok{, }\FloatTok{1.0}\NormalTok{)}
\NormalTok{df.cut <-}\StringTok{ }\KeywordTok{df_cut_by_sliced_quantiles}\NormalTok{(}
\NormalTok{  df, vars.cts2quantile, seq.quantiles, vars.group_by, vars.arrange)}
\end{Highlighting}
\end{Shaded}

\begin{verbatim}
## Joining, by = "quant.perc"Joining, by = "quant.perc"
\end{verbatim}

\begin{Shaded}
\begin{Highlighting}[]
\NormalTok{vars.quantile.cut <-}\StringTok{ }\NormalTok{df.cut}\OperatorTok{$}\NormalTok{vars.quantile.cut}
\NormalTok{df.with.cut.quant <-}\StringTok{ }\NormalTok{df.cut}\OperatorTok{$}\NormalTok{df.with.cut.quant}
\NormalTok{df.grp.L1 <-}\StringTok{ }\NormalTok{df.cut}\OperatorTok{$}\NormalTok{df.grp.L1}
\end{Highlighting}
\end{Shaded}

\begin{Shaded}
\begin{Highlighting}[]
\CommentTok{# Cut Variables Generated}
\KeywordTok{names}\NormalTok{(vars.quantile.cut)}
\end{Highlighting}
\end{Shaded}

\begin{verbatim}
## [1] "wealthIdx_Qs0e1n3" "hgt0_Qs0e1n3"      "wgt0_Qs0e1n3"
\end{verbatim}

\begin{Shaded}
\begin{Highlighting}[]
\KeywordTok{summary}\NormalTok{(vars.quantile.cut)}
\end{Highlighting}
\end{Shaded}

\begin{verbatim}
##                   wealthIdx_Qs0e1n3                      hgt0_Qs0e1n3                                wgt0_Qs0e1n3  
##  [1,5.2]; (1) of Qs0e1n3   :10958   [40.6,48.5]; (1) of Qs0e1n3:10232   [1.4e+03,2.84e+03]; (1) of Qs0e1n3 :10105  
##  (5.2,8.3]; (2) of Qs0e1n3 :13812   (48.5,50.2]; (2) of Qs0e1n3: 9895   (2.84e+03,3.21e+03]; (2) of Qs0e1n3:10056  
##  (8.3,19.3]; (3) of Qs0e1n3:10295   (50.2,58]; (3) of Qs0e1n3  : 9908   (3.21e+03,5.49e+03]; (3) of Qs0e1n3: 9858  
##                                     NA's                       : 5030   NA's                               : 5046
\end{verbatim}

\begin{Shaded}
\begin{Highlighting}[]
\CommentTok{# options(repr.matrix.max.rows=50, repr.matrix.max.cols=20)}
\CommentTok{# df.with.cut.quant}
\end{Highlighting}
\end{Shaded}

\hypertarget{individual-variables-quantile-cuts-review-results}{%
\subparagraph{Individual Variables' Quantile Cuts Review
Results}\label{individual-variables-quantile-cuts-review-results}}

\begin{Shaded}
\begin{Highlighting}[]
\CommentTok{# Group By Results}
\NormalTok{f.count <-}\StringTok{ }\ControlFlowTok{function}\NormalTok{(df, var.cts, seq.quantiles) \{}
\NormalTok{    df }\OperatorTok\StringTok{ }\KeywordTok{select}\NormalTok{(S.country, indi.id, }
\NormalTok{                  svymthRound, }\KeywordTok{matches}\NormalTok{(}\KeywordTok{paste0}\NormalTok{(var.cts, }\DataTypeTok{collapse=}\StringTok{'|'}\NormalTok{))) }\OperatorTok
\StringTok{        }\KeywordTok{group_by}\NormalTok{(}\OperatorTok{!!}\KeywordTok{sym}\NormalTok{(}\KeywordTok{f_var_rename}\NormalTok{(}\KeywordTok{paste0}\NormalTok{(var.cts,}\StringTok{'_q'}\NormalTok{), seq.quantiles))) }\OperatorTok
\StringTok{        }\KeywordTok{summarise_all}\NormalTok{(}\KeywordTok{funs}\NormalTok{(}\DataTypeTok{n=}\KeywordTok{n}\NormalTok{()))}
\NormalTok{\}}
\end{Highlighting}
\end{Shaded}

\begin{Shaded}
\begin{Highlighting}[]
\CommentTok{# Full Panel Results}
\KeywordTok{lapply}\NormalTok{(vars.cts2quantile, f.count, }
       \DataTypeTok{df=}\NormalTok{df.with.cut.quant, }\DataTypeTok{seq.quantiles=}\NormalTok{seq.quantiles)}
\end{Highlighting}
\end{Shaded}

\begin{verbatim}
## Warning: Factor `hgt0_Qs0e1n3` contains implicit NA, consider using `forcats::fct_explicit_na`
\end{verbatim}

\begin{verbatim}
## Warning: Factor `wgt0_Qs0e1n3` contains implicit NA, consider using `forcats::fct_explicit_na`
\end{verbatim}

\begin{verbatim}
## [[1]]
## # A tibble: 3 x 5
##   wealthIdx_Qs0e1n3          S.country_n indi.id_n svymthRound_n wealthIdx_n
##   <fct>                            <int>     <int>         <int>       <int>
## 1 [1,5.2]; (1) of Qs0e1n3          10958     10958         10958       10958
## 2 (5.2,8.3]; (2) of Qs0e1n3        13812     13812         13812       13812
## 3 (8.3,19.3]; (3) of Qs0e1n3       10295     10295         10295       10295
## 
## [[2]]
## # A tibble: 4 x 5
##   hgt0_Qs0e1n3                S.country_n indi.id_n svymthRound_n hgt0_n
##   <fct>                             <int>     <int>         <int>  <int>
## 1 [40.6,48.5]; (1) of Qs0e1n3       10232     10232         10232  10232
## 2 (48.5,50.2]; (2) of Qs0e1n3        9895      9895          9895   9895
## 3 (50.2,58]; (3) of Qs0e1n3          9908      9908          9908   9908
## 4 <NA>                               5030      5030          5030   5030
## 
## [[3]]
## # A tibble: 4 x 5
##   wgt0_Qs0e1n3                        S.country_n indi.id_n svymthRound_n wgt0_n
##   <fct>                                     <int>     <int>         <int>  <int>
## 1 [1.4e+03,2.84e+03]; (1) of Qs0e1n3        10105     10105         10105  10105
## 2 (2.84e+03,3.21e+03]; (2) of Qs0e1n3       10056     10056         10056  10056
## 3 (3.21e+03,5.49e+03]; (3) of Qs0e1n3        9858      9858          9858   9858
## 4 <NA>                                       5046      5046          5046   5046
\end{verbatim}

\begin{Shaded}
\begin{Highlighting}[]
\CommentTok{# Results Individual Slice}
\KeywordTok{lapply}\NormalTok{(vars.cts2quantile, f.count,}
       \DataTypeTok{df=}\NormalTok{(df.with.cut.quant }\OperatorTok\StringTok{ }
\StringTok{             }\KeywordTok{group_by}\NormalTok{(}\OperatorTok{!!!}\KeywordTok{syms}\NormalTok{(vars.group_by)) }\OperatorTok\StringTok{ }
\StringTok{             }\KeywordTok{arrange}\NormalTok{(}\OperatorTok{!!!}\KeywordTok{syms}\NormalTok{(vars.arrange)) }\OperatorTok\StringTok{ }\KeywordTok{slice}\NormalTok{(1L)),}
       \DataTypeTok{seq.quantiles =}\NormalTok{ seq.quantiles)}
\end{Highlighting}
\end{Shaded}

\begin{verbatim}
## Warning: Factor `hgt0_Qs0e1n3` contains implicit NA, consider using `forcats::fct_explicit_na`
\end{verbatim}

\begin{verbatim}
## Warning: Factor `wgt0_Qs0e1n3` contains implicit NA, consider using `forcats::fct_explicit_na`
\end{verbatim}

\begin{verbatim}
## [[1]]
## # A tibble: 3 x 5
##   wealthIdx_Qs0e1n3          S.country_n indi.id_n svymthRound_n wealthIdx_n
##   <fct>                            <int>     <int>         <int>       <int>
## 1 [1,5.2]; (1) of Qs0e1n3            683       683           683         683
## 2 (5.2,8.3]; (2) of Qs0e1n3          768       768           768         768
## 3 (8.3,19.3]; (3) of Qs0e1n3         572       572           572         572
## 
## [[2]]
## # A tibble: 4 x 5
##   hgt0_Qs0e1n3                S.country_n indi.id_n svymthRound_n hgt0_n
##   <fct>                             <int>     <int>         <int>  <int>
## 1 [40.6,48.5]; (1) of Qs0e1n3         580       580           580    580
## 2 (48.5,50.2]; (2) of Qs0e1n3         561       561           561    561
## 3 (50.2,58]; (3) of Qs0e1n3           568       568           568    568
## 4 <NA>                                314       314           314    314
## 
## [[3]]
## # A tibble: 4 x 5
##   wgt0_Qs0e1n3                        S.country_n indi.id_n svymthRound_n wgt0_n
##   <fct>                                     <int>     <int>         <int>  <int>
## 1 [1.4e+03,2.84e+03]; (1) of Qs0e1n3          569       569           569    569
## 2 (2.84e+03,3.21e+03]; (2) of Qs0e1n3         569       569           569    569
## 3 (3.21e+03,5.49e+03]; (3) of Qs0e1n3         570       570           570    570
## 4 <NA>                                        315       315           315    315
\end{verbatim}

\hypertarget{differential-quantiles-for-different-variables-then-combine-to-form-new-groups}{%
\paragraph{Differential Quantiles for Different Variables Then Combine
to Form New
Groups}\label{differential-quantiles-for-different-variables-then-combine-to-form-new-groups}}

Collect together different quantile base variables and their percentile
cuttings quantile rules. Input Parameters.

\begin{Shaded}
\begin{Highlighting}[]
\CommentTok{# Generate Categorical Variables of Quantiles}
\NormalTok{vars.group_by <-}\StringTok{ }\KeywordTok{c}\NormalTok{(}\StringTok{'indi.id'}\NormalTok{)}
\NormalTok{vars.arrange <-}\StringTok{ }\KeywordTok{c}\NormalTok{(}\StringTok{'indi.id'}\NormalTok{, }\StringTok{'svymthRound'}\NormalTok{)}
\end{Highlighting}
\end{Shaded}

\begin{Shaded}
\begin{Highlighting}[]
\CommentTok{# Quantile Variables and Quantiles}
\NormalTok{vars.cts2quantile.wealth <-}\StringTok{ }\KeywordTok{c}\NormalTok{(}\StringTok{'wealthIdx'}\NormalTok{)}
\NormalTok{seq.quantiles.wealth <-}\StringTok{ }\KeywordTok{c}\NormalTok{(}\DecValTok{0}\NormalTok{, }\FloatTok{.5}\NormalTok{, }\FloatTok{1.0}\NormalTok{)}
\NormalTok{vars.cts2quantile.wgthgt <-}\StringTok{ }\KeywordTok{c}\NormalTok{(}\StringTok{'hgt0'}\NormalTok{, }\StringTok{'wgt0'}\NormalTok{)}
\NormalTok{seq.quantiles.wgthgt <-}\StringTok{ }\KeywordTok{c}\NormalTok{(}\DecValTok{0}\NormalTok{, }\FloatTok{.3333}\NormalTok{, }\FloatTok{0.6666}\NormalTok{, }\FloatTok{1.0}\NormalTok{)}
\NormalTok{drop.any.quantile.na <-}\StringTok{ }\OtherTok{TRUE}
\CommentTok{# collect to list}
\NormalTok{list.cts2quantile <-}\StringTok{ }\KeywordTok{list}\NormalTok{(}\KeywordTok{list}\NormalTok{(}\DataTypeTok{vars=}\NormalTok{vars.cts2quantile.wealth,}
                               \DataTypeTok{prob=}\NormalTok{seq.quantiles.wealth),}
                          \KeywordTok{list}\NormalTok{(}\DataTypeTok{vars=}\NormalTok{vars.cts2quantile.wgthgt,}
                               \DataTypeTok{prob=}\NormalTok{seq.quantiles.wgthgt))}
\end{Highlighting}
\end{Shaded}

\hypertarget{check-if-within-group-variables-are-the-same}{%
\paragraph{Check if Within Group Variables Are The
Same}\label{check-if-within-group-variables-are-the-same}}

Need to make sure quantile variables are unique within groups

\begin{Shaded}
\begin{Highlighting}[]
\NormalTok{vars.cts2quantile <-}\StringTok{ }\KeywordTok{unlist}\NormalTok{(}\KeywordTok{lapply}\NormalTok{(list.cts2quantile, }\ControlFlowTok{function}\NormalTok{(elist) elist}\OperatorTok{$}\NormalTok{vars))}
\KeywordTok{f_check_distinct_ingroup}\NormalTok{(df, vars.group_by, }\DataTypeTok{vars.values_in_group=}\NormalTok{vars.cts2quantile)}
\end{Highlighting}
\end{Shaded}

\hypertarget{keep-only-non-na-for-all-quantile-variables}{%
\subparagraph{Keep only non-NA for all Quantile
Variables}\label{keep-only-non-na-for-all-quantile-variables}}

\begin{Shaded}
\begin{Highlighting}[]
\CommentTok{# Original dimensions}
\KeywordTok{dim}\NormalTok{(df)}
\end{Highlighting}
\end{Shaded}

\begin{verbatim}
## [1] 35065    15
\end{verbatim}

\begin{Shaded}
\begin{Highlighting}[]
\CommentTok{# All Continuous Variables from lists}
\NormalTok{vars.cts2quantile <-}\StringTok{ }\KeywordTok{unlist}\NormalTok{(}\KeywordTok{lapply}\NormalTok{(list.cts2quantile, }\ControlFlowTok{function}\NormalTok{(elist) elist}\OperatorTok{$}\NormalTok{vars))}
\NormalTok{vars.cts2quantile}
\end{Highlighting}
\end{Shaded}

\begin{verbatim}
## [1] "wealthIdx" "hgt0"      "wgt0"
\end{verbatim}

\begin{Shaded}
\begin{Highlighting}[]
\CommentTok{# Keep only if not NA for all Quantile variables}
\ControlFlowTok{if}\NormalTok{ (drop.any.quantile.na) \{}
\NormalTok{    df.select <-}\StringTok{ }\NormalTok{df }\OperatorTok\StringTok{ }\KeywordTok{drop_na}\NormalTok{(}\KeywordTok{c}\NormalTok{(vars.group_by, vars.arrange, vars.cts2quantile))}
\NormalTok{\}}
\KeywordTok{dim}\NormalTok{(df.select)}
\end{Highlighting}
\end{Shaded}

\begin{verbatim}
## [1] 30019    15
\end{verbatim}

\hypertarget{apply-quantiles-for-each-quantile-variable}{%
\subparagraph{Apply Quantiles for Each Quantile
Variable}\label{apply-quantiles-for-each-quantile-variable}}

\begin{Shaded}
\begin{Highlighting}[]
\CommentTok{# Dealing with a list of quantile variables}
\NormalTok{df.cut.wealth <-}\StringTok{ }\KeywordTok{df_cut_by_sliced_quantiles}\NormalTok{(}
\NormalTok{  df.select, vars.cts2quantile.wealth, seq.quantiles.wealth, vars.group_by, vars.arrange)}
\KeywordTok{summary}\NormalTok{(df.cut.wealth}\OperatorTok{$}\NormalTok{vars.quantile.cut)}
\end{Highlighting}
\end{Shaded}

\begin{verbatim}
##                   wealthIdx_Qs0e1n2
##  [1,7.3]; (1) of Qs0e1n2   :14936  
##  (7.3,19.3]; (2) of Qs0e1n2:15083
\end{verbatim}

\begin{Shaded}
\begin{Highlighting}[]
\CommentTok{# summary((df.cut.wealth$df.with.cut.quant)[['wealthIdx_Qs0e1n2']])}
\CommentTok{# df.cut.wealth$df.with.cut.quant %>% filter(is.na(wealthIdx_Qs0e1n2))}
\CommentTok{# df.cut.wealth$df.with.cut.quant %>% filter(indi.id == 500)}
\end{Highlighting}
\end{Shaded}

\begin{Shaded}
\begin{Highlighting}[]
\NormalTok{df.cut.wgthgt <-}\StringTok{ }\KeywordTok{df_cut_by_sliced_quantiles}\NormalTok{(}
\NormalTok{  df.select, vars.cts2quantile.wgthgt, seq.quantiles.wgthgt, vars.group_by, vars.arrange)}
\end{Highlighting}
\end{Shaded}

\begin{verbatim}
## Joining, by = "quant.perc"
\end{verbatim}

\begin{Shaded}
\begin{Highlighting}[]
\KeywordTok{summary}\NormalTok{(df.cut.wgthgt}\OperatorTok{$}\NormalTok{vars.quantile.cut)}
\end{Highlighting}
\end{Shaded}

\begin{verbatim}
##                       hgt0_Qs0e1n3                                wgt0_Qs0e1n3  
##  [40.6,48.5]; (1) of Qs0e1n3:10216   [1.4e+03,2.84e+03]; (1) of Qs0e1n3 :10105  
##  (48.5,50.2]; (2) of Qs0e1n3: 9895   (2.84e+03,3.21e+03]; (2) of Qs0e1n3:10056  
##  (50.2,58]; (3) of Qs0e1n3  : 9908   (3.21e+03,5.49e+03]; (3) of Qs0e1n3: 9858
\end{verbatim}

\hypertarget{apply-quantiles-functionally}{%
\subparagraph{Apply Quantiles
Functionally}\label{apply-quantiles-functionally}}

\begin{Shaded}
\begin{Highlighting}[]
\CommentTok{# Function to handle list inputs with different quantiles vars and probabilities}
\NormalTok{df_cut_by_sliced_quantiles_grps <-}\StringTok{ }
\StringTok{  }\ControlFlowTok{function}\NormalTok{(quantile.grp.list, df, vars.group_by, vars.arrange) \{}
\NormalTok{    vars.cts2quantile <-}\StringTok{ }\NormalTok{quantile.grp.list}\OperatorTok{$}\NormalTok{vars}
\NormalTok{    seq.quantiles <-}\StringTok{ }\NormalTok{quantile.grp.list}\OperatorTok{$}\NormalTok{prob}
    \KeywordTok{return}\NormalTok{(}\KeywordTok{df_cut_by_sliced_quantiles}\NormalTok{(}
\NormalTok{      df, vars.cts2quantile, seq.quantiles, vars.group_by, vars.arrange))}
\NormalTok{\}}
\end{Highlighting}
\end{Shaded}

\begin{Shaded}
\begin{Highlighting}[]
\CommentTok{# Apply function}
\NormalTok{df.cut.list <-}\StringTok{ }\KeywordTok{lapply}\NormalTok{(}
\NormalTok{  list.cts2quantile, df_cut_by_sliced_quantiles_grps,}
  \DataTypeTok{df=}\NormalTok{df.select, }\DataTypeTok{vars.group_by=}\NormalTok{vars.group_by, }\DataTypeTok{vars.arrange=}\NormalTok{vars.arrange)}
\end{Highlighting}
\end{Shaded}

\begin{verbatim}
## Joining, by = "quant.perc"
\end{verbatim}

\begin{Shaded}
\begin{Highlighting}[]
\CommentTok{# Reduce Resulting Matrixes Together}
\NormalTok{df.with.cut.quant.all <-}\StringTok{ }\KeywordTok{lapply}\NormalTok{(}
\NormalTok{  df.cut.list, }\ControlFlowTok{function}\NormalTok{(elist) elist}\OperatorTok{$}\NormalTok{df.with.cut.quant) }\OperatorTok\StringTok{ }\KeywordTok{reduce}\NormalTok{(left_join)}
\end{Highlighting}
\end{Shaded}

\begin{verbatim}
## Joining, by = c("S.country", "vil.id", "indi.id", "sex", "svymthRound", "momEdu", "wealthIdx", "hgt", "wgt", "hgt0", "wgt0", "prot", "cal", "p.A.prot",
## "p.A.nProt")
\end{verbatim}

\begin{Shaded}
\begin{Highlighting}[]
\KeywordTok{dim}\NormalTok{(df.with.cut.quant.all)}
\end{Highlighting}
\end{Shaded}

\begin{verbatim}
## [1] 30019    18
\end{verbatim}

\begin{Shaded}
\begin{Highlighting}[]
\CommentTok{# Obrain Newly Created Quantile Group Variables}
\NormalTok{vars.quantile.cut.all <-}\StringTok{ }\KeywordTok{unlist}\NormalTok{(}
  \KeywordTok{lapply}\NormalTok{(df.cut.list, }\ControlFlowTok{function}\NormalTok{(elist) }\KeywordTok{names}\NormalTok{(elist}\OperatorTok{$}\NormalTok{vars.quantile.cut)))}
\NormalTok{vars.quantile.cut.all}
\end{Highlighting}
\end{Shaded}

\begin{verbatim}
## [1] "wealthIdx_Qs0e1n2" "hgt0_Qs0e1n3"      "wgt0_Qs0e1n3"
\end{verbatim}

\hypertarget{summarize-by-groups}{%
\subparagraph{Summarize by Groups}\label{summarize-by-groups}}

Summarize by all groups.

\begin{Shaded}
\begin{Highlighting}[]
\KeywordTok{summary}\NormalTok{(df.with.cut.quant.all }\OperatorTok\StringTok{ }\KeywordTok{select}\NormalTok{(}\KeywordTok{one_of}\NormalTok{(vars.quantile.cut.all)))}
\end{Highlighting}
\end{Shaded}

\begin{verbatim}
##                   wealthIdx_Qs0e1n2                      hgt0_Qs0e1n3                                wgt0_Qs0e1n3  
##  [1,7.3]; (1) of Qs0e1n2   :14936   [40.6,48.5]; (1) of Qs0e1n3:10216   [1.4e+03,2.84e+03]; (1) of Qs0e1n3 :10105  
##  (7.3,19.3]; (2) of Qs0e1n2:15083   (48.5,50.2]; (2) of Qs0e1n3: 9895   (2.84e+03,3.21e+03]; (2) of Qs0e1n3:10056  
##                                     (50.2,58]; (3) of Qs0e1n3  : 9908   (3.21e+03,5.49e+03]; (3) of Qs0e1n3: 9858
\end{verbatim}

\begin{Shaded}
\begin{Highlighting}[]
\CommentTok{# df.with.cut.quant.all %>%}
\CommentTok{#     group_by(!!!syms(vars.quantile.cut.all)) %>%}
\CommentTok{#     summarise_at(vars.cts2quantile, funs(mean, n()))}
\end{Highlighting}
\end{Shaded}

\hypertarget{generate-joint-quantile-vars-unique-groups}{%
\subparagraph{Generate Joint Quantile Vars Unique
Groups}\label{generate-joint-quantile-vars-unique-groups}}

\begin{Shaded}
\begin{Highlighting}[]
\CommentTok{# Generate Joint Quantile Index Variable}
\NormalTok{var.qjnt.grp.idx <-}\StringTok{ 'group.index'}
\NormalTok{df.with.cut.quant.all <-}\StringTok{ }\NormalTok{df.with.cut.quant.all }\OperatorTok\StringTok{ }
\StringTok{  }\KeywordTok{mutate}\NormalTok{(}\OperatorTok{!!}\DataTypeTok{var.qjnt.grp.idx :=} \KeywordTok{group_indices}\NormalTok{(., }\OperatorTok{!!!}\KeywordTok{syms}\NormalTok{(vars.quantile.cut.all)))}
\end{Highlighting}
\end{Shaded}

\begin{Shaded}
\begin{Highlighting}[]
\NormalTok{arr.group.idx <-}\StringTok{ }\KeywordTok{t}\NormalTok{(}\KeywordTok{sort}\NormalTok{(}\KeywordTok{unique}\NormalTok{(df.with.cut.quant.all[[var.qjnt.grp.idx]])))}
\NormalTok{arr.group.idx}
\end{Highlighting}
\end{Shaded}

\begin{verbatim}
##      [,1] [,2] [,3] [,4] [,5] [,6] [,7] [,8] [,9] [,10] [,11] [,12] [,13] [,14] [,15] [,16] [,17] [,18]
## [1,]    1    2    3    4    5    6    7    8    9    10    11    12    13    14    15    16    17    18
\end{verbatim}

\begin{Shaded}
\begin{Highlighting}[]
\KeywordTok{head}\NormalTok{(df.with.cut.quant.all }\OperatorTok\StringTok{ }
\StringTok{  }\KeywordTok{group_by}\NormalTok{(}\OperatorTok{!!!}\KeywordTok{syms}\NormalTok{(vars.quantile.cut.all), }\OperatorTok{!!}\KeywordTok{sym}\NormalTok{(var.qjnt.grp.idx)) }\OperatorTok
\StringTok{  }\KeywordTok{summarise_at}\NormalTok{(vars.cts2quantile, }\KeywordTok{funs}\NormalTok{(mean, }\KeywordTok{n}\NormalTok{())), }\DecValTok{10}\NormalTok{) }\OperatorTok\StringTok{ }
\StringTok{  }\KeywordTok{kable}\NormalTok{() }\OperatorTok\StringTok{ }\KeywordTok{kable_styling_fc_wide}\NormalTok{()}
\end{Highlighting}
\end{Shaded}

\begin{table}[!h]
\centering
\resizebox{\linewidth}{!}{
\begin{tabular}{l|l|l|r|r|r|r|r|r|r}
\hline
wealthIdx\_Qs0e1n2 & hgt0\_Qs0e1n3 & wgt0\_Qs0e1n3 & group.index & wealthIdx\_mean & hgt0\_mean & wgt0\_mean & wealthIdx\_n & hgt0\_n & wgt0\_n\\
\hline
\rowcolor{gray!6}  [1,7.3]; (1) of Qs0e1n2 & [40.6,48.5]; (1) of Qs0e1n3 & [1.4e+03,2.84e+03]; (1) of Qs0e1n3 & 1 & 5.306477 & 46.56259 & 2497.543 & 3304 & 3304 & 3304\\
\hline
[1,7.3]; (1) of Qs0e1n2 & [40.6,48.5]; (1) of Qs0e1n3 & (2.84e+03,3.21e+03]; (2) of Qs0e1n3 & 2 & 5.077300 & 47.61424 & 2992.620 & 1348 & 1348 & 1348\\
\hline
\rowcolor{gray!6}  [1,7.3]; (1) of Qs0e1n2 & [40.6,48.5]; (1) of Qs0e1n3 & (3.21e+03,5.49e+03]; (3) of Qs0e1n3 & 3 & 3.639226 & 47.74586 & 3428.613 & 362 & 362 & 362\\
\hline
[1,7.3]; (1) of Qs0e1n2 & (48.5,50.2]; (2) of Qs0e1n3 & [1.4e+03,2.84e+03]; (1) of Qs0e1n3 & 4 & 6.042504 & 49.24233 & 2671.040 & 1134 & 1134 & 1134\\
\hline
\rowcolor{gray!6}  [1,7.3]; (1) of Qs0e1n2 & (48.5,50.2]; (2) of Qs0e1n3 & (2.84e+03,3.21e+03]; (2) of Qs0e1n3 & 5 & 5.355494 & 49.34579 & 3030.472 & 2184 & 2184 & 2184\\
\hline
[1,7.3]; (1) of Qs0e1n2 & (48.5,50.2]; (2) of Qs0e1n3 & (3.21e+03,5.49e+03]; (3) of Qs0e1n3 & 6 & 4.360647 & 49.61496 & 3480.880 & 1484 & 1484 & 1484\\
\hline
\rowcolor{gray!6}  [1,7.3]; (1) of Qs0e1n2 & (50.2,58]; (3) of Qs0e1n3 & [1.4e+03,2.84e+03]; (1) of Qs0e1n3 & 7 & 6.254082 & 51.16327 & 2665.767 & 196 & 196 & 196\\
\hline
[1,7.3]; (1) of Qs0e1n2 & (50.2,58]; (3) of Qs0e1n3 & (2.84e+03,3.21e+03]; (2) of Qs0e1n3 & 8 & 5.451433 & 50.96835 & 3047.780 & 1466 & 1466 & 1466\\
\hline
\rowcolor{gray!6}  [1,7.3]; (1) of Qs0e1n2 & (50.2,58]; (3) of Qs0e1n3 & (3.21e+03,5.49e+03]; (3) of Qs0e1n3 & 9 & 4.055986 & 51.83008 & 3660.124 & 3458 & 3458 & 3458\\
\hline
(7.3,19.3]; (2) of Qs0e1n2 & [40.6,48.5]; (1) of Qs0e1n3 & [1.4e+03,2.84e+03]; (1) of Qs0e1n3 & 10 & 9.860733 & 46.79267 & 2539.984 & 3438 & 3438 & 3438\\
\hline
\end{tabular}}
\end{table}

\begin{Shaded}
\begin{Highlighting}[]
\KeywordTok{head}\NormalTok{(df.with.cut.quant.all }\OperatorTok\StringTok{ }
\StringTok{  }\KeywordTok{group_by}\NormalTok{(}\OperatorTok{!!!}\KeywordTok{syms}\NormalTok{(vars.group_by)) }\OperatorTok\StringTok{ }
\StringTok{  }\KeywordTok{arrange}\NormalTok{(}\OperatorTok{!!!}\KeywordTok{syms}\NormalTok{(vars.arrange)) }\OperatorTok\StringTok{ }\KeywordTok{slice}\NormalTok{(1L) }\OperatorTok
\StringTok{  }\KeywordTok{group_by}\NormalTok{(}\OperatorTok{!!!}\KeywordTok{syms}\NormalTok{(vars.quantile.cut.all), }\OperatorTok{!!}\KeywordTok{sym}\NormalTok{(var.qjnt.grp.idx)) }\OperatorTok
\StringTok{  }\KeywordTok{summarise_at}\NormalTok{(vars.cts2quantile, }\KeywordTok{funs}\NormalTok{(mean, }\KeywordTok{n}\NormalTok{())), }\DecValTok{10}\NormalTok{) }\OperatorTok\StringTok{ }
\StringTok{  }\KeywordTok{kable}\NormalTok{() }\OperatorTok\StringTok{ }\KeywordTok{kable_styling_fc_wide}\NormalTok{()}
\end{Highlighting}
\end{Shaded}

\begin{table}[!h]
\centering
\resizebox{\linewidth}{!}{
\begin{tabular}{l|l|l|r|r|r|r|r|r|r}
\hline
wealthIdx\_Qs0e1n2 & hgt0\_Qs0e1n3 & wgt0\_Qs0e1n3 & group.index & wealthIdx\_mean & hgt0\_mean & wgt0\_mean & wealthIdx\_n & hgt0\_n & wgt0\_n\\
\hline
\rowcolor{gray!6}  [1,7.3]; (1) of Qs0e1n2 & [40.6,48.5]; (1) of Qs0e1n3 & [1.4e+03,2.84e+03]; (1) of Qs0e1n3 & 1 & 5.200526 & 46.55632 & 2498.762 & 190 & 190 & 190\\
\hline
[1,7.3]; (1) of Qs0e1n2 & [40.6,48.5]; (1) of Qs0e1n3 & (2.84e+03,3.21e+03]; (2) of Qs0e1n3 & 2 & 4.958974 & 47.60256 & 2992.736 & 78 & 78 & 78\\
\hline
\rowcolor{gray!6}  [1,7.3]; (1) of Qs0e1n2 & [40.6,48.5]; (1) of Qs0e1n3 & (3.21e+03,5.49e+03]; (3) of Qs0e1n3 & 3 & 3.563636 & 47.73182 & 3430.941 & 22 & 22 & 22\\
\hline
[1,7.3]; (1) of Qs0e1n2 & (48.5,50.2]; (2) of Qs0e1n3 & [1.4e+03,2.84e+03]; (1) of Qs0e1n3 & 4 & 5.989063 & 49.24375 & 2671.014 & 64 & 64 & 64\\
\hline
\rowcolor{gray!6}  [1,7.3]; (1) of Qs0e1n2 & (48.5,50.2]; (2) of Qs0e1n3 & (2.84e+03,3.21e+03]; (2) of Qs0e1n3 & 5 & 5.246032 & 49.34603 & 3031.429 & 126 & 126 & 126\\
\hline
[1,7.3]; (1) of Qs0e1n2 & (48.5,50.2]; (2) of Qs0e1n3 & (3.21e+03,5.49e+03]; (3) of Qs0e1n3 & 6 & 4.235227 & 49.61136 & 3484.544 & 88 & 88 & 88\\
\hline
\rowcolor{gray!6}  [1,7.3]; (1) of Qs0e1n2 & (50.2,58]; (3) of Qs0e1n3 & [1.4e+03,2.84e+03]; (1) of Qs0e1n3 & 7 & 6.218182 & 51.15455 & 2665.818 & 11 & 11 & 11\\
\hline
[1,7.3]; (1) of Qs0e1n2 & (50.2,58]; (3) of Qs0e1n3 & (2.84e+03,3.21e+03]; (2) of Qs0e1n3 & 8 & 5.360714 & 50.96905 & 3048.073 & 84 & 84 & 84\\
\hline
\rowcolor{gray!6}  [1,7.3]; (1) of Qs0e1n2 & (50.2,58]; (3) of Qs0e1n3 & (3.21e+03,5.49e+03]; (3) of Qs0e1n3 & 9 & 3.944927 & 51.83623 & 3667.147 & 207 & 207 & 207\\
\hline
(7.3,19.3]; (2) of Qs0e1n2 & [40.6,48.5]; (1) of Qs0e1n3 & [1.4e+03,2.84e+03]; (1) of Qs0e1n3 & 10 & 9.860733 & 46.79267 & 2539.984 & 191 & 191 & 191\\
\hline
\end{tabular}}
\end{table}

\hypertarget{change-values-based-on-index}{%
\subparagraph{Change values Based on
Index}\label{change-values-based-on-index}}

Index from 1 to 18, change input values based on index

\begin{Shaded}
\begin{Highlighting}[]
\CommentTok{# arr.group.idx.subsidy <- arr.group.idx*2 - ((arr.group.idx)^2)*0.01}
\NormalTok{arr.group.idx.subsidy <-}\StringTok{ }\NormalTok{arr.group.idx}\OperatorTok{*}\DecValTok{2}
\KeywordTok{head}\NormalTok{(df.with.cut.quant.all }\OperatorTok
\StringTok{        }\KeywordTok{mutate}\NormalTok{(}\DataTypeTok{more_prot =}\NormalTok{ prot }\OperatorTok{+}\StringTok{ }\NormalTok{arr.group.idx.subsidy[}\OperatorTok{!!}\KeywordTok{sym}\NormalTok{(var.qjnt.grp.idx)]) }\OperatorTok
\StringTok{        }\KeywordTok{group_by}\NormalTok{(}\OperatorTok{!!!}\KeywordTok{syms}\NormalTok{(vars.quantile.cut.all), }\OperatorTok{!!}\KeywordTok{sym}\NormalTok{(var.qjnt.grp.idx))  }\OperatorTok
\StringTok{        }\KeywordTok{summarise_at}\NormalTok{(}\KeywordTok{c}\NormalTok{(}\StringTok{'more_prot'}\NormalTok{, }\StringTok{'prot'}\NormalTok{), }\KeywordTok{funs}\NormalTok{(}\KeywordTok{mean}\NormalTok{(., }\DataTypeTok{na.rm=}\OtherTok{TRUE}\NormalTok{))), }\DecValTok{10}\NormalTok{) }\OperatorTok\StringTok{ }
\StringTok{  }\KeywordTok{kable}\NormalTok{() }\OperatorTok\StringTok{ }\KeywordTok{kable_styling_fc_wide}\NormalTok{()}
\end{Highlighting}
\end{Shaded}

\begin{table}[!h]
\centering
\resizebox{\linewidth}{!}{
\begin{tabular}{l|l|l|r|r|r}
\hline
wealthIdx\_Qs0e1n2 & hgt0\_Qs0e1n3 & wgt0\_Qs0e1n3 & group.index & more\_prot & prot\\
\hline
\rowcolor{gray!6}  [1,7.3]; (1) of Qs0e1n2 & [40.6,48.5]; (1) of Qs0e1n3 & [1.4e+03,2.84e+03]; (1) of Qs0e1n3 & 1 & 14.08242 & 12.08242\\
\hline
[1,7.3]; (1) of Qs0e1n2 & [40.6,48.5]; (1) of Qs0e1n3 & (2.84e+03,3.21e+03]; (2) of Qs0e1n3 & 2 & 15.89847 & 11.89847\\
\hline
\rowcolor{gray!6}  [1,7.3]; (1) of Qs0e1n2 & [40.6,48.5]; (1) of Qs0e1n3 & (3.21e+03,5.49e+03]; (3) of Qs0e1n3 & 3 & 27.15484 & 21.15484\\
\hline
[1,7.3]; (1) of Qs0e1n2 & (48.5,50.2]; (2) of Qs0e1n3 & [1.4e+03,2.84e+03]; (1) of Qs0e1n3 & 4 & 18.90528 & 10.90528\\
\hline
\rowcolor{gray!6}  [1,7.3]; (1) of Qs0e1n2 & (48.5,50.2]; (2) of Qs0e1n3 & (2.84e+03,3.21e+03]; (2) of Qs0e1n3 & 5 & 22.32498 & 12.32498\\
\hline
[1,7.3]; (1) of Qs0e1n2 & (48.5,50.2]; (2) of Qs0e1n3 & (3.21e+03,5.49e+03]; (3) of Qs0e1n3 & 6 & 28.63120 & 16.63120\\
\hline
\rowcolor{gray!6}  [1,7.3]; (1) of Qs0e1n2 & (50.2,58]; (3) of Qs0e1n3 & [1.4e+03,2.84e+03]; (1) of Qs0e1n3 & 7 & 25.47638 & 11.47638\\
\hline
[1,7.3]; (1) of Qs0e1n2 & (50.2,58]; (3) of Qs0e1n3 & (2.84e+03,3.21e+03]; (2) of Qs0e1n3 & 8 & 28.02607 & 12.02607\\
\hline
\rowcolor{gray!6}  [1,7.3]; (1) of Qs0e1n2 & (50.2,58]; (3) of Qs0e1n3 & (3.21e+03,5.49e+03]; (3) of Qs0e1n3 & 9 & 34.69356 & 16.69356\\
\hline
(7.3,19.3]; (2) of Qs0e1n2 & [40.6,48.5]; (1) of Qs0e1n3 & [1.4e+03,2.84e+03]; (1) of Qs0e1n3 & 10 & 30.73473 & 10.73473\\
\hline
\end{tabular}}
\end{table}

\end{document}
