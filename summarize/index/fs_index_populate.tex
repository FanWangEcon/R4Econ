% Options for packages loaded elsewhere
\PassOptionsToPackage{unicode}{hyperref}
\PassOptionsToPackage{hyphens}{url}
\PassOptionsToPackage{dvipsnames,svgnames*,x11names*}{xcolor}
%
\documentclass[
]{article}
\usepackage{lmodern}
\usepackage{amssymb,amsmath}
\usepackage{ifxetex,ifluatex}
\ifnum 0\ifxetex 1\fi\ifluatex 1\fi=0 % if pdftex
  \usepackage[T1]{fontenc}
  \usepackage[utf8]{inputenc}
  \usepackage{textcomp} % provide euro and other symbols
\else % if luatex or xetex
  \usepackage{unicode-math}
  \defaultfontfeatures{Scale=MatchLowercase}
  \defaultfontfeatures[\rmfamily]{Ligatures=TeX,Scale=1}
\fi
% Use upquote if available, for straight quotes in verbatim environments
\IfFileExists{upquote.sty}{\usepackage{upquote}}{}
\IfFileExists{microtype.sty}{% use microtype if available
  \usepackage[]{microtype}
  \UseMicrotypeSet[protrusion]{basicmath} % disable protrusion for tt fonts
}{}
\makeatletter
\@ifundefined{KOMAClassName}{% if non-KOMA class
  \IfFileExists{parskip.sty}{%
    \usepackage{parskip}
  }{% else
    \setlength{\parindent}{0pt}
    \setlength{\parskip}{6pt plus 2pt minus 1pt}}
}{% if KOMA class
  \KOMAoptions{parskip=half}}
\makeatother
\usepackage{xcolor}
\IfFileExists{xurl.sty}{\usepackage{xurl}}{} % add URL line breaks if available
\IfFileExists{bookmark.sty}{\usepackage{bookmark}}{\usepackage{hyperref}}
\hypersetup{
  pdftitle={R Example DPLYR Generate Sorted Index, Ordinal Deviation Negative and Positive Index, and Expand Value from Lowest Index to All Rows},
  pdfauthor={Fan Wang},
  colorlinks=true,
  linkcolor=Maroon,
  filecolor=Maroon,
  citecolor=Blue,
  urlcolor=blue,
  pdfcreator={LaTeX via pandoc}}
\urlstyle{same} % disable monospaced font for URLs
\usepackage[margin=1in]{geometry}
\usepackage{color}
\usepackage{fancyvrb}
\newcommand{\VerbBar}{|}
\newcommand{\VERB}{\Verb[commandchars=\\\{\}]}
\DefineVerbatimEnvironment{Highlighting}{Verbatim}{commandchars=\\\{\}}
% Add ',fontsize=\small' for more characters per line
\usepackage{framed}
\definecolor{shadecolor}{RGB}{248,248,248}
\newenvironment{Shaded}{\begin{snugshade}}{\end{snugshade}}
\newcommand{\AlertTok}[1]{\textcolor[rgb]{0.94,0.16,0.16}{#1}}
\newcommand{\AnnotationTok}[1]{\textcolor[rgb]{0.56,0.35,0.01}{\textbf{\textit{#1}}}}
\newcommand{\AttributeTok}[1]{\textcolor[rgb]{0.77,0.63,0.00}{#1}}
\newcommand{\BaseNTok}[1]{\textcolor[rgb]{0.00,0.00,0.81}{#1}}
\newcommand{\BuiltInTok}[1]{#1}
\newcommand{\CharTok}[1]{\textcolor[rgb]{0.31,0.60,0.02}{#1}}
\newcommand{\CommentTok}[1]{\textcolor[rgb]{0.56,0.35,0.01}{\textit{#1}}}
\newcommand{\CommentVarTok}[1]{\textcolor[rgb]{0.56,0.35,0.01}{\textbf{\textit{#1}}}}
\newcommand{\ConstantTok}[1]{\textcolor[rgb]{0.00,0.00,0.00}{#1}}
\newcommand{\ControlFlowTok}[1]{\textcolor[rgb]{0.13,0.29,0.53}{\textbf{#1}}}
\newcommand{\DataTypeTok}[1]{\textcolor[rgb]{0.13,0.29,0.53}{#1}}
\newcommand{\DecValTok}[1]{\textcolor[rgb]{0.00,0.00,0.81}{#1}}
\newcommand{\DocumentationTok}[1]{\textcolor[rgb]{0.56,0.35,0.01}{\textbf{\textit{#1}}}}
\newcommand{\ErrorTok}[1]{\textcolor[rgb]{0.64,0.00,0.00}{\textbf{#1}}}
\newcommand{\ExtensionTok}[1]{#1}
\newcommand{\FloatTok}[1]{\textcolor[rgb]{0.00,0.00,0.81}{#1}}
\newcommand{\FunctionTok}[1]{\textcolor[rgb]{0.00,0.00,0.00}{#1}}
\newcommand{\ImportTok}[1]{#1}
\newcommand{\InformationTok}[1]{\textcolor[rgb]{0.56,0.35,0.01}{\textbf{\textit{#1}}}}
\newcommand{\KeywordTok}[1]{\textcolor[rgb]{0.13,0.29,0.53}{\textbf{#1}}}
\newcommand{\NormalTok}[1]{#1}
\newcommand{\OperatorTok}[1]{\textcolor[rgb]{0.81,0.36,0.00}{\textbf{#1}}}
\newcommand{\OtherTok}[1]{\textcolor[rgb]{0.56,0.35,0.01}{#1}}
\newcommand{\PreprocessorTok}[1]{\textcolor[rgb]{0.56,0.35,0.01}{\textit{#1}}}
\newcommand{\RegionMarkerTok}[1]{#1}
\newcommand{\SpecialCharTok}[1]{\textcolor[rgb]{0.00,0.00,0.00}{#1}}
\newcommand{\SpecialStringTok}[1]{\textcolor[rgb]{0.31,0.60,0.02}{#1}}
\newcommand{\StringTok}[1]{\textcolor[rgb]{0.31,0.60,0.02}{#1}}
\newcommand{\VariableTok}[1]{\textcolor[rgb]{0.00,0.00,0.00}{#1}}
\newcommand{\VerbatimStringTok}[1]{\textcolor[rgb]{0.31,0.60,0.02}{#1}}
\newcommand{\WarningTok}[1]{\textcolor[rgb]{0.56,0.35,0.01}{\textbf{\textit{#1}}}}
\usepackage{graphicx,grffile}
\makeatletter
\def\maxwidth{\ifdim\Gin@nat@width>\linewidth\linewidth\else\Gin@nat@width\fi}
\def\maxheight{\ifdim\Gin@nat@height>\textheight\textheight\else\Gin@nat@height\fi}
\makeatother
% Scale images if necessary, so that they will not overflow the page
% margins by default, and it is still possible to overwrite the defaults
% using explicit options in \includegraphics[width, height, ...]{}
\setkeys{Gin}{width=\maxwidth,height=\maxheight,keepaspectratio}
% Set default figure placement to htbp
\makeatletter
\def\fps@figure{htbp}
\makeatother
\setlength{\emergencystretch}{3em} % prevent overfull lines
\providecommand{\tightlist}{%
  \setlength{\itemsep}{0pt}\setlength{\parskip}{0pt}}
\setcounter{secnumdepth}{-\maxdimen} % remove section numbering

\title{R Example DPLYR Generate Sorted Index, Ordinal Deviation Negative and
Positive Index, and Expand Value from Lowest Index to All Rows}
\author{Fan Wang}
\date{}

\begin{document}
\maketitle

Go back to \href{http://fanwangecon.github.io/}{fan}'s
\href{https://fanwangecon.github.io/REconTools/}{REconTools} Package,
\href{https://fanwangecon.github.io/R4Econ/}{R4Econ} Repository, or
\href{https://fanwangecon.github.io/Stat4Econ/}{Intro Stats with R}
Repository.

\begin{Shaded}
\begin{Highlighting}[]
\KeywordTok{rm}\NormalTok{(}\DataTypeTok{list =} \KeywordTok{ls}\NormalTok{(}\DataTypeTok{all.names =} \OtherTok{TRUE}\NormalTok{))}
\KeywordTok{options}\NormalTok{(}\DataTypeTok{knitr.duplicate.label =} \StringTok{'allow'}\NormalTok{)}
\end{Highlighting}
\end{Shaded}

\begin{Shaded}
\begin{Highlighting}[]
\KeywordTok{library}\NormalTok{(tidyverse)}
\KeywordTok{library}\NormalTok{(knitr)}
\KeywordTok{library}\NormalTok{(kableExtra)}
\KeywordTok{library}\NormalTok{(REconTools)}
\CommentTok{# file name}
\NormalTok{st_file_name =}\StringTok{ 'fs_index_populate'}
\CommentTok{# Generate R File}
\KeywordTok{try}\NormalTok{(}\KeywordTok{purl}\NormalTok{(}\KeywordTok{paste0}\NormalTok{(st_file_name, }\StringTok{".Rmd"}\NormalTok{), }\DataTypeTok{output=}\KeywordTok{paste0}\NormalTok{(st_file_name, }\StringTok{".R"}\NormalTok{), }\DataTypeTok{documentation =} \DecValTok{2}\NormalTok{))}
\CommentTok{# Generate PDF and HTML}
\CommentTok{# rmarkdown::render("C:/Users/fan/R4Econ/summarize/index/fs_index_populate.Rmd", "pdf_document")}
\CommentTok{# rmarkdown::render("C:/Users/fan/R4Econ/summarize/index/fs_index_populate.Rmd", "html_document")}
\end{Highlighting}
\end{Shaded}

\hypertarget{generate-sorted-index-within-group-and-spread}{%
\subsection{Generate Sorted Index within Group and
Spread}\label{generate-sorted-index-within-group-and-spread}}

\hypertarget{generate-sorted-index-within-group-with-repeating-values}{%
\subsubsection{Generate Sorted Index within Group with Repeating
Values}\label{generate-sorted-index-within-group-with-repeating-values}}

There is a variable, sort by this variable, then generate index from 1
to N representing sorted values of this index. If there are repeating
values, still assign index, different index each value.

\begin{itemize}
\tightlist
\item
  r generate index sort
\item
  dplyr mutate equals index
\end{itemize}

\begin{Shaded}
\begin{Highlighting}[]
\CommentTok{# Sort and generate variable equal to sorted index}
\NormalTok{df_iris <-}\StringTok{ }\NormalTok{iris }\OperatorTok\StringTok{ }\KeywordTok{arrange}\NormalTok{(Sepal.Length) }\OperatorTok
\StringTok{              }\KeywordTok{mutate}\NormalTok{(}\DataTypeTok{Sepal.Len.Index =} \KeywordTok{row_number}\NormalTok{()) }\OperatorTok
\StringTok{              }\KeywordTok{select}\NormalTok{(Sepal.Length, Sepal.Len.Index, }\KeywordTok{everything}\NormalTok{())}

\CommentTok{# Show results Head 10}
\NormalTok{df_iris }\OperatorTok\StringTok{ }\KeywordTok{head}\NormalTok{(}\DecValTok{10}\NormalTok{) }\OperatorTok
\StringTok{  }\KeywordTok{kable}\NormalTok{() }\OperatorTok
\StringTok{  }\KeywordTok{kable_styling}\NormalTok{(}\DataTypeTok{bootstrap_options =} \KeywordTok{c}\NormalTok{(}\StringTok{"striped"}\NormalTok{, }\StringTok{"hover"}\NormalTok{, }\StringTok{"condensed"}\NormalTok{, }\StringTok{"responsive"}\NormalTok{))}
\end{Highlighting}
\end{Shaded}

Sepal.Length

Sepal.Len.Index

Sepal.Width

Petal.Length

Petal.Width

Species

4.3

1

3.0

1.1

0.1

setosa

4.4

2

2.9

1.4

0.2

setosa

4.4

3

3.0

1.3

0.2

setosa

4.4

4

3.2

1.3

0.2

setosa

4.5

5

2.3

1.3

0.3

setosa

4.6

6

3.1

1.5

0.2

setosa

4.6

7

3.4

1.4

0.3

setosa

4.6

8

3.6

1.0

0.2

setosa

4.6

9

3.2

1.4

0.2

setosa

4.7

10

3.2

1.3

0.2

setosa

\hypertarget{populate-value-from-lowest-index-to-all-other-rows}{%
\subsubsection{Populate Value from Lowest Index to All other
Rows}\label{populate-value-from-lowest-index-to-all-other-rows}}

We would like to calculate for example the ratio of each individual's
highest to the the person with the lowest height in a dataset. We first
need to generated sorted index from lowest to highest, and then populate
the lowest height to all rows, and then divide.

\emph{Search Terms}:

\begin{itemize}
\tightlist
\item
  r spread value to all rows from one row
\item
  r other rows equal to the value of one row
\item
  Conditional assignment of one variable to the value of one of two
  other variables
\item
  dplyr mutate conditional
\item
  dplyr value from one row to all rows
\item
  dplyr mutate equal to value in another cell
\end{itemize}

\emph{Links}:

\begin{itemize}
\tightlist
\item
  see: dplyr
  \href{https://dplyr.tidyverse.org/reference/ranking.html}{rank}
\item
  see: dplyr
  \href{https://dplyr.tidyverse.org/reference/case_when.html}{case\_when}
\end{itemize}

\hypertarget{short-method-mutate-and-min}{%
\paragraph{Short Method: mutate and
min}\label{short-method-mutate-and-min}}

We just want the lowest value to be in its own column, so that we can
compute various statistics using the lowest value variable and the
original variable.

\begin{Shaded}
\begin{Highlighting}[]
\CommentTok{# 1. Sort}
\NormalTok{df_iris_m1 <-}\StringTok{ }\NormalTok{iris }\OperatorTok\StringTok{ }\KeywordTok{mutate}\NormalTok{(}\DataTypeTok{Sepal.Len.Lowest.all =} \KeywordTok{min}\NormalTok{(Sepal.Length)) }\OperatorTok
\StringTok{                }\KeywordTok{select}\NormalTok{(Sepal.Length, Sepal.Len.Lowest.all, }\KeywordTok{everything}\NormalTok{())}


\CommentTok{# Show results Head 10}
\NormalTok{df_iris_m1 }\OperatorTok\StringTok{ }\KeywordTok{head}\NormalTok{(}\DecValTok{10}\NormalTok{) }\OperatorTok
\StringTok{  }\KeywordTok{kable}\NormalTok{() }\OperatorTok
\StringTok{  }\KeywordTok{kable_styling}\NormalTok{(}\DataTypeTok{bootstrap_options =} \KeywordTok{c}\NormalTok{(}\StringTok{"striped"}\NormalTok{, }\StringTok{"hover"}\NormalTok{, }\StringTok{"condensed"}\NormalTok{, }\StringTok{"responsive"}\NormalTok{))}
\end{Highlighting}
\end{Shaded}

Sepal.Length

Sepal.Len.Lowest.all

Sepal.Width

Petal.Length

Petal.Width

Species

5.1

4.3

3.5

1.4

0.2

setosa

4.9

4.3

3.0

1.4

0.2

setosa

4.7

4.3

3.2

1.3

0.2

setosa

4.6

4.3

3.1

1.5

0.2

setosa

5.0

4.3

3.6

1.4

0.2

setosa

5.4

4.3

3.9

1.7

0.4

setosa

4.6

4.3

3.4

1.4

0.3

setosa

5.0

4.3

3.4

1.5

0.2

setosa

4.4

4.3

2.9

1.4

0.2

setosa

4.9

4.3

3.1

1.5

0.1

setosa

\hypertarget{long-method-row_number-and-case_when}{%
\paragraph{Long Method: row\_number and
case\_when}\label{long-method-row_number-and-case_when}}

This is the long method, using row\_number, and case\_when. The benefit
of this method is that it generates several intermediate variables that
might be useful. And the key final step is to set a new variable
(A=\emph{Sepal.Len.Lowest.all}) equal to another variable's
(B=\emph{Sepal.Length}'s) value at the index that satisfies condition
based a third variable (C=\emph{Sepal.Len.Index}).

\begin{Shaded}
\begin{Highlighting}[]
\CommentTok{# 1. Sort}
\CommentTok{# 2. generate index}
\CommentTok{# 3. value at lowest index (case_when)}
\CommentTok{# 4. spread value from lowest index to other rows}
\CommentTok{# Note step 4 does not require step 3}
\NormalTok{df_iris_m2 <-}\StringTok{ }\NormalTok{iris }\OperatorTok\StringTok{ }\KeywordTok{arrange}\NormalTok{(Sepal.Length) }\OperatorTok
\StringTok{              }\KeywordTok{mutate}\NormalTok{(}\DataTypeTok{Sepal.Len.Index =} \KeywordTok{row_number}\NormalTok{()) }\OperatorTok
\StringTok{              }\KeywordTok{mutate}\NormalTok{(}\DataTypeTok{Sepal.Len.Lowest.one =}
                       \KeywordTok{case_when}\NormalTok{(}\KeywordTok{row_number}\NormalTok{()}\OperatorTok{==}\DecValTok{1} \OperatorTok{~}\StringTok{ }\NormalTok{Sepal.Length)) }\OperatorTok
\StringTok{              }\KeywordTok{mutate}\NormalTok{(}\DataTypeTok{Sepal.Len.Lowest.all =}
\NormalTok{                       Sepal.Length[Sepal.Len.Index}\OperatorTok{==}\DecValTok{1}\NormalTok{]) }\OperatorTok
\StringTok{              }\KeywordTok{select}\NormalTok{(Sepal.Length, Sepal.Len.Index,}
\NormalTok{                     Sepal.Len.Lowest.one, Sepal.Len.Lowest.all)}


\CommentTok{# Show results Head 10}
\NormalTok{df_iris_m2 }\OperatorTok\StringTok{ }\KeywordTok{head}\NormalTok{(}\DecValTok{10}\NormalTok{) }\OperatorTok
\StringTok{  }\KeywordTok{kable}\NormalTok{() }\OperatorTok
\StringTok{  }\KeywordTok{kable_styling}\NormalTok{(}\DataTypeTok{bootstrap_options =} \KeywordTok{c}\NormalTok{(}\StringTok{"striped"}\NormalTok{, }\StringTok{"hover"}\NormalTok{, }\StringTok{"condensed"}\NormalTok{, }\StringTok{"responsive"}\NormalTok{))}
\end{Highlighting}
\end{Shaded}

Sepal.Length

Sepal.Len.Index

Sepal.Len.Lowest.one

Sepal.Len.Lowest.all

4.3

1

4.3

4.3

4.4

2

NA

4.3

4.4

3

NA

4.3

4.4

4

NA

4.3

4.5

5

NA

4.3

4.6

6

NA

4.3

4.6

7

NA

4.3

4.6

8

NA

4.3

4.6

9

NA

4.3

4.7

10

NA

4.3

\hypertarget{generate-sorted-index-based-on-deviations}{%
\subsection{Generate Sorted Index based on
Deviations}\label{generate-sorted-index-based-on-deviations}}

\hypertarget{generate-positive-and-negative-index-based-on-ordered-deviation-from-some-number}{%
\subsubsection{Generate Positive and Negative Index based on Ordered
Deviation from some
Number}\label{generate-positive-and-negative-index-based-on-ordered-deviation-from-some-number}}

There is a variable that is continuous, substract a number from this
variable, and generate index based on deviations. Think of the index as
generating intervals indicating where the value lies. 0th index
indicates the largest value in sequence that is smaller than or equal to
number \(x\), 1st index indicates the smallest value in sequence that is
larger than number \(x\).

The solution below is a little bit convoluated and long, there is likely
a much quicker way. The process below shows various intermediary outputs
that help arrive at deviation index \emph{Sepal.Len.Devi.Index} from
initial sorted index \emph{Sepal.Len.Index}.

\emph{search}:

\begin{itemize}
\tightlist
\item
  dplyr arrange ignore na
\item
  dplyr index deviation from order number sequence
\item
  dplyr index below above
\item
  dplyr index order below above value
\end{itemize}

\begin{Shaded}
\begin{Highlighting}[]
\CommentTok{# 1. Sort and generate variable equal to sorted index}
\CommentTok{# 2. Plus or minus deviations from some value}
\CommentTok{# 3. Find the zero, which means, the number closests to zero including zero from the negative side}
\CommentTok{# 4. Find the index at the highest zero and below deviation point}
\CommentTok{# 5. Difference of zero index and original sorted index}
\NormalTok{sc_val_x <-}\StringTok{ }\FloatTok{4.65}
\NormalTok{df_iris_deviate <-}\StringTok{ }\NormalTok{iris }\OperatorTok\StringTok{ }\KeywordTok{arrange}\NormalTok{(Sepal.Length) }\OperatorTok
\StringTok{              }\KeywordTok{mutate}\NormalTok{(}\DataTypeTok{Sepal.Len.Index =} \KeywordTok{row_number}\NormalTok{()) }\OperatorTok
\StringTok{              }\KeywordTok{mutate}\NormalTok{(}\DataTypeTok{Sepal.Len.Devi =}\NormalTok{ (Sepal.Length }\OperatorTok{-}\StringTok{ }\NormalTok{sc_val_x)) }\OperatorTok
\StringTok{              }\KeywordTok{mutate}\NormalTok{(}\DataTypeTok{Sepal.Len.Devi.Neg =}
                       \KeywordTok{case_when}\NormalTok{(Sepal.Len.Devi }\OperatorTok{<=}\StringTok{ }\DecValTok{0} \OperatorTok{~}\StringTok{ }\NormalTok{(}\OperatorTok{-}\DecValTok{1}\NormalTok{)}\OperatorTok{*}\NormalTok{(Sepal.Len.Devi))) }\OperatorTok
\StringTok{              }\KeywordTok{arrange}\NormalTok{((Sepal.Len.Devi.Neg), }\KeywordTok{desc}\NormalTok{(Sepal.Len.Index)) }\OperatorTok
\StringTok{              }\KeywordTok{mutate}\NormalTok{(}\DataTypeTok{Sepal.Len.Index.Zero =}
                       \KeywordTok{case_when}\NormalTok{(}\KeywordTok{row_number}\NormalTok{() }\OperatorTok{==}\StringTok{ }\DecValTok{1} \OperatorTok{~}\StringTok{ }\NormalTok{Sepal.Len.Index)) }\OperatorTok
\StringTok{              }\KeywordTok{mutate}\NormalTok{(}\DataTypeTok{Sepal.Len.Devi.Index =}
\NormalTok{                       Sepal.Len.Index }\OperatorTok{-}\StringTok{ }\NormalTok{Sepal.Len.Index.Zero[}\KeywordTok{row_number}\NormalTok{() }\OperatorTok{==}\StringTok{ }\DecValTok{1}\NormalTok{]) }\OperatorTok
\StringTok{              }\KeywordTok{arrange}\NormalTok{(Sepal.Len.Index) }\OperatorTok
\StringTok{              }\KeywordTok{select}\NormalTok{(Sepal.Length, Sepal.Len.Index, Sepal.Len.Devi,}
\NormalTok{                     Sepal.Len.Devi.Neg, Sepal.Len.Index.Zero, Sepal.Len.Devi.Index)}


\CommentTok{# Show results Head 10}
\NormalTok{df_iris_deviate }\OperatorTok\StringTok{ }\KeywordTok{head}\NormalTok{(}\DecValTok{20}\NormalTok{) }\OperatorTok
\StringTok{  }\KeywordTok{kable}\NormalTok{() }\OperatorTok
\StringTok{  }\KeywordTok{kable_styling}\NormalTok{(}\DataTypeTok{bootstrap_options =} \KeywordTok{c}\NormalTok{(}\StringTok{"striped"}\NormalTok{, }\StringTok{"hover"}\NormalTok{, }\StringTok{"condensed"}\NormalTok{, }\StringTok{"responsive"}\NormalTok{))}
\end{Highlighting}
\end{Shaded}

Sepal.Length

Sepal.Len.Index

Sepal.Len.Devi

Sepal.Len.Devi.Neg

Sepal.Len.Index.Zero

Sepal.Len.Devi.Index

4.3

1

-0.35

0.35

NA

-8

4.4

2

-0.25

0.25

NA

-7

4.4

3

-0.25

0.25

NA

-6

4.4

4

-0.25

0.25

NA

-5

4.5

5

-0.15

0.15

NA

-4

4.6

6

-0.05

0.05

NA

-3

4.6

7

-0.05

0.05

NA

-2

4.6

8

-0.05

0.05

NA

-1

4.6

9

-0.05

0.05

9

0

4.7

10

0.05

NA

NA

1

4.7

11

0.05

NA

NA

2

4.8

12

0.15

NA

NA

3

4.8

13

0.15

NA

NA

4

4.8

14

0.15

NA

NA

5

4.8

15

0.15

NA

NA

6

4.8

16

0.15

NA

NA

7

4.9

17

0.25

NA

NA

8

4.9

18

0.25

NA

NA

9

4.9

19

0.25

NA

NA

10

4.9

20

0.25

NA

NA

11

\end{document}
